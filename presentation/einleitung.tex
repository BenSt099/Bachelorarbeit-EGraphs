\section{Einleitung}

\subsection{Foliengestaltung}

\begin{frame}
  \frametitle{Aufz\"ahlungen}
  Hier beginnt eine ganz normale Aufzählung:
  \begin{itemize}
    \item Mit ein paar Unterpunkten,
    \begin{itemize}
      \item die wiederum Unterpunkte haben können.
      \item und so weiter
    \end{itemize}
    \item Das ist der nächste Unterpunkt
    \item Wie man sieht, werden die Texte, die auf einer Folie stehen, vertikal zentriert.
  \end{itemize}
\end{frame}


\begin{frame}
  \frametitle{Blöcke}

  Sehr praktisch ist die Verwendung von Blöcken:
  \begin{block}{Normaler Blocktitel}
    Blöcke sind zur Hervorhebung gedacht. In normalen Blöcken können
    wichtige Erkenntnisse (Zwischenergebnisse) stehen, die nicht
    unbemerkt bleiben dürfen.
  \end{block}

  \begin{exampleblock}{Beispiel-Blocktitel}
    Diese Blöcke sind für Beispiele o.ä. gedacht.
  \end{exampleblock}

  \begin{alertblock}{Alarm-Blocktitel}
    Diese Blöcke beinhalten für gewöhnlich Problembeschreibungen.
  \end{alertblock}
\end{frame}

\begin{frame}
  \frametitle{Weitere Hilfen}

  Es gibt viele Quellen, die für \LaTeX Beamer herangezogen werden können. Besonders gut sind:
  \begin{itemize}
    \item Der Beamer User Guide von Till Tantau:\\\texttt{http://www.math.binghamton.edu/erik/beameruserguide.pdf}
    \item Ein sehr gutes Beamer Tutorial von Ki-Joo Kim:\\\texttt{http://saikat.guha.cc/ref/beamer\_guide.pdf}
    \item Der jeweilige Betreuer der Abschlussarbeit :-)
  \end{itemize}

\end{frame}

\subsection{subsections bitte meiden, sie werden nur in Gliederungen angezeigt.}
