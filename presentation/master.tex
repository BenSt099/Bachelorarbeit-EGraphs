%----------------------------------------------------------------------------------------
%	PACKAGES AND THEMES
%----------------------------------------------------------------------------------------
\documentclass[aspectratio=169,xcolor=dvipsnames]{beamer}
\usetheme{SimplePlus}

\usepackage{hyperref}
\usepackage{color}
\usepackage{graphicx} % Allows including images
\usepackage{subcaption} % Subfigures
\usepackage{booktabs} % Allows the use of \toprule, \midrule and \bottomrule in tables
\usepackage[ngerman]{babel}
\setbeamertemplate{bibliography item}{\insertbiblabel}
\usepackage{algpseudocodex}

\algrenewcommand\algorithmicrequire{\textbf{Voraussetzung:}}
\algrenewcommand\algorithmicfunction{\textbf{Funktion}}

\usepackage{datetime}
\newdate{vortragsdatum}{01}{04}{2025}

\usepackage{bootstrapcolors}

%----------------------------------------------------------------------------------------
%	TITLE PAGE
%----------------------------------------------------------------------------------------

\title[short title]{\vspace{-10mm}{\bf\normalsize \color{black} Bachelorarbeit} \vspace{3mm}\\ \\ \normalsize Entwicklung einer Anwendung zum Vermitteln von Lerninhalten über E-Graphs und Equality Saturation für die Lehre} % The short title appears at the bottom of every slide, the full title is only on the title page
%\subtitle{Subtitle}

\author[rjb] {\bf Ben Steinhauer} 

\institute[HHU] % Your institution as it will appear on the bottom of every slide, may be shorthand to save space
{
    Lehrstuhl für Softwaretechnik und Programmiersprachen \\
    Heinrich-Heine-Universität Düsseldorf% Your institution for the title page
}
%\date{\today} % Date, can be changed to a custom date
\date{\displaydate{vortragsdatum}}

\definecolor{lightblue1}{rgb}{0.92, 0.92, 0.97}
\definecolor{hhuUniBlau}{RGB}{0,106,179}

%----------------------------------------------------------------------------------------
%	CHAPTER TITLE SLIDES
%----------------------------------------------------------------------------------------

% Define a flag to control the display of the title slide
\newif\ifshowtitleslide

% Set the default behavior to not show the title slide
\showtitleslidefalse

% Redefine the \AtBeginSection command to check the flag
\AtBeginSection[]{
  \ifshowtitleslide
    \begin{frame}
      \vfill
      \centering
      \begin{beamercolorbox}[sep=8pt,center,shadow=true,rounded=true]{title}
        \usebeamerfont{title}\insertsectionhead\par%
      \end{beamercolorbox}
      \vfill
    \end{frame}
  \fi
  % Reset the flag to the default state after each section
  \showtitleslidefalse
}

% New command to enable the title slide for the next section
\newcommand{\EnableTitleSlide}{\showtitleslidetrue}

%----------------------------------------------------------------------------------------
%	PRESENTATION SLIDES
%----------------------------------------------------------------------------------------

\begin{document}

\begin{frame}
    \thispagestyle{empty}
    % Print the title page as the first slide
    %\titlepage
    \centering

    {\bf\color{black} Bachelorarbeit}\vspace{4mm}

    {\color{hhuUniBlau}\large Entwicklung einer Anwendung zum Vermitteln von Lerninhalten über E-Graphs und Equality Saturation für die Lehre}\vspace{4mm}

    {\color{hhuUniBlau} Ben Steinhauer}\vspace{3mm}

    {\footnotesize Lehrstuhl für Softwaretechnik und Programmiersprachen}\vspace{-1mm}

    {\footnotesize Heinrich-Heine-Universität Düsseldorf}\vspace{4mm}

    {\footnotesize 01.04.2025}

\end{frame}

%\begin{frame}{Überblick}
    % Throughout your presentation, if you choose to use \section{} and \subsection{} commands, these will automatically be printed on this slide as an overview of your presentation
    \tableofcontents
\end{frame}

\section{Einleitung}\label{sec:einleitung}

\subsection{Motivation}

\noindent In der Informatik nimmt das Optimieren von Ausdrücken eine wichtige Rolle ein. 
Ausdrücke können in Form von mathematischen Termen vorkommen oder als Programmcode auftreten.
Häufig ist der Compiler für diese Aufgabe zuständig, da manche Optimierungen nicht durch den Programmierer vorgenommen werden können. 
Der Compiler kann verschiedene Techniken anwenden, zum Beispiel Teile des Zwischencodes (engl. \textit{Intermediate Representation}) durch effizientere zu ersetzen.
Findet diese Ersetzung sequentiell statt, tritt das sogenannte \textit{Phase Ordering Problem} auf. Dabei endet die Kette an Optimierungen in einem lokalen Optimum~\cite{phaseorder-2009}.
Neben Backtracking- und Brute-Force-Methoden sind \textit{E-Graphs} eine Möglichkeit, dieses Problem zu umgehen. 

\noindent Die Abkürzung \textit{E-Graph} steht für \textit{Equality-} oder \textit{Equivalence-Graph} und beschreibt eine Datenstruktur, die es
erlaubt, äquivalente Ausdrücke kompakt abzuspeichern.
Ursprünglich wurden sie in der automatisierten Beweisführung eingesetzt~\cite{2021-egg}. Heute findet sie, auch dank \textit{Equality Saturation}, Anwendung in Compilern.
Als Beispiel dient \textit{Cranelift}~\footnote{\hspace{1.5mm}\url{https://cranelift.dev/}}, ein Compiler Backend, das \textit{E-Graphs} für die Optimierung verwendet.

\noindent \textit{E-Graphs} und \textit{Equality Saturation} sind nicht-triviale Konzepte, die indessen an Wichtigkeit zunehmen. 
Das schrittweise Heranführen von Studentinnen und Studenten an diese Themen erfordert eine Methode, die interaktiv funktioniert, da das Anwenden des Gelernten 
das Erlernen von Konzepten erleichtert. Damit soll sich diese Bachelorarbeit beschäftigen. 

\subsection{Ziel der Arbeit}

\vspace{5mm}
\begin{center}
    {\itshape
    \rmfamily
    \glqq Show me and I forget.
    Teach me and I remember. 
    Involve me and I learn.\grqq}
    \vspace{-3mm}
    \begin{flushright}
        \footnotesize
        --- Vermeintlich 
        Benjamin Franklin \\
        zugeschrieben
    \end{flushright}
\end{center}\vspace{3mm}

Das Ziel dieser Bachelorarbeit ist es, eine Anwendung für die Lehre zu entwickeln, die Studentinnen und Studenten Wissen zu den Themen \textit{E-Graphs} und \textit{Equality Saturation}
vermittelt. Dabei sollen sie die Möglichkeit haben, sich sowohl auf theoretischer als auch praktischer Ebene mit den Themen auseinander setzen zu können.
Die Grundlage der theoretischen Ebene bildet diese Arbeit, in der notwendige Hintergrundkenntnisse erarbeitet werden. Außerdem wird ein Einblick in die Implementierung gegeben. 
Die praktische Ebene besteht aus der Anwendung, mit deren Hilfe Schritt für Schritt aufgezeigt wird, wie \textit{E-Graphs} und \textit{Equality Saturation} funktionieren.
Für grö{\ss}tmöglichen Nutzen soll die Anwendung zudem plattformunabhängig sein und möglichst nur von \textit{Open-Source-Software} (OSS) Gebrauch machen.
Damit wird das Problem der unterschiedlichen Betriebssysteme der Studenten umgangen und zeitgleich die Hürden für Erweiterungen gesenkt.

\subsection{Verwandte Arbeiten}\label{sub:verwandtearbeiten}

\noindent\textbf{egg} Das Akronym \textit{egg} bildet sich aus den Wörtern \textit{e-graphs good} und ist eine in der Programmiersprache \textit{Rust} geschriebene Bibliothek.
Die Bibliothek implementiert die Datenstruktur \textit{E-Graph} und stellt Funktionen zur Manipulation dieser zur Verfügung.
Neben der Erzeugung von \textit{E-Graphs} kann auch \textit{Equality Saturation} auf diesen ausgeführt werden.
Das gesamte Projekt wurde in einem Paper vorgestellt, auf das sich diese Arbeit teilweise stützt~\cite{2021-egg}.

\noindent\textbf{egglog} Das weiterführende System \textit{egglog} basiert auf \textit{egg} und \textit{Datalog}. 
\textit{Datalog} ist eine \glqq [...] rekursive Abfragesprache für Datenbanken [...] \grqq~\cite{2023-egglog}.
Es gibt eine Webdemo, mit der das System ausprobiert werden kann~\footnote{\hspace{1.5mm}\url{https://egraphs-good.github.io/egglog/}}.
Auch dieses Projekt wurde in einem Paper vorgestellt~\cite{2023-egglog}.


\subsection{Aufbau der Arbeit}

\noindent {\itshape Kapitel \ref{sec:grundlagen} - Grundlagen}\vspace{-2mm}

In Kapitel zwei werden die nötigen theoretischen Grundlagen vermittelt. Beginnend mit einem naiven Ansatz zur Optimierung, werden schrittweise Verbesserungen
eingeführt, die schließlich in der Erstellung von \textit{E-Graphs} und das Anwenden von \textit{Equality Saturation} resultieren. 

\vspace{6mm}

\noindent {\itshape Kapitel \ref{sec:entscheidungen} - Technologiestack}\vspace{-2mm}

Der Fokus im vierten Kapitel liegt auf den getroffenen Entscheidungen hinsichtlich Softwarelösungen, die in der Arbeit zum Einsatz kamen. Dazu werden Argumente für deren Verwendung erörtert.

\vspace{6mm}

\noindent {\itshape Kapitel \ref{sec:architektur} - Architektur der Anwendung}\vspace{-2mm}

Im dritten Kapitel wird die Architektur der Anwendung skizziert sowie die Intention hinter einzelnen Komponenten erläutert. Zusätzlich wird die Kommunikation zwischen 
Server und Weboberfläche beleuchtet. 

\vspace{6mm}

\noindent {\itshape Kapitel \ref{sec:entwicklung} - Entwicklung der Anwendung}\vspace{-2mm}

In Kapitel fünf wird ein detailierter Einblick in die Vorgehensweise während der Entwicklung ermöglicht. Außerdem wird die praktische Umsetzung aller Komponenten erklärt.

\vspace{6mm}

\noindent {\itshape Kapitel \ref{sec:probleme} - Komplikationen während der Entwicklung}\vspace{-2mm}

Im sechsten Kapitel werden Probleme thematisiert, zu denen es während dieser Arbeit kam. Die Ursachen der Probleme werden dabei untersucht und mögliche Lösungsansätze präsentiert. 

\vspace{6mm}

\noindent {\itshape Kapitel \ref{sec:reflexion} - Reflexion der Arbeit}\vspace{-2mm}

Gegenstand des siebten Kapitels ist eine kritische Reflexion der Arbeit. Hierzu wird analysiert, inwiefern das Ziel der Arbeit erreicht wurde und an welchen Faktoren dies gemessen werden kann.

\vspace{6mm}

\noindent {\itshape Kapitel \ref{sec:fazit} - Fazit und Ausblick}\vspace{-2mm}

Im abschließenden Kapitel wird ein Fazit der Arbeit gezogen und ein Ausblick auf mögliche Erweiterungen der Software gegeben.


\EnableTitleSlide
\section{Hintergrund}
% Komplexe Zahlen
\begin{frame}{Komplexe Zahlen $\mathbb{C}$}    
    \begin{columns}[c]
        \column{0.4\textwidth}
            \begin{itemize}
                %\item Zahlen der Form $c=a+bi$ mit $a,b \in \mathbb{R}$
                \item $\mathbb{C} := \{a+ bi: a, b \in \mathbb{R}\}$
                \begin{itemize}
                    \item Realteil $Re(c)=a$
                    \item Imaginärteil $Im(c)=b$
                    \item Imaginäre Einheit $i$: $i^2 := -1$
                \end{itemize}
                \item $|c| = \sqrt{a^2+b^2}$
                \item $\mathbb{N} \subset \mathbb{Z} \subset \mathbb{Q} \subset \mathbb{R} \subset \mathbb{C}$
            \end{itemize}
            \column{0.3\textwidth}
            \begin{exampleblock}{Beispiel}
                \vspace{-0.45cm}
                \begin{align*}
                    c_1&=3 + 4i \\[0.2cm]
                    |c_1|&=\sqrt{3^2+4^2}\\ 
                    &=\sqrt{25} \\
                    &=5
                \end{align*}
            \end{exampleblock}
            \column{0.2\textwidth}
    \end{columns}
\end{frame}

% Komplexe Ebene
\begin{frame}{Komplexe Ebene}    
    \begin{columns}[c]
        \column{.45\textwidth}
            \begin{itemize}
                \item "`Gaußsche Zahlenebene"'
                \item Komplexen Zahl $c=a+bi$ als kartesische Koordinate $(a,b) \in \mathbb{R}^2$
                \item x-Achse codiert den Realteil $Re(c)$
                \item y-Achse codiert den Imaginärteil $Im(c)$
            \end{itemize}
            \onslide<2->{
            \begin{exampleblock}{Beispiel}
                $c_{{\scriptscriptstyle \color{red}{\text{P1}}}}=-1.2 + 1i$
            \end{exampleblock}
            }
        \column{.45\textwidth}
            % \only<1>{\vspace{0.18cm}
            % \includegraphics[scale=0.75]{utils/komplexe_Ebene/ebene_plain.pdf}}
            % \only<2->{
            % \includegraphics[scale=0.75]{utils/komplexe_Ebene/ebene_plain_example.pdf}}
    \end{columns}
\end{frame}



% Folgen
%\begin{frame}{Folgen}
%    \begin{itemize}
%        \item Durch Muster oder Regeln definierte (unendliche) Aufzählung von Zahlen
%        \item Folgeglieder $(a_n)_{n \in \mathbb{N}}$
%        \item Definierbar über Reihen, Algorithmen, Rekursionsvorschriften $\dots$
%        \item Beschränktheit nach oben durch $k$ $\Leftrightarrow$ $\forall n \in \mathbb{N} : a_n \leq k$ 
%    \end{itemize}
%    \vspace{0.2cm}
%    \begin{block}{Beispiel: Fibonaccifolge}
%        $f_{n}=f_{n-1}+f_{n-2}$ mit $f_1=f_2=1$
%    \end{block}
%    \begin{exampleblock}{Ergebnis: $(f_n)_{n \in \mathbb{N}}$}
%        $1, 1, 2, 3, 5, 8, 13, 21, 34, 55, 89, 144, 233, 377, 610, 987, 1597, 2584 \dots$
%    \end{exampleblock}
%\end{frame}


% Mandelbrot-Menge
\begin{frame}{Mandelbrot-Menge}
    \begin{columns}[c]
        \column{.6\textwidth}
            \begin{itemize}
                \item $ \color<2>{lightgray}\mathbb{M} :=\{ c \in \mathbb{C} \ | \ \color<1,2,3>{black}{z_{n+1} = z^{2}_{n} + c} \;\; \color<2>{lightgray}\textrm{ist beschränkt} \}$ mit $z_0=0$
                \begin{itemize}
                    \item $(z_n)_{n \in \mathbb{N}}$ ist rekursiv definierte Folge
                \end{itemize}
                \item $c \in \mathbb{M}$ als schwarze Punkte in der komplexen Zahlenebene markiert 
                \item Beschränktheit in Praxis nur nach $m \in \mathbb{N}_0$ Iterationen abschätzbar:
            \end{itemize}\vspace{0.25cm}
            \begin{block}{\centering Beschränktheitskriterium}
                %Ist ein Folgenglied weiter als $2$ vom Ursprung entfernt, ist $c$ nicht Teil der Mandelbrotmenge \\
                \centering
                $\exists n \leq m: |z_n| > 2 \Longrightarrow c \notin \mathbb{M}$
            \end{block}
        \column{.40\textwidth}
            % \includegraphics[scale=0.34]{../Code/Mandelbrot_axes/Fig1_axis_big.jpg}
\end{columns}
\end{frame}

\EnableTitleSlide
\section{Methoden}

% \begin{frame}{$\mathbb{M}$ Visualisieren}
%     \begin{center}
%         \includegraphics[scale=0.38]{../Code/Mandelbrot_blackwhite/blackwhite_zoom1.png}
%     \end{center}
% \end{frame}

% \begin{frame}{$\mathbb{M}$ Visualisieren}
%     \begin{center}
%         \includegraphics[scale=0.41]{../Code/Mandelbrot_vortrag/folien/mvisualisieren1.pdf}
%     \end{center}
% \end{frame}

% \begin{frame}{$\mathbb{M}$ Visualisieren}
%     \begin{center}
%         \includegraphics[scale=0.41]{../Code/Mandelbrot_vortrag/folien/mvisualisieren2.pdf}
%     \end{center}
% \end{frame}

% \begin{frame}{$\mathbb{M}$ Visualisieren}
%     \begin{center}
%         \includegraphics[scale=0.41]{../Code/Mandelbrot_vortrag/folien/mvisualisieren3.pdf}
%     \end{center}
% \end{frame}

\begin{frame}{$\mathbb{M}$ Visualisieren}
    \begin{center}
        \begin{algorithmic}
            \Function{ist\_Teil\_von\_Mandelbrot}{c}
            \State $i \gets 0$
            \State $iterationen \gets 20$
            \State $z_{n+1}, z_n = 0$
            \While{$i < iterationen$}
            
            \State $z_{n+1} = {z_n}^2 + c$

            \If{$|z_{n+1}| > 2$}
            \State \Return $0$
            \EndIf

            \State $i \gets i + 1$
            \EndWhile

            \State \Return $1$
            \EndFunction
        \end{algorithmic}
    \end{center}
\end{frame}

% \begin{frame}{$\mathbb{M}$ Visualisieren}
%     \begin{center}
%         \includegraphics[scale=0.41]{../Code/Mandelbrot_vortrag/folien/mvisualisieren4.pdf}
%     \end{center}
% \end{frame}

% \begin{frame}{Parallelisieren}
%     \begin{center}
%         \includegraphics[scale=0.39]{../Code/Mandelbrot_vortrag/folien/matrix_parallel.pdf}
%     \end{center}
% \end{frame}

% \begin{frame}{Einfärbung}
%     \begin{center}
%         \includegraphics[scale=0.12]{utils/Region6_1.jpg}
%     \end{center}
% \end{frame}

\begin{frame}{Einfärbung - diskret}
    \begin{center}
        \begin{algorithmic}
            \Function{ist\_Teil\_von\_Mandelbrot}{c}
            \State $i \gets 0$
            \State $iterationen \gets 40$
            \State $z_{n+1}, z_n = 0$
            \While{$i < iterationen$}
            
            \State $z_{n+1} = {z_n}^2 + c$

            \If{$|z_{n+1}| > 2$}
            \State \Return \BoxedString[fill=yellow]{$i$}
            \LComment{return 0}
            \EndIf

            \State $i \gets i + 1$
            \EndWhile

            \State \Return \BoxedString[fill=yellow]{$i$}
            \LComment{return 1}
            \EndFunction
        \end{algorithmic}
    \end{center}
\end{frame}

\begin{frame}{Einfärbung - diskret}
    % \begin{columns}
    %     \column{.5\textwidth}
    %         \begin{itemize}
    %             \item Einfärbung von $c \notin \mathbb{M}$
    %             \item Ab welcher Iteration $n$ ist $|z_n| > 2$?
    %         \end{itemize} 
    %         \vspace{0.2cm}
    %         \begin{center}
    %             \includegraphics[scale=0.8]{../Code/Mandelbrot_colored/iterations_colors.pdf}
    %         \end{center}
    %     \column{.45\textwidth}
    %         \includegraphics[scale=0.09]{../Code/Mandelbrot_colored/Fig2_color_pil.jpg}
    % \end{columns}
\end{frame}

% \begin{frame}{Einfärbung - kontinuierlich}
%     \begin{center}
%         % \includegraphics[scale=0.33]{../Code/Mandelbrot_vortrag/folien/einfaerbung.pdf}
%     \end{center}
% \end{frame}

% \begin{frame}{Zoom}
%     \begin{center}
%         \includegraphics[scale=0.33]{../Code/Mandelbrot_vortrag/folien/zoom1.pdf}
%     \end{center}
% \end{frame}

% \begin{frame}{Zoom}
%     \begin{center}
%         \includegraphics[scale=0.33]{../Code/Mandelbrot_vortrag/folien/zoom2.pdf}
%     \end{center}
% \end{frame}

% \begin{frame}{Zoom}
%     \begin{center}
%         \includegraphics[scale=0.33]{../Code/Mandelbrot_vortrag/folien/zoom3.pdf}
%     \end{center}
% \end{frame}

% \begin{frame}{Zoomfahrt}
%     \begin{center}
%         \includegraphics[scale=0.51]{../Code/Mandelbrot_vortrag/folien/zoomfahrt.pdf}
%     \end{center}
% \end{frame}

% \begin{frame}{Zoomfahrt}
%     \begin{center}
%         \includegraphics[scale=0.38]{../Code/Mandelbrot_vortrag/folien/zoomfahrt-explanation.pdf}
%     \end{center}
% \end{frame}

\section{Ergebnispräsentation}

\begin{frame}
  \frametitle{Grafiken}
  Grafiken werden wie gewohnt eingebunden (nur ohne \texttt{figure}-Umgebung):
  \begin{center}
    \includegraphics[width=0.8\textwidth]{fig/beispiel.eps}
  \end{center}
\end{frame}


%\EnableTitleSlide
\section{Diskussion}

\begin{frame}{Diskussion}

\end{frame}

\section{Referenzen}

% \begin{frame}{Referenzen}
%     \footnotesize{
%         \begin{thebibliography}{99}
%             \bibitem[1]{Mandelbrot2004} Benoit B. Mandelbrot (2004)
%             \newblock Fractals and Chaos: The Mandelbrot Set and Beyond
%             \newblock Springer New York, S. 9 -- 26.

%             \bibitem[2]{Andreadis2015} Ioannis Andreadis und Theodoros E. Karakasidis  (2015)
%             \newblock On a numerical approximation of the boundary structure and of the area of the Mandelbrot set
%             \newblock \emph{Nonlinear Dynamics}, Jg. 8, S. 929 -- 935.

%             \bibitem[3]{Shishikura1998} Mitsuhiro Shishikura  (1998)
%             \newblock The Hausdorff Dimension of the Boundary of the Mandelbrot Set and Julia Sets
%             \newblock \emph{Annals of Mathematics}, Jg. 147, Nr. 2, S. 225 -- 267.
%         \end{thebibliography}
%     }
% \end{frame}

\begin{frame}{Referenzen}
    \nocite{2021-egg}
    \nocite{devito}
    \printbibliography
\end{frame}

\EnableTitleSlide
\section{Weiterführendes}

\begin{frame}{Daily Mandelbrot}
    \begin{center}
        % \includegraphics[scale=0.45]{utils/mm_x.png}
    \end{center}
\end{frame}

\begin{frame}{Buddhabrot}
    \begin{columns}
        \column{.5\textwidth}
            \begin{itemize}
                \item Basiert auf der gleichen rekursiven Formel wie $\mathbb{M}$
                \item Visualisiert die Folgeglieder $\{z^c_0, \dots, z^c_n\}$ für zufällige $c \notin \mathbb{M}$
                \item Je häufiger ein Folgeglied berechnet wird, desto heller der Punkt
                \item Detailgrad wächst mit der Anzahl an Iterationen $n$ und der Anzahl ausgewählter $c \notin \mathbb{M}$
            \end{itemize}
        \column{.45\textwidth}
            % \includegraphics[scale=0.42]{../Code/Mandelbrot_buddhabrot/Fig_buddhabrot.png}
    \end{columns}
\end{frame}

\begin{frame}{Julia-Mengen}
    Die Mandelbrotmenge weist Parallelen zu \textit{Julia-Mengen} $K_c$ auf und lässt sich darüber auch definieren:
    \begin{itemize}
        \item $K_c := \{ z_0 \in \mathbb{C} \ | \ (z^{2}_{n} + c)_{n \in \mathbb{N}} \;\; \textrm{ist beschränkt} \}$
        \item $\mathbb{M}=\{c \in \mathbb{C} \ | \ 0 \in K_c\} $
        \item An Randpunkten $r \in \mathbb{M}$ liegt in $\mathbb{M}$ und $K_c$ lokal ein ähnliches Muster vor
        \item Genau dann, wenn $c\in\mathbb{M}$, dann ist $K_c$ zusammenhängend
    \end{itemize}
\end{frame}

% %------------------------------------------------
% \section{First Section}
% %------------------------------------------------

% \begin{frame}{Bullet Points}
%     \begin{itemize}
%         \item Lorem ipsum dolor sit amet, consectetur adipiscing elit
%         \item Aliquam blandit faucibus nisi, sit amet dapibus enim tempus eu
%         \item Nulla commodo, erat quis gravida posuere, elit lacus lobortis est, quis porttitor odio mauris at libero
%         \item Nam cursus est eget velit posuere pellentesque
%         \item Vestibulum faucibus velit a augue condimentum quis convallis nulla gravida
%     \end{itemize}
% \end{frame}

% %------------------------------------------------

% \begin{frame}{Blocks of Highlighted Text}
%     In this slide, some important text will be \alert{highlighted} because it's important. Please, don't abuse it.

%     \begin{block}{Block}
%         Sample text
%     \end{block}

%     \begin{alertblock}{Alertblock}
%         Sample text in red box
%     \end{alertblock}

%     \begin{examples}
%         Sample text in green box. The title of the block is ``Examples".
%     \end{examples}
% \end{frame}

% %------------------------------------------------

% \begin{frame}{Multiple Columns}
%     \begin{columns}[c] % The "c" option specifies centered vertical alignment while the "t" option is used for top vertical alignment

%         \column{.45\textwidth} % Left column and width
%         \textbf{Heading}
%         \begin{enumerate}
%             \item Statement
%             \item Explanation
%             \item Example
%         \end{enumerate}

%         \column{.5\textwidth} % Right column and width
%         Lorem ipsum dolor sit amet, consectetur adipiscing elit. Integer lectus nisl, ultricies in feugiat rutrum, porttitor sit amet augue. Aliquam ut tortor mauris. Sed volutpat ante purus, quis accumsan dolor.

%     \end{columns}
% \end{frame}

% %------------------------------------------------
% \section{Second Section}
% %------------------------------------------------

% \begin{frame}{Table}
%     \begin{table}
%         \begin{tabular}{l l l}
%             \toprule
%             \textbf{Treatments} & \textbf{Response 1} & \textbf{Response 2} \\
%             \midrule
%             Treatment 1         & 0.0003262           & 0.562               \\
%             Treatment 2         & 0.0015681           & 0.910               \\
%             Treatment 3         & 0.0009271           & 0.296               \\
%             \bottomrule
%         \end{tabular}
%         \caption{Table caption}
%     \end{table}
% \end{frame}

% %------------------------------------------------

% \begin{frame}{Theorem}
%     \begin{theorem}[Mass--energy equivalence]
%         $E = mc^2$
%     \end{theorem}
% \end{frame}

% %------------------------------------------------

% \begin{frame}{Figure}
%     Uncomment the code on this slide to include your own image from the same directory as the template .TeX file.
%     %\begin{figure}
%     %\includegraphics[width=0.8\linewidth]{test}
%     %\end{figure}
% \end{frame}

% %------------------------------------------------

% \begin{frame}[fragile] % Need to use the fragile option when verbatim is used in the slide
%     \frametitle{Citation}
%     An example of the \verb|\cite| command to cite within the presentation:\\~

%     This statement requires citation \cite{p1}.
% \end{frame}

% %------------------------------------------------

% \begin{frame}{References}
%     % Beamer does not support BibTeX so references must be inserted manually as below
%     \footnotesize{
%         \begin{thebibliography}{99}
%             \bibitem[Smith, 2012]{p1} John Smith (2012)
%             \newblock Title of the publication
%             \newblock \emph{Journal Name} 12(3), 45 -- 678.
%         \end{thebibliography}
%     }
% \end{frame}

% %------------------------------------------------

% \begin{frame}
%     \Huge{\centerline{\textbf{The End}}}
% \end{frame}

% %----------------------------------------------------------------------------------------

\end{document}