%----------------------------------------------------------------------------------------
%	PACKAGES AND THEMES
%----------------------------------------------------------------------------------------
\documentclass[aspectratio=169,xcolor=dvipsnames]{beamer}
\usetheme{SimplePlus}

\usepackage{hyperref}
\usepackage{color}
\usepackage{graphicx} % Allows including images
\usepackage{subcaption} % Subfigures
\usepackage{booktabs} % Allows the use of \toprule, \midrule and \bottomrule in tables
\usepackage[ngerman]{babel}
\setbeamertemplate{bibliography item}{\insertbiblabel}
\usepackage{algpseudocodex}

\algrenewcommand\algorithmicrequire{\textbf{Voraussetzung:}}
\algrenewcommand\algorithmicfunction{\textbf{Funktion}}

\usepackage{datetime}
\newdate{vortragsdatum}{01}{04}{2025}

\usepackage{bootstrapcolors}

%----------------------------------------------------------------------------------------
%	TITLE PAGE
%----------------------------------------------------------------------------------------

\title[short title]{\vspace{-10mm}{\bf\normalsize \color{black} Bachelorarbeit} \vspace{3mm}\\ \\ \normalsize Entwicklung einer Anwendung zum Vermitteln von Lerninhalten über E-Graphs und Equality Saturation für die Lehre} % The short title appears at the bottom of every slide, the full title is only on the title page
%\subtitle{Subtitle}

\author[rjb] {\bf Ben Steinhauer} 

\institute[HHU] % Your institution as it will appear on the bottom of every slide, may be shorthand to save space
{
    Lehrstuhl für Softwaretechnik und Programmiersprachen \\
    Heinrich-Heine-Universität Düsseldorf% Your institution for the title page
}
%\date{\today} % Date, can be changed to a custom date
\date{\displaydate{vortragsdatum}}

\definecolor{lightblue1}{rgb}{0.92, 0.92, 0.97}
\definecolor{hhuUniBlau}{RGB}{0,106,179}

%----------------------------------------------------------------------------------------
%	CHAPTER TITLE SLIDES
%----------------------------------------------------------------------------------------

% Define a flag to control the display of the title slide
\newif\ifshowtitleslide

% Set the default behavior to not show the title slide
\showtitleslidefalse

% Redefine the \AtBeginSection command to check the flag
\AtBeginSection[]{
  \ifshowtitleslide
    \begin{frame}
      \vfill
      \centering
      \begin{beamercolorbox}[sep=8pt,center,shadow=true,rounded=true]{title}
        \usebeamerfont{title}\insertsectionhead\par%
      \end{beamercolorbox}
      \vfill
    \end{frame}
  \fi
  % Reset the flag to the default state after each section
  \showtitleslidefalse
}

% New command to enable the title slide for the next section
\newcommand{\EnableTitleSlide}{\showtitleslidetrue}

%----------------------------------------------------------------------------------------
%	PRESENTATION SLIDES
%----------------------------------------------------------------------------------------

\begin{document}

\begin{frame}
    \thispagestyle{empty}
    % Print the title page as the first slide
    %\titlepage
    \centering

    {\bf\color{black} Bachelorarbeit}\vspace{4mm}

    {\color{hhuUniBlau}\large Entwicklung einer Anwendung zum Vermitteln von Lerninhalten über E-Graphs und Equality Saturation für die Lehre}\vspace{4mm}

    {\color{hhuUniBlau} Ben Steinhauer}\vspace{3mm}

    {\footnotesize Lehrstuhl für Softwaretechnik und Programmiersprachen}\vspace{-1mm}

    {\footnotesize Heinrich-Heine-Universität Düsseldorf}\vspace{4mm}

    {\footnotesize 01.04.2025}

\end{frame}

%\begin{frame}{Überblick}
    % Throughout your presentation, if you choose to use \section{} and \subsection{} commands, these will automatically be printed on this slide as an overview of your presentation
    \tableofcontents
\end{frame}

\section{Einleitung}

\subsection{Motivation}

%~\cite{phaseorder-2009}

\subsection{Ziel der Arbeit}

Das Ziel dieser Bachelorarbeit ist es, ein sinnvolles Werkzeug für die Lehre zu erstellen,
um Studentinnen und Studenten die Themen \textbf{E-Graphs} und \textbf{Equality Saturation} näher zu bringen.
Dabei sollen sie die Möglichkeit haben, sich sowohl auf theoretischer als auch praktischer Ebene mit E-Graphs auseinander setzen zu können.
Die theoretische Ebene soll den Studenten die notwendigen Hintergrundkenntnisse vermitteln sowie einen Enblick in die Implementierung geben.
Die praktische Ebene soll Schritt für Schritt aufzeigen, wie der \textbf{E-Graph} aufgebaut wird, und wie an diesem \textbf{Equality Saturation} durchgeführt werden kann.
Für grö{\ss}tmöglichen Nutzen soll die Anwendung plattformunabhängig sein und möglichst nur von \textit{Open-Source-Software} (OSS) Gebrauch machen.
Damit wird das Problem der unterschiedlichen Betriebssysteme der Studenten umgangen und zeitgleich die Hürden für Erweiterungen gesenkt.

\subsection{Aufbau der Arbeit}

\noindent {\itshape Kapitel \ref{sec:grundlagen} - Grundlagen}

Dieses Kapitel zeichnet den Beginn der Arbeit, indem notwendige (mathematische) Kenntnisse über \textbf{E-Graphs} und \textbf{Equality Saturation} vermittelt werden.

\noindent {\itshape Kapitel \ref{sec:entwicklung} - Entwicklung}

\noindent {\itshape Kapitel \ref{sec:entwicklung} - Architektur}

\noindent {\itshape Kapitel \ref{sec:entscheidungenundprobleme} - Entscheidungen und Probleme}

\noindent {\itshape Kapitel \ref{sec:zusammenfassung} - Zusammenfassung}

\subsection{Verwandte Arbeiten}

\noindent\textbf{egg} Das Akronym \textit{egg} steht für \textit{e-graphs good} und ist eine in der Programmiersprache \textit{Rust} geschriebene Bibliothek.
Die Bibliothek implementiert die Datenstruktur E-Graph und stellt Funktionen für die Interaktion mit dieser zur Verfügung.
Neben der Erzeugung von E-Graphs kann auch \textit{Equality Saturation} auf diesen durchgeführt werden.
Das gesamte Projekt wurde in einem Paper vorgestellt, auf welches sich diese Arbeit stützt~\cite{2021-egg}.


\EnableTitleSlide
\section{Hintergrund}
% Komplexe Zahlen
\begin{frame}{Komplexe Zahlen $\mathbb{C}$}    
    \begin{columns}[c]
        \column{0.4\textwidth}
            \begin{itemize}
                %\item Zahlen der Form $c=a+bi$ mit $a,b \in \mathbb{R}$
                \item $\mathbb{C} := \{a+ bi: a, b \in \mathbb{R}\}$
                \begin{itemize}
                    \item Realteil $Re(c)=a$
                    \item Imaginärteil $Im(c)=b$
                    \item Imaginäre Einheit $i$: $i^2 := -1$
                \end{itemize}
                \item $|c| = \sqrt{a^2+b^2}$
                \item $\mathbb{N} \subset \mathbb{Z} \subset \mathbb{Q} \subset \mathbb{R} \subset \mathbb{C}$
            \end{itemize}
            \column{0.3\textwidth}
            \begin{exampleblock}{Beispiel}
                \vspace{-0.45cm}
                \begin{align*}
                    c_1&=3 + 4i \\[0.2cm]
                    |c_1|&=\sqrt{3^2+4^2}\\ 
                    &=\sqrt{25} \\
                    &=5
                \end{align*}
            \end{exampleblock}
            \column{0.2\textwidth}
    \end{columns}
\end{frame}

% Komplexe Ebene
\begin{frame}{Komplexe Ebene}    
    \begin{columns}[c]
        \column{.45\textwidth}
            \begin{itemize}
                \item "`Gaußsche Zahlenebene"'
                \item Komplexen Zahl $c=a+bi$ als kartesische Koordinate $(a,b) \in \mathbb{R}^2$
                \item x-Achse codiert den Realteil $Re(c)$
                \item y-Achse codiert den Imaginärteil $Im(c)$
            \end{itemize}
            \onslide<2->{
            \begin{exampleblock}{Beispiel}
                $c_{{\scriptscriptstyle \color{red}{\text{P1}}}}=-1.2 + 1i$
            \end{exampleblock}
            }
        \column{.45\textwidth}
            % \only<1>{\vspace{0.18cm}
            % \includegraphics[scale=0.75]{utils/komplexe_Ebene/ebene_plain.pdf}}
            % \only<2->{
            % \includegraphics[scale=0.75]{utils/komplexe_Ebene/ebene_plain_example.pdf}}
    \end{columns}
\end{frame}



% Folgen
%\begin{frame}{Folgen}
%    \begin{itemize}
%        \item Durch Muster oder Regeln definierte (unendliche) Aufzählung von Zahlen
%        \item Folgeglieder $(a_n)_{n \in \mathbb{N}}$
%        \item Definierbar über Reihen, Algorithmen, Rekursionsvorschriften $\dots$
%        \item Beschränktheit nach oben durch $k$ $\Leftrightarrow$ $\forall n \in \mathbb{N} : a_n \leq k$ 
%    \end{itemize}
%    \vspace{0.2cm}
%    \begin{block}{Beispiel: Fibonaccifolge}
%        $f_{n}=f_{n-1}+f_{n-2}$ mit $f_1=f_2=1$
%    \end{block}
%    \begin{exampleblock}{Ergebnis: $(f_n)_{n \in \mathbb{N}}$}
%        $1, 1, 2, 3, 5, 8, 13, 21, 34, 55, 89, 144, 233, 377, 610, 987, 1597, 2584 \dots$
%    \end{exampleblock}
%\end{frame}


% Mandelbrot-Menge
\begin{frame}{Mandelbrot-Menge}
    \begin{columns}[c]
        \column{.6\textwidth}
            \begin{itemize}
                \item $ \color<2>{lightgray}\mathbb{M} :=\{ c \in \mathbb{C} \ | \ \color<1,2,3>{black}{z_{n+1} = z^{2}_{n} + c} \;\; \color<2>{lightgray}\textrm{ist beschränkt} \}$ mit $z_0=0$
                \begin{itemize}
                    \item $(z_n)_{n \in \mathbb{N}}$ ist rekursiv definierte Folge
                \end{itemize}
                \item $c \in \mathbb{M}$ als schwarze Punkte in der komplexen Zahlenebene markiert 
                \item Beschränktheit in Praxis nur nach $m \in \mathbb{N}_0$ Iterationen abschätzbar:
            \end{itemize}\vspace{0.25cm}
            \begin{block}{\centering Beschränktheitskriterium}
                %Ist ein Folgenglied weiter als $2$ vom Ursprung entfernt, ist $c$ nicht Teil der Mandelbrotmenge \\
                \centering
                $\exists n \leq m: |z_n| > 2 \Longrightarrow c \notin \mathbb{M}$
            \end{block}
        \column{.40\textwidth}
            % \includegraphics[scale=0.34]{../Code/Mandelbrot_axes/Fig1_axis_big.jpg}
\end{columns}
\end{frame}

\EnableTitleSlide
\section{Aufbau \& Entwicklung \\ der Anwendung}

\begin{frame}{Anwendung (1)}
    
\end{frame}

\begin{frame}{Anwendung (2)}
    
\end{frame}

\begin{frame}{Anwendung (3)}
    
\end{frame}

\begin{frame}{Aufbau}
    \begin{figure}[H]
        \centering
        \includegraphics[scale=0.43]{utils/components.png}
        \caption{Architekturdiagramm der Anwendung}
        \label{fig:comps}
    \end{figure}
\end{frame}


\EnableTitleSlide
\section{Ergebnisse \& Diskussion}

\begin{frame}[fragile]{Komplikationen während der Entwicklung}
    \begin{enumerate}
        \item \onslide<1->{\textbf{Kombination zweier Implementierungen}}
        
        \only{
            \vspace{8mm}
            \begin{center}
                \includegraphics[scale=0.75]{utils/colabegg.png}
            \end{center}
        }<1>

        \item \onslide<2->{\textbf{Darstellung der Datenstruktur}}
        
        \only{
            \vspace{2mm}
            \begin{center}
                \includegraphics[scale=0.4]{utils/egraph_exp.png}
            \end{center}
        }<2>

        \begin{onlyenv}<3>
            \vspace{3mm}
            \begin{center}
\begin{minted}[fontsize=\small]{python}    
def egraph_to_dot(self, nodesep=0.5, ranksep=0.5, marked_eclasses = []):
    dot_commands = [
        "digraph parent { graph [compound=true, nodesep=" + str(nodesep)
        + ", ranksep=" + str(ranksep) + "]\n" + """node [fillcolor=white 
        fontname=\"Times-Bold\" fontsize=20 shape=record style=\"rounded, 
        filled\"]\n"""
    ]
    # ... insgesamt 106 Zeilen lang
    return "".join(dot_commands)
\end{minted}
        \end{center}
        \end{onlyenv}

        \item \onslide<4->{\textbf{Spezialfall: Kreis im E-Graph}}
        
        \only<4>{\vspace{3mm}\begin{center}\includegraphics[scale=0.3]{utils/loop.png}\end{center}}
        
\begin{onlyenv}<5>
    \vspace{3mm}
    \begin{center}
\begin{minted}[fontsize=\small]{python}    
def equality_saturation(rules, eterm_id, egraph):
    #
    while True and timeout < 3:
        v = egraph.version
        timeout += 1
        #
        if v == egraph.version:
            break
\end{minted}
\end{center}
\end{onlyenv}

\begin{onlyenv}<6>
    \vspace{3mm}
    \begin{center}
\begin{minted}[fontsize=\small]{python}    
def equality_saturation(rules, eterm_id, egraph):
    #
    while True:
        best_term = _extract_term(eterm_id, egraph)
        if old_term == best_term:
            break
        old_term = best_term
        #  
\end{minted}
\end{center}
\end{onlyenv}
        
        \item \onslide<7->{\color{gray-500}\textbf{Pfadangabe im Server}\color{black}}
        
\begin{onlyenv}<8>
    \vspace{3mm}
    \begin{center}
\begin{minted}[fontsize=\small]{python}    
app.mount("/", 
StaticFiles(directory=realpath(f"{realpath(__file__)}/../static"), html=True),
    name="static"
) # https://github.com/fastapi/fastapi/issues/3550
\end{minted}
\end{center}
\end{onlyenv}

    \end{enumerate}
\end{frame}

\begin{frame}{Ergebnisse}
    \begin{enumerate}
        \item funktionstüchtige Anwendung
        \item plattformunabhängig \& getestet
        \item basiert ausschließlich auf Open-Source-Software
        \item erfüllt Anforderungen des Exposés~\cite{expose}:
        \begin{itemize}
            \color{gray-500}
            \item Benutzeroberfläche im Browser
            \item Erzeugen \& Visualisieren von E-Graphs
            \item Erstellen \& Anwenden von rewrite rules, vordefinierte rewrite rules
            \item Debugging-Feature
            \item Extraktion des optimalen Terms
            \item Export des E-Graphs in gängige Formate
            \item Eingaben als Session abspeichern
            \item Dokumentation
            \color{black}
        \end{itemize}
    \end{enumerate}
\end{frame}

\begin{frame}{Qualität der Software}
    nach ISO/OEC 20510:2011 Standard:\vspace{4mm}

    \begin{enumerate}
        \bf
        \item Übertragbarkeit
        \item Wartbarkeit
        \item Sicherheit
        \item Zuverlässigkeit
        \item Funktionale Eignung
        \item Performance
        \item Kompatibilität
        \item Benutzbarkeit
    \end{enumerate}
\end{frame}

\begin{frame}{Erweiterungen}
    \begin{enumerate}
        \item \textbf{Erweitertes Testing}
        \begin{itemize}
            \item Vergleichsmethode: eigene Implementierung vs. egg 
        \end{itemize}
        \item \textbf{Erstellung von E-Graphs visualisieren}
        \begin{itemize}
            \item schrittweises Darstellung vom Aufbauprozess (AST zu E-Graph) 
        \end{itemize}
        \item \textbf{Abindung an egg}
        \begin{itemize}
            \item Benutzer kann zwischen eigener Implementierung und egg als Backend wählen
            \item benötigt Methoden in Rust
        \end{itemize}
        \item \textbf{E-Class Analysis}
        \begin{itemize}
            \item Conditional and Dynamic Rewrites
            \item Constant Folding
        \end{itemize}
    \end{enumerate}
\end{frame}

%\input{slides/diskussion.tex}

\section{Referenzen}

\begin{frame}{Referenzen}
    \footnotesize{
        \begin{thebibliography}{99}
            \bibitem[1]{Mandelbrot2004} Benoit B. Mandelbrot (2004)
            \newblock Fractals and Chaos: The Mandelbrot Set and Beyond
            \newblock Springer New York, S. 9 -- 26.

            \bibitem[2]{Andreadis2015} Ioannis Andreadis und Theodoros E. Karakasidis  (2015)
            \newblock On a numerical approximation of the boundary structure and of the area of the Mandelbrot set
            \newblock \emph{Nonlinear Dynamics}, Jg. 8, S. 929 -- 935.

            \bibitem[3]{Shishikura1998} Mitsuhiro Shishikura  (1998)
            \newblock The Hausdorff Dimension of the Boundary of the Mandelbrot Set and Julia Sets
            \newblock \emph{Annals of Mathematics}, Jg. 147, Nr. 2, S. 225 -- 267.
        \end{thebibliography}
    }
\end{frame}

\EnableTitleSlide
\section{Weiterführendes}

\begin{frame}{Weiterführendes 1}
   
\end{frame}


% %------------------------------------------------
% \section{First Section}
% %------------------------------------------------

% \begin{frame}{Bullet Points}
%     \begin{itemize}
%         \item Lorem ipsum dolor sit amet, consectetur adipiscing elit
%         \item Aliquam blandit faucibus nisi, sit amet dapibus enim tempus eu
%         \item Nulla commodo, erat quis gravida posuere, elit lacus lobortis est, quis porttitor odio mauris at libero
%         \item Nam cursus est eget velit posuere pellentesque
%         \item Vestibulum faucibus velit a augue condimentum quis convallis nulla gravida
%     \end{itemize}
% \end{frame}

% %------------------------------------------------

% \begin{frame}{Blocks of Highlighted Text}
%     In this slide, some important text will be \alert{highlighted} because it's important. Please, don't abuse it.

%     \begin{block}{Block}
%         Sample text
%     \end{block}

%     \begin{alertblock}{Alertblock}
%         Sample text in red box
%     \end{alertblock}

%     \begin{examples}
%         Sample text in green box. The title of the block is ``Examples".
%     \end{examples}
% \end{frame}

% %------------------------------------------------

% \begin{frame}{Multiple Columns}
%     \begin{columns}[c] % The "c" option specifies centered vertical alignment while the "t" option is used for top vertical alignment

%         \column{.45\textwidth} % Left column and width
%         \textbf{Heading}
%         \begin{enumerate}
%             \item Statement
%             \item Explanation
%             \item Example
%         \end{enumerate}

%         \column{.5\textwidth} % Right column and width
%         Lorem ipsum dolor sit amet, consectetur adipiscing elit. Integer lectus nisl, ultricies in feugiat rutrum, porttitor sit amet augue. Aliquam ut tortor mauris. Sed volutpat ante purus, quis accumsan dolor.

%     \end{columns}
% \end{frame}

% %------------------------------------------------
% \section{Second Section}
% %------------------------------------------------

% \begin{frame}{Table}
%     \begin{table}
%         \begin{tabular}{l l l}
%             \toprule
%             \textbf{Treatments} & \textbf{Response 1} & \textbf{Response 2} \\
%             \midrule
%             Treatment 1         & 0.0003262           & 0.562               \\
%             Treatment 2         & 0.0015681           & 0.910               \\
%             Treatment 3         & 0.0009271           & 0.296               \\
%             \bottomrule
%         \end{tabular}
%         \caption{Table caption}
%     \end{table}
% \end{frame}

% %------------------------------------------------

% \begin{frame}{Theorem}
%     \begin{theorem}[Mass--energy equivalence]
%         $E = mc^2$
%     \end{theorem}
% \end{frame}

% %------------------------------------------------

% \begin{frame}{Figure}
%     Uncomment the code on this slide to include your own image from the same directory as the template .TeX file.
%     %\begin{figure}
%     %\includegraphics[width=0.8\linewidth]{test}
%     %\end{figure}
% \end{frame}

% %------------------------------------------------

% \begin{frame}[fragile] % Need to use the fragile option when verbatim is used in the slide
%     \frametitle{Citation}
%     An example of the \verb|\cite| command to cite within the presentation:\\~

%     This statement requires citation \cite{p1}.
% \end{frame}

% %------------------------------------------------

% \begin{frame}{References}
%     % Beamer does not support BibTeX so references must be inserted manually as below
%     \footnotesize{
%         \begin{thebibliography}{99}
%             \bibitem[Smith, 2012]{p1} John Smith (2012)
%             \newblock Title of the publication
%             \newblock \emph{Journal Name} 12(3), 45 -- 678.
%         \end{thebibliography}
%     }
% \end{frame}

% %------------------------------------------------

% \begin{frame}
%     \Huge{\centerline{\textbf{The End}}}
% \end{frame}

% %----------------------------------------------------------------------------------------

\end{document}