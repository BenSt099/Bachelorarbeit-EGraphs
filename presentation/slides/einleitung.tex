%\EnableTitleSlide
\section{Einleitung}

\begin{frame}{Einleitung}
    \begin{columns}[c] % The "c" option specifies centered vertical alignment while the "t" option is used for top vertical alignment

        \column{.45\textwidth} % Left column and width
        \textbf{Mandelbrot-Menge $\mathbb{M}$}
        \begin{itemize}
            \item 1979 durch Benoît Mandelbrot entdeckt
            \item Teilmenge der komplexen Zahlen
            \item Fraktal                           %(unendlich detailliert bei Vergrößerung) -> Bei Grafischer Darstellung
            \begin{itemize}
                \item selbstähnlich                 %(nicht strikt selbstähnlich)
            \end{itemize}
        \end{itemize}

        \column{.5\textwidth} % Right column and width
        %\includegraphics[scale=0.045]{utils/Fig1_color_pil.jpg}
        % \includegraphics[scale=0.36]{../Code/Mandelbrot_blackwhite/blackwhite_zoom2.png}
    \end{columns}
\end{frame}


% Eigenschaften der Mandelbromenge
\begin{frame}{Eigenschaften der Mandelbrotmenge}
    \textbf{Verwendung:}
    \begin{itemize}
        \item computergenerierte Kunst
        \item mathematischer Forschungsgegenstand % Gegenstand mathematischer Forschung
    \end{itemize} \vspace{0.3cm}
    \pause
    \textbf{Forschungsstand:} 
    \begin{itemize}
        \item global zusammenhängen (d.h. ohne Inseln)
        \begin{itemize}
            \item lokaler Zusammenhang vermutet ("`MLC-Vermutung"')  \cite{Mandelbrot2004}
        \end{itemize}
        \item Numerische Schätzung des \textit{Flächeninhalts}: $1,5052$ \cite{Andreadis2015}
        \item Besitzt eine \textit{fraktale Dimension} von $D=2$ (Hausdorffdimension) \cite{Shishikura1998}
    \end{itemize}
\end{frame}