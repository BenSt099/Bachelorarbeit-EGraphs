\EnableTitleSlide
\section{Hintergrund}
% Komplexe Zahlen
\begin{frame}{Komplexe Zahlen $\mathbb{C}$}    
    \begin{columns}[c]
        \column{0.4\textwidth}
            \begin{itemize}
                %\item Zahlen der Form $c=a+bi$ mit $a,b \in \mathbb{R}$
                \item $\mathbb{C} := \{a+ bi: a, b \in \mathbb{R}\}$
                \begin{itemize}
                    \item Realteil $Re(c)=a$
                    \item Imaginärteil $Im(c)=b$
                    \item Imaginäre Einheit $i$: $i^2 := -1$
                \end{itemize}
                \item $|c| = \sqrt{a^2+b^2}$
                \item $\mathbb{N} \subset \mathbb{Z} \subset \mathbb{Q} \subset \mathbb{R} \subset \mathbb{C}$
            \end{itemize}
            \column{0.3\textwidth}
            \begin{exampleblock}{Beispiel}
                \vspace{-0.45cm}
                \begin{align*}
                    c_1&=3 + 4i \\[0.2cm]
                    |c_1|&=\sqrt{3^2+4^2}\\ 
                    &=\sqrt{25} \\
                    &=5
                \end{align*}
            \end{exampleblock}
            \column{0.2\textwidth}
    \end{columns}
\end{frame}

% Komplexe Ebene
\begin{frame}{Komplexe Ebene}    
    \begin{columns}[c]
        \column{.45\textwidth}
            \begin{itemize}
                \item "`Gaußsche Zahlenebene"'
                \item Komplexen Zahl $c=a+bi$ als kartesische Koordinate $(a,b) \in \mathbb{R}^2$
                \item x-Achse codiert den Realteil $Re(c)$
                \item y-Achse codiert den Imaginärteil $Im(c)$
            \end{itemize}
            \onslide<2->{
            \begin{exampleblock}{Beispiel}
                $c_{{\scriptscriptstyle \color{red}{\text{P1}}}}=-1.2 + 1i$
            \end{exampleblock}
            }
        \column{.45\textwidth}
            % \only<1>{\vspace{0.18cm}
            % \includegraphics[scale=0.75]{utils/komplexe_Ebene/ebene_plain.pdf}}
            % \only<2->{
            % \includegraphics[scale=0.75]{utils/komplexe_Ebene/ebene_plain_example.pdf}}
    \end{columns}
\end{frame}



% Folgen
%\begin{frame}{Folgen}
%    \begin{itemize}
%        \item Durch Muster oder Regeln definierte (unendliche) Aufzählung von Zahlen
%        \item Folgeglieder $(a_n)_{n \in \mathbb{N}}$
%        \item Definierbar über Reihen, Algorithmen, Rekursionsvorschriften $\dots$
%        \item Beschränktheit nach oben durch $k$ $\Leftrightarrow$ $\forall n \in \mathbb{N} : a_n \leq k$ 
%    \end{itemize}
%    \vspace{0.2cm}
%    \begin{block}{Beispiel: Fibonaccifolge}
%        $f_{n}=f_{n-1}+f_{n-2}$ mit $f_1=f_2=1$
%    \end{block}
%    \begin{exampleblock}{Ergebnis: $(f_n)_{n \in \mathbb{N}}$}
%        $1, 1, 2, 3, 5, 8, 13, 21, 34, 55, 89, 144, 233, 377, 610, 987, 1597, 2584 \dots$
%    \end{exampleblock}
%\end{frame}


% Mandelbrot-Menge
\begin{frame}{Mandelbrot-Menge}
    \begin{columns}[c]
        \column{.6\textwidth}
            \begin{itemize}
                \item $ \color<2>{lightgray}\mathbb{M} :=\{ c \in \mathbb{C} \ | \ \color<1,2,3>{black}{z_{n+1} = z^{2}_{n} + c} \;\; \color<2>{lightgray}\textrm{ist beschränkt} \}$ mit $z_0=0$
                \begin{itemize}
                    \item $(z_n)_{n \in \mathbb{N}}$ ist rekursiv definierte Folge
                \end{itemize}
                \item $c \in \mathbb{M}$ als schwarze Punkte in der komplexen Zahlenebene markiert 
                \item Beschränktheit in Praxis nur nach $m \in \mathbb{N}_0$ Iterationen abschätzbar:
            \end{itemize}\vspace{0.25cm}
            \begin{block}{\centering Beschränktheitskriterium}
                %Ist ein Folgenglied weiter als $2$ vom Ursprung entfernt, ist $c$ nicht Teil der Mandelbrotmenge \\
                \centering
                $\exists n \leq m: |z_n| > 2 \Longrightarrow c \notin \mathbb{M}$
            \end{block}
        \column{.40\textwidth}
            % \includegraphics[scale=0.34]{../Code/Mandelbrot_axes/Fig1_axis_big.jpg}
\end{columns}
\end{frame}