% Fassen Sie die Fragestellung, Motivation und Ergebnisse Ihrer Arbeit
% hier in wenigen Worten zusammen. Die Zusammenfassung sollte den Umfang einer Seite nicht überschreiten.


% Studentinnen und Studenten können Lehrinhalte auf unterschiedliche Weise vermittelt werden.
% Ziel dabei ist das Verstehen und Anwenden des Gelernten zu fördern.
% Eine Möglichkeit der Wissensvermittlung besteht darin, den Studentinnen und Studenten ein
% Werkzeug an die Hand zu geben, mit dem sie das Gelernte interaktiv anwenden können.
% Die in dieser Bachelorarbeit entwickelte Anwendung für die Lehre soll genau dieses Ziel erreichen, 
% indem sie die Interaktion mit \textbf{E-Graphs} und \textbf{Equality Saturation} möglich macht.

% \textbf{E-Graphs} lassen sich für die Optimierung von mathematischen Ausdrücken nutzen.
% Hierbei wird mit einem initialen Ausdruck gestartet, der durch Anwenden verschiedener 
% \textit{rewrite rules} in andere Ausdrücke umgeformt werden kann.
% Diese Ausdrücke lassen sich alle kompakt in einem \textbf{E-Graph} speichern. Aus diesem
% kann anschließend durch \textbf{Equality Saturation} ein optimaler Ausdruck extrahiert werden.

% Im Zuge dieser Arbeit wurde die Datenstruktur \textit{E-Graph} auf Grundlage zweier bereits existierender Implementierungen
% in \textit{Python} implementiert. Die Interaktion mit diesem regelt eine lokal laufende Website.
% Mit diesem Werkzeug ist es Studentinnen und Studenten möglich, die Umformung durch Anwenden von \textit{rewrite rules}
% zu beobachten sowie den Prozess der \textbf{Equality Saturation} selbst durchzuführen.

