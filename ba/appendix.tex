\section{Installation der Anwendung}\label{sec:installation}

\begin{enumerate}
  \item Das Repository des Projektes kann auf \textbf{GitHub} unter folgender Adresse erreicht werden: \url{https://github.com/BenSt099/Bachelorarbeit-EGraphs}.
  \item Das Repository klonen oder herunterladen.
  \item Nach dem Entpacken bitte die notwendigen Dependencies mit folgendem Befehl installieren:
  
  \begin{verbatim}
    pip install -r requirements.txt
  \end{verbatim}

  \item Zum Starten der Anwendung ein Terminal im \textit{src}-Ordner öffnen und folgenden Befehl ausführen:

  \begin{verbatim}
    fastapi run server.py
  \end{verbatim}
  
  \item Falls sich der Browser nicht automatisch öffnet, kann die Anwendung unter folgender Adresse erreicht werden:
  \url{http://127.0.0.1:8000}
  \item Die Dokumentation kann im Browser in der Navigationsleiste oder unter der Adresse \url{http://127.0.0.1:8000/dokumentation.html} gefunden werden.
\end{enumerate}

\section{Ausführen der Tests}

\begin{enumerate}
  \item Die notwendige Installation (siehe vorheriger Abschnitt \ref{sec:installation}) vornehmen.
  \item Bitte die notwendigen Dependencies zum Testen hinzufügen:
  
  \begin{verbatim}
    pip install pytest==8.3.3
  \end{verbatim}

  \begin{verbatim}
    pip install httpx==0.27.2
  \end{verbatim}

  \item Im Ordner mit den zwei Unterordnern \textit{src} und \textit{tests} ein Terminal öffnen und folgenden Befehl ausführen:
  
  \begin{verbatim}
    pytest
  \end{verbatim}

  \item Die Tests sollten automatisch von \textit{pytest} gefunden und ausgeführt werden. Für mehr Informationen enthält jede Testdatei am Anfang einen Kommentar, der erklärt, welche Komponenten
  getestet werden.
\end{enumerate}
