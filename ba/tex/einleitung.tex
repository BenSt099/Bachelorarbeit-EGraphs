\section{Einleitung}\label{sec:einleitung}

\subsection{Motivation}

\noindent In der Informatik nimmt das Optimieren von Ausdrücken eine wichtige Rolle ein. 
Ausdrücke können in Form von mathematischen Termen vorkommen oder als Programmcode auftreten.
Häufig ist der Compiler für diese Aufgabe zuständig, da manche Optimierungen nicht durch den Programmierer vorgenommen werden können. 
Der Compiler kann verschiedene Techniken anwenden, zum Beispiel Teile des Zwischencodes (engl. \textit{Intermediate Representation}) durch effizientere zu ersetzen.
Findet diese Ersetzung sequentiell statt, tritt das sogenannte \textit{Phase Ordering Problem} auf. Dabei endet die Kette an Optimierungen in einem lokalen Optimum~\cite{phaseorder-2009}.
Neben Backtracking- und Brute-Force-Methoden sind \textit{E-Graphs} eine Möglichkeit, dieses Problem zu umgehen. 

\noindent Die Abkürzung \textit{E-Graph} steht für \textit{Equality-} oder \textit{Equivalence-Graph} und beschreibt eine Datenstruktur, die es
erlaubt, äquivalente Ausdrücke kompakt abzuspeichern.
Ursprünglich wurden sie in der automatisierten Beweisführung eingesetzt~\cite{2021-egg}. Heute findet sie, auch dank \textit{Equality Saturation}, Anwendung in Compilern.
Als Beispiel dient \textit{Cranelift}~\footnote{\hspace{1.5mm}\url{https://cranelift.dev/}}, ein Compiler Backend, das \textit{E-Graphs} für die Optimierung verwendet.

\noindent \textit{E-Graphs} und \textit{Equality Saturation} sind nicht-triviale Konzepte, die indessen an Wichtigkeit zunehmen. 
Das schrittweise Heranführen von Studentinnen und Studenten an diese Themen erfordert eine Methode, die interaktiv funktioniert, da das Anwenden des Gelernten 
das Erlernen von Konzepten erleichtert. Damit soll sich diese Bachelorarbeit beschäftigen. 

\subsection{Ziel der Arbeit}

\vspace{5mm}
\begin{center}
    {\itshape
    \rmfamily
    \glqq Show me and I forget.
    Teach me and I remember. 
    Involve me and I learn.\grqq}
    \vspace{-3mm}
    \begin{flushright}
        \footnotesize
        --- Vermeintlich 
        Benjamin Franklin \\
        zugeschrieben
    \end{flushright}
\end{center}\vspace{3mm}

Das Ziel dieser Bachelorarbeit ist es, eine Anwendung für die Lehre zu entwickeln, die Studentinnen und Studenten Wissen zu den Themen \textit{E-Graphs} und \textit{Equality Saturation}
vermittelt. Dabei sollen sie die Möglichkeit haben, sich sowohl auf theoretischer als auch praktischer Ebene mit den Themen auseinander setzen zu können.
Die Grundlage der theoretischen Ebene bildet diese Arbeit, in der notwendige Hintergrundkenntnisse erarbeitet werden. Außerdem wird ein Einblick in die Implementierung gegeben. 
Die praktische Ebene besteht aus der Anwendung, mit deren Hilfe Schritt für Schritt aufgezeigt wird, wie \textit{E-Graphs} und \textit{Equality Saturation} funktionieren.
Für grö{\ss}tmöglichen Nutzen soll die Anwendung zudem plattformunabhängig sein und möglichst nur von \textit{Open-Source-Software} (OSS) Gebrauch machen.
Damit wird das Problem der unterschiedlichen Betriebssysteme der Studenten umgangen und zeitgleich die Hürden für Erweiterungen gesenkt.

\subsection{Verwandte Arbeiten}\label{sub:verwandtearbeiten}

\noindent\textbf{egg} Das Akronym \textit{egg} bildet sich aus den Wörtern \textit{e-graphs good} und ist eine in der Programmiersprache \textit{Rust} geschriebene Bibliothek.
Die Bibliothek implementiert die Datenstruktur \textit{E-Graph} und stellt Funktionen zur Manipulation dieser zur Verfügung.
Neben der Erzeugung von \textit{E-Graphs} kann auch \textit{Equality Saturation} auf diesen ausgeführt werden.
Das gesamte Projekt wurde in einem Paper vorgestellt, auf das sich diese Arbeit teilweise stützt~\cite{2021-egg}.

\noindent\textbf{egglog} Das weiterführende System \textit{egglog} basiert auf \textit{egg} und \textit{Datalog}. 
\textit{Datalog} ist eine \glqq [...] rekursive Abfragesprache für Datenbanken [...] \grqq~\cite{2023-egglog}.
Es gibt eine Webdemo, mit der das System ausprobiert werden kann~\footnote{\hspace{1.5mm}\url{https://egraphs-good.github.io/egglog/}}.
Auch dieses Projekt wurde in einem Paper vorgestellt~\cite{2023-egglog}.


\subsection{Aufbau der Arbeit}

\noindent {\itshape Kapitel \ref{sec:grundlagen} - Grundlagen}\vspace{-2mm}

In Kapitel zwei werden die nötigen theoretischen Grundlagen vermittelt. Beginnend mit einem naiven Ansatz zur Optimierung, werden schrittweise Verbesserungen
eingeführt, die schließlich in der Erstellung von \textit{E-Graphs} und das Anwenden von \textit{Equality Saturation} resultieren. 

\vspace{6mm}

\noindent {\itshape Kapitel \ref{sec:architektur} - Architektur der Anwendung}\vspace{-2mm}

Im dritten Kapitel wird die Architektur der Anwendung skizziert sowie die Intention hinter einzelnen Komponenten erläutert. Zusätzlich wird die Kommunikation zwischen 
Server und Weboberfläche beleuchtet. 

\vspace{6mm}

\noindent {\itshape Kapitel \ref{sec:entscheidungen} - Entscheidungen}\vspace{-2mm}

Der Fokus im vierten Kapitel liegt auf den getroffenen Entscheidungen hinsichtlich Softwarelösungen, die in der Arbeit zum Einsatz kamen. Dazu werden Argumente für deren Verwendung erörtert.

\vspace{6mm}

\noindent {\itshape Kapitel \ref{sec:entwicklung} - Entwicklung der Anwendung}\vspace{-2mm}

In Kapitel fünf wird ein detailierter Einblick in die Vorgehensweise während der Entwicklung ermöglicht. Außerdem wird die praktische Umsetzung aller Komponenten erklärt.

\vspace{6mm}

\noindent {\itshape Kapitel \ref{sec:probleme} - Probleme}\vspace{-2mm}

Im sechsten Kapitel werden Probleme thematisiert, zu denen es während dieser Arbeit kam. Die Ursachen der Probleme werden dabei untersucht und mögliche Lösungsansätze präsentiert. 

\vspace{6mm}

\noindent {\itshape Kapitel \ref{sec:reflexion} - Reflexion der Arbeit}\vspace{-2mm}

Gegenstand des siebten Kapitels ist eine kritische Reflexion der Arbeit. Hierzu wird analysiert, inwiefern das Ziel der Arbeit erreicht wurde und an welchen Faktoren dies gemessen werden kann.

\vspace{6mm}

\noindent {\itshape Kapitel \ref{sec:fazit} - Fazit und Ausblick}\vspace{-2mm}

Im abschließenden Kapitel wird ein Fazit der Arbeit gezogen und ein Ausblick auf mögliche Erweiterungen der Software gegeben.
