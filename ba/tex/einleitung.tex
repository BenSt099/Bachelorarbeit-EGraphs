\section{Einleitung}

\subsection{Motivation}

\noindent Die Abkürzung \textit{E-Graphs} steht für \textit{Equality-} oder \textit{Equivalence-}Graphs und beschreibt eine Datenstruktur, die es
 erlaubt, äquivalente (mathematische) Ausdrücke kompakt abzuspeichern, indem
Äquivalenzklassen der Ausdrücke abgespeichert werden. Die Äquivalenzrelation ist dabei kongruent. 
Ursprünglich wurde diese Datenstruktur nur in der automatisierten Beweisführung verwendet, wird mittlerweile aber auch für Compiler eingesetzt~\cite{2021-egg}.
Als Beispiel dient \textit{Cranelift}, ein Compiler Backend, welches \textit{E-Graphs} für die Optimierung verwendet~\cite{cranelift}.

\noindent Im Compiler gibt es verschiedene Technicken, um den Zwischencode (engl. \textit{Intermediate Representation}) zu optimieren, bevor dieser in 
Assemblersprache übersetzt wird. Optimierungen werden dabei sequentiell durchgeführt. Bei dem Ausführen von Optimierungen wird der Compiler mit dem \textit{Phase Ordering Problem}
konfrontiert. Beim \textit{Phase Ordering Problem} geht es um die Frage, in welcher Reihenfolge die Optimierungen auf das eingegebene 
Programm angewendet werden sollen. Bei einer falschen Reihenfolge kann ein lokales Optimum erreicht werden, wodurch die Effizienz nachlässt~\cite{phaseorder-2009}.

\noindent Es gibt mehrere Möglichkeiten, dieses Problem zu lösen. Neben Backtracking und Brute-Force sind \textit{E-Graphs} eine Möglichkeit, da sie durch \textit{Equality Saturation}
in der Lage sind, viele mögliche Optimierungen des Eingabe-Programms gleichzeitig effizient zu speichern und das beste aus diesen zu extrahieren.

\subsection{Ziel der Arbeit}

Das Ziel dieser Bachelorarbeit ist es, ein sinnvolles Werkzeug für die Lehre zu erstellen,
um Studentinnen und Studenten die Themen \textbf{E-Graphs} und \textbf{Equality Saturation} näher zu bringen.
Dabei sollen sie die Möglichkeit haben, sich sowohl auf theoretischer als auch praktischer Ebene mit E-Graphs auseinander setzen zu können.
Die theoretische Ebene soll den Studenten die notwendigen Hintergrundkenntnisse vermitteln sowie einen Enblick in die Implementierung geben.
Die praktische Ebene soll Schritt für Schritt aufzeigen, wie der \textbf{E-Graph} aufgebaut wird, und wie an diesem \textbf{Equality Saturation} durchgeführt werden kann.
Für grö{\ss}tmöglichen Nutzen soll die Anwendung plattformunabhängig sein und möglichst nur von \textit{Open-Source-Software} (OSS) Gebrauch machen.
Damit wird das Problem der unterschiedlichen Betriebssysteme der Studenten umgangen und zeitgleich die Hürden für Erweiterungen gesenkt.

\subsection{Verwandte Arbeiten}\label{sub:verwandtearbeiten}

\noindent\textbf{egg} Das Akronym \textit{egg} steht für \textit{e-graphs good} und ist eine in der Programmiersprache \textit{Rust} geschriebene Bibliothek.
Die Bibliothek implementiert die Datenstruktur E-Graph und stellt Funktionen für die Interaktion mit dieser zur Verfügung.
Neben der Erzeugung von E-Graphs kann auch \textit{Equality Saturation} auf diesen durchgeführt werden.
Das gesamte Projekt wurde in einem Paper vorgestellt, auf welches sich diese Arbeit stützt~\cite{2021-egg}.

\noindent\textbf{egglog} Das weiterführende System \textit{egglog} basiert auf \textit{egg} und \textit{Datalog}. 
\textit{Datalog} ist eine \glqq [...] rekursive Abfragesprache für Datenbanken [...] \grqq~\cite{2023-egglog}.
Es gibt eine Webdemo, mit der das System ausprobiert werden kann~\footnote{\url{https://egraphs-good.github.io/egglog/}}.
Auch dieses Projekt wurde in einem Paper vorgestellt~\cite{2023-egglog}.


\subsection{Aufbau der Arbeit}

\noindent {\itshape Kapitel \ref{sec:grundlagen} - Grundlagen}\vspace{-2mm}

Dieses Kapitel zeichnet den Beginn der Arbeit, indem notwendige Kenntnisse über E-Graphs und Equality Saturation vermittelt werden.
\vspace{6mm}

\noindent {\itshape Kapitel \ref{sec:architektur} - Architektur}\vspace{-2mm}

Kapitel drei gibt einen Überblick über die einzelnen Komponenten der Anwendung und zeigt das Zusammenspiel dieser untereinander.
\vspace{6mm}

\noindent {\itshape Kapitel \ref{sec:entwicklung} - Entwicklung der Anwendung}\vspace{-2mm}

Kapitel vier ermöglicht einen detailierten Einblick in die Vorgehensweise während der Entwicklung und beschreibt, wie einzelne Komponenten praktisch umgesetzt wurden.
\vspace{6mm}

\noindent {\itshape Kapitel \ref{sec:entscheidungenundprobleme} - Entscheidungen und Probleme}\vspace{-2mm}

Kapitel fünf behandelt alle Entscheidungen, die während der Entwicklungsphase getroffen wurden sowie etwaige Probleme, die währenddessen auftraten.
\vspace{6mm}

\noindent {\itshape Kapitel \ref{sec:reflektion} - Reflektion}\vspace{-2mm}

Kapitel sechs erläutert, inwiefern das Ziel der Arbeit erreicht wurde und an welchen Faktoren der Erfolg gemessen werden kann.
\vspace{6mm}

\noindent {\itshape Kapitel \ref{sec:fazit} - Fazit und Ausblick}\vspace{-2mm}

Das letze Kapitel schließt mit einem Fazit der Arbeit und gibt einen Ausblick auf mögliche Erweiterungen der Software.