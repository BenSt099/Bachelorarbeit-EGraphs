\section{Einleitung}\label{sec:einleitung}

In den folgenden Abschnitten wird das Thema der Arbeit dargestellt sowie ihr Ziel definiert.
Danach wird auf verwandte Arbeiten eingegangen.
Abschließend wird im letzten Abschnitt der Aufbau der Arbeit geschildert.

\subsection{Motivation}

\noindent In der Informatik nimmt das Optimieren von Ausdrücken eine wichtige Rolle ein. 
Diese können in Form von mathematischen Termen vorkommen oder als Programmcode auftreten.
Häufig ist der Compiler für diese Optimierung zuständig, da manche Optimierungen nicht durch den Programmierer vorgenommen werden können. 
Der Compiler kann verschiedene Techniken anwenden, zum Beispiel Teile des Zwischencodes (engl. \textit{Intermediate Representation}) durch effizientere zu ersetzen.
Findet diese Ersetzung sequentiell statt, tritt das sogenannte \textit{Phase Ordering Problem} auf. 
Dabei werden Ersetzungen basierend auf dem derzeitigen optimalen Ergebnis vorgenommen, ohne zukünftige Verbesserungen miteinzubeziehen.
Die Kette an Optimierungen kann daher in einem lokalen Optimum enden~\cite{phaseorder-2009}.
Neben Backtracking- und Brute-Force-Methoden sind \textit{E-Graphs} eine Möglichkeit, dieses Problem zu umgehen. 

\noindent Die Abkürzung \textit{E-Graph} steht für \textit{Equality-} oder \textit{Equivalence-Graph} und beschreibt eine Datenstruktur, die es
erlaubt, äquivalente Ausdrücke kompakt abzuspeichern.
Ursprünglich wurden sie in der automatisierten Beweisführung eingesetzt~\cite{2021-egg}. Heute finden sie, auch dank \textit{Equality Saturation}, Anwendung in Compilern, in 
Deep Learning Verfahren~\cite{yang2021} und in linearer Algebra~\cite{wang2020}.
Als Beispiel aus dem Compilerbau dient \textit{Cranelift}~\footnote{\hspace{1.5mm}\url{https://cranelift.dev/}}, ein Compiler Backend, das E-Graphs für die Optimierung verwendet.

\noindent \textit{E-Graphs} und \textit{Equality Saturation} sind nicht-triviale Konzepte, die in den letzte Jahren wieder an Wichtigkeit gewonnen haben~\cite[S.~2]{eqsatexploration}.
Das schrittweise Heranführen von Studentinnen und Studenten an diese Themen erfordert eine Methode, die interaktiv funktioniert, da die unmittelbare Anwendung des Gelernten 
das Erlernen von Konzepten erleichtert. Dies wird in der vorliegenden Bachelorarbeit thematisiert. 
\newpage

\subsection{Ziel der Arbeit}

\vspace{5mm}
\begin{center}
    {\itshape
    \rmfamily
    \glqq Show me and I forget.
    Teach me and I remember. 
    Involve me and I learn.\grqq}
    \vspace{-3mm}
    \begin{flushright}
        \footnotesize
        --- Vermeintlich 
        Benjamin Franklin \\
        zugeschrieben~\cite{franklin}
    \end{flushright}
\end{center}\vspace{3mm}

Ziel dieser Bachelorarbeit ist es, eine Anwendung für die Lehre zu entwickeln, die Studentinnen und Studenten Wissen zu den Themen \textit{E-Graphs} und \textit{Equality Saturation}
vermittelt. Dabei sollen sie die Möglichkeit haben, sich sowohl auf theoretischer als auch auf praktischer Ebene mit den Themen auseinander setzen zu können.
Die Grundlage der theoretischen Ebene bildet diese Arbeit, in der notwendige Hintergrundkenntnisse erarbeitet werden. Außerdem wird ein Einblick in die Implementierung gegeben. 
Die praktische Ebene besteht aus der Anwendung, mit deren Hilfe Schritt für Schritt aufgezeigt wird, wie E-Graphs und Equality Saturation funktionieren.
Für grö{\ss}tmöglichen Nutzen soll die Anwendung zudem plattformunabhängig sein und möglichst nur von \textit{Open-Source-Software} (OSS) Gebrauch machen.
Damit wird das Problem der unterschiedlichen Betriebssysteme der Studentinnen und Studenten umgangen und zeitgleich die Hürden für Erweiterungen gesenkt.

\subsection{Verwandte Arbeiten}\label{sub:verwandtearbeiten}

Die folgenden Arbeiten haben sich bereits mit den Themen E-Graphs und Equality Saturation beschäftigt und dienen als Grundlage dieser Arbeit.

\noindent\textbf{egg} Das Akronym \textit{egg} bildet sich aus den Wörtern \textit{e-graphs good} und ist eine in der Programmiersprache \textit{Rust} geschriebene Bibliothek.
Die Bibliothek implementiert die Datenstruktur E-Graph und stellt Funktionen zur Manipulation dieser zur Verfügung.
Neben der Erzeugung von E-Graphs kann auch Equality Saturation auf diesen ausgeführt werden.
Da \textit{egg} keine Möglichkeit zur schrittweisen Ausführung von Befehlen bietet, wurde auf eine eigene Implementierung gesetzt. Die Implementierung stützt sich dabei auf ein Paper~\cite{2021-egg}, in dem 
das Projekt und insbesondere auch der Pseudocode vorgestellt werden.

\noindent\textbf{Intro to EGraphs} In diesem Google Colab Notebook~\cite{devito} wurde ein Prototyp basierend auf dem Paper zu \textit{egg} erstellt.
Das Notebook stammt von Zachary Devito und zeigt darüber hinaus die Implementierung von Methoden, zu denen es im Paper zu \textit{egg} keine weiteren Informationen gibt. 
Daher wurden einige der Methoden in dieser Arbeit übernommen.

\subsection{Aufbau der Arbeit}

In diesem Kapitel wurden die Motivation der Arbeit, ihr Ziel und themenverwandte Arbeiten dargelegt. Die restlichen Kapitel sind wie folgt aufgebaut:

\noindent {\itshape Kapitel \ref{sec:grundlagen} - Grundlagen}\vspace{-2mm}

In Kapitel 2 werden die notwendigen theoretischen Grundlagen vermittelt. Beginnend mit einem naiven Ansatz zur Optimierung, werden schrittweise Verbesserungen
eingeführt, die schließlich in der Erstellung von E-Graphs und in der Anwendung von Equality Saturation resultieren. 

\vspace{6mm}

\noindent {\itshape Kapitel \ref{sec:entscheidungen} - Technologiestack}\vspace{-2mm}

Der Fokus im dritten Kapitel liegt auf den getroffenen Entscheidungen hinsichtlich der Softwarelösungen, die in der Arbeit zum Einsatz kamen. Dazu werden Argumente für deren Verwendung erörtert.

\vspace{6mm}

\noindent {\itshape Kapitel \ref{sec:architektur} - Architektur der Anwendung}\vspace{-2mm}

Im vierten Kapitel wird die Architektur der Anwendung skizziert sowie die Intention hinter einzelnen Komponenten erläutert. Zusätzlich wird die Kommunikation zwischen 
Server und Weboberfläche beleuchtet. 

\vspace{6mm}

\noindent {\itshape Kapitel \ref{sec:entwicklung} - Entwicklung der Anwendung}\vspace{-2mm}

In Kapitel 5 wird ein detailierter Einblick in die Vorgehensweise während der Entwicklung ermöglicht. Außerdem wird die praktische Umsetzung aller Komponenten erklärt.

\vspace{6mm}

\noindent {\itshape Kapitel \ref{sec:reflexion} - Reflexion der Arbeit}\vspace{-2mm}

Gegenstand des sechsten Kapitels ist eine kritische Reflexion der Arbeit. Hierzu wird analysiert, welche Komplikationen während der Arbeit auftraten,
inwiefern das Ziel der Arbeit erreicht wurde und an welchen Faktoren dies gemessen werden kann.

\vspace{6mm}

\noindent {\itshape Kapitel \ref{sec:fazit} - Fazit und Ausblick}\vspace{-2mm}

Im abschließenden Kapitel wird ein Fazit zu der Arbeit gezogen und ein Ausblick auf mögliche Erweiterungen der Software gegeben.
