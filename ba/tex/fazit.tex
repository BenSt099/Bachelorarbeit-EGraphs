\section{Fazit und Ausblick}\label{sec:fazit}

\subsection{Fazit}

Die in dieser Arbeit entwickelte Lernanwendung gibt Studentinnen und Studenten die Möglichkeit, sich auf interaktive Weise mit den Themen \textbf{E-Graphs} und 
\textbf{Equality Saturation} auseinander zu setzen. Dabei können sie \textit{E-Graphs} erstellen, \textit{rewrite rules} hinzufügen und Schritt für Schritt 
auf den \textit{E-Graph} anwenden sowie einen optimalen Term extrahieren. Die Anwendung ist getestet und funktioniert auf den gängigen Plattformen.

\subsection{Ausblick}

Aufgrund der zeitlichen Beschränkungen konnten verschiedene Ansätze und Weiterentwicklungen nicht mehr ausprobiert und in die Anwendung integriert werden.
Nachfolgend sind mehrere Aspekte aufgelistet, die als Erweiterung denkbar wären. 

\textbf{Erweitertes Testing} Momentan ist das Testen von E-Graphs aufwendig, da auf die internen Datenstrukuren zugegriffen werden muss.
Denkbar wäre hier die Implementierung einer Vergleichsmethode, die für einen gegebenen Ausdruck die beiden E-Graphs vergleicht,
die mit der eigenen Implementierung und mit der von \textit{egg}~(~\cite{2021-egg}) erstellt wurden. 
Dadurch lässt sich die Qualität der Software einfacher überprüfen und Fehler können schneller erkannt werden.
Mit drei Wochen Zeit wäre zu rechnen.

\textbf{Erstellung von E-Graphs visualisieren} Die Visualisierungen setzen bei der Anwendung von \textit{rewrite rules} auf den E-Graph an.
Möglich wäre hier eine detailierte Schritt für Schritt Darstellung, die auch den Aufbau eines E-Graphs zeigt. Je nach Detailgrad könnte dies in ein bis zwei Wochen umgesetzt werden.

\textbf{Anbindung an egg} Die Anwendung arbeitet zurzeit mit einer eigenen Implementierung von E-Graphs. Dem Benutzer könnte eine weitere Option zur Verfügung stehen, eine weitere
Implementierung, zum Beispiel die von \textit{egg}~(~\cite{2021-egg}), als Basis zu nutzen.
Für dieses Unterfangen müsste jedoch der Debug-Output von \textit{egg} geparst und verarbeitet werden oder die Methoden der Bibliothek selbst müssten modifiziert werden.
Dabei kann der Umstand ausgenutzt werden, dass Methoden, die in \textit{Rust} geschrieben wurden, mit wenigen Modifikationen auch von \textit{Python} aus aufgerufen werden können.
Für dieses Vorhaben müsste wahrscheinlich sechs Wochen eingeplant werden.

\textbf{EClass Analysis}
Die Implementierung ist zurzeit nicht in der Lage eine \textit{EClass Analysis} durchzuführen. Bei dieser Art von Analyse kommen mehrere Techniken zum Einsatz, darunter
\textit{Conditional and Dynamic rewrites} und \textit{Constant Folding}. Zwei bis drei Wochen wären vermutlich für die Umsetzung nötig.
