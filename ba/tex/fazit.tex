\section{Fazit und Ausblick}\label{sec:fazit}

\subsection{Fazit}

\subsection{Ausblick}

Momentan ist das Testen von EGraphs aufwendig. Vorstellbar wäre hier die Implementierung einer Vergleichsmethode, die für einen gegebenen Ausdruck die beiden EGraphs vergleicht,
die mit der eigenen Implementierung und mit der von \textbf{egg} erstellt wurden. Dadurch lässt sich die Qualität der Software einfacher überprüfen und Fehler können schneller
erkannt werden.

\noindent Die Visualisierung von EGraphs setzt bei Anwendung der RewriteRules an. Denkbar wäre eine detailierte Schritt-für-Schritt Darstellung vom Aufbau eines EGraphs.
Je nach Detailgrad könnte dies in drei bis vier Wochen umgesetzt werden.

\noindent Eine Anbindung der Anwendung an \textbf{egg} wäre ebenfalls denkbar. Dafür wären jedoch \textit{Rust}-Kenntnisse erforderlich. An sich wäre dies umsetzbar, da 
\textit{Rust}-Methoden, mit wenigen Modifikationen, auch von Python aus aufgerufen werden können.
Der Zeitaufwand könnte hier sechs Wochen betragen.

\noindent Das System ist zurzeit nicht in der Lage, eine Analyse durchzuführen, um sicher zu gehen, dass RewriteRules mathematisch sicher sind. Das bedeutet zum Beispiel, dass 
bei einer Divisions-Regel nicht überprüft wird, ob der Nenner eventuell null werden könnte. 
Auch \textit{Constant Folding} lässt sich erreichen.
Diese Art der Analyse wird \textit{EClass-Analysis} genannt.  