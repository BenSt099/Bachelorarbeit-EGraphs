\section{Fazit und Ausblick}\label{sec:fazit}

In diesem Kapitel wird das Ergebnis der Arbeit präsentiert und ein abschließendes Fazit gezogen. Zum Schluss werden mögliche Erweiterungen der Software aufgezeigt. 

\subsection{Fazit}

Momentan werden Studentinnen und Studenten Lehrinhalte an Universitäten durch Vorlesungen vermittelt. Das Ziel der vorliegenden Arbeit war es, eine Lernsoftware für die Lehre
zu entwickeln, mit deren Hilfe die Konzepte \textbf{E-Graphs} und \textbf{Equality Saturation} auf interaktive Weise erlernt und vertieft werden können.
Als Vorbereitung und Einführung in beide Konzepte können Studentinnen und Studenten Kapitel~\ref{sec:grundlagen} dieser Arbeit nutzen, in dem die theoretischen Grundlagen erörtert werden.
Anschließend können sie die Anwendung herunterladen und starten. 

Die Anwendung öffnet sich in einem Browser-Fenster und bietet eine Website als Benutzeroberfläche. Über diese können Studentinnen und Studenten einen E-Graph erzeugen und visualisieren, 
\textit{rewrite rules} hinzufügen und auf diesen anwenden und Equality Saturation durchführen. 
Dabei können sie während der Ausführung mithilfe des Debugging-Features die Mechanismen hinter diesen Konzepten beobachten und bei Bedarf die Ausführung auch wiederholen.
Eine Dokumentation erklärt ihnen dabei, wie die Anwendung bedient werden soll. Zusätzlich kann die Sitzung und die \textit{rewrite rules} in einer Datei abgespeichert werden, um
sie später wiederherstellen zu können. Auch der E-Graph kann in gängige Formate wie PDF exportiert werden.
Für das problemlose Benutzen der Software ist sie plattformunabhängig gestaltet worden und setzt auf Open-Source-Software.





Insgesamt konnte anhand dieser Arbeit aufgezeigt werden, wie beim Entwickeln einer Lernsoftware vorgegangen werden sollte, wie Inhalte verständlich vermittelt werden können und welche
Anforderungen die Software erfüllen muss.
Damit leistet diese Arbeit einen Beitrag zur Fortentwicklung der Lehre und kann als Paradigma für vergleichbare Projekte genutzt werden.

\subsection{Ausblick}

Aufgrund der zeitlichen Beschränkungen konnten verschiedene Ansätze und Weiterentwicklungen nicht mehr ausprobiert und in die Anwendung integriert werden.
Nachfolgend sind mehrere Aspekte aufgelistet, die als Erweiterung denkbar wären. 

\textbf{Erweitertes Testing} Momentan ist das Testen von E-Graphs aufwendig, da auf die internen Datenstrukuren zugegriffen werden muss.
Denkbar wäre hier die Implementierung einer Vergleichsmethode, die für einen gegebenen Ausdruck die beiden E-Graphs vergleicht,
die mit der eigenen Implementierung und mit der von \textit{egg} (\cite{2021-egg}) erstellt wurden. 
Dadurch lässt sich die Qualität der Software einfacher überprüfen und Fehler können schneller erkannt werden.
Mit zwei bis drei Wochen wäre hier vermutlich zu rechnen.

\textbf{Erstellung von E-Graphs visualisieren} Die Visualisierungen setzen bei der Anwendung von \textit{rewrite rules} auf den E-Graph an.
Möglich wäre hier eine detailierte Schritt für Schritt Darstellung, die auch den Aufbau eines E-Graphs zeigt. 
Je nach Detailgrad könnte dies in ein bis zwei Wochen umgesetzt werden.

\textbf{Anbindung an egg} Die Anwendung arbeitet zurzeit mit einer eigenen Implementierung von E-Graphs. Dem Benutzer könnte eine weitere Option zur Verfügung stehen, eine weitere
Implementierung, zum Beispiel die von \textit{egg}~(~\cite{2021-egg}), als Basis zu nutzen.
Für dieses Unterfangen müsste jedoch der Debug-Output von \textit{egg} geparst und verarbeitet werden oder die Methoden der Bibliothek selbst müssten modifiziert werden.
Dabei kann der Umstand ausgenutzt werden, dass Methoden, die in \textit{Rust} geschrieben wurden, mit wenigen Modifikationen auch von \textit{Python} aus aufgerufen werden können.
Für dieses Vorhaben müsste wahrscheinlich ein ganzer Monat eingeplant werden.

\textbf{EClass Analysis}
Die Implementierung ist zurzeit nicht in der Lage eine \textit{EClass Analysis} durchzuführen. Bei dieser Art von Analyse kommen mehrere Techniken zum Einsatz, darunter
\textit{Conditional and Dynamic rewrites} und \textit{Constant Folding}~\cite{2021-egg}. Zwei bis drei Wochen wären mutmaßlich für die Umsetzung nötig.
