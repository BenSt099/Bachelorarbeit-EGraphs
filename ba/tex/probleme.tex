\section{Probleme}\label{sec:probleme}

In diesem Kapitel werden Probleme beschrieben, die während der Entwicklung auftraten und wie diese gelöst werden konnten.

\noindent\textbf{Kombination zweier Implementierungen} Die Idee hinter der Kombination der beiden Implementierungen war.





\noindent\textbf{DOT-Format} Fpr die eigene Implementierung wurde eine Methode benötigt, die \textit{E-Graphs} visualisieren kann.
Eine entsprechende Methode (egraph\_to\_dot()) gibt es in der Datei \textit{EGraph.py}. Obwohl sich im Rahmen dieser Arbeit an die
PEP 8 Vorgaben gehalten wurde, bildet diese Methode eine Ausnahme. Sie ist über 100 Zeilen lang und Bedarf einer langen Erklärung.
Auch die Abdeckung durch Tests ist nicht ideal, denn das Testen der Methode bedarf einer manuellen Ausführung der Tests mit anschließender Prüfung der Visualisierungen
durch einen Menschen. Dieses Vorgehen hatte mehrere Gründe.

Im Verlauf der Entwicklung hat sich herausgestellt, dass das \textit{DOT-Format} keine Reihenfolge zwingend einhält. Damit kann es unter Umständen zu
einer falschen Darstellung kommen, da die Pfeile verkehrt herum auf die Klassen zeigen.
Zudem ergibt sich ein Fehler in der Darstellung durch einen Spezialfall, der weiter unten erläutert wird.



- PEP 8 als Quelle in references.bib


\noindent\textbf{Spezialfall: Kreis im E-Graph}



