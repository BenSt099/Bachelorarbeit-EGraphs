\section{Probleme}\label{sec:probleme}

In diesem Kapitel werden einige der Probleme, die während der Entwicklung auftraten, diskutiert und deren Ursachen sowie Lösungsansätze beleuchtet.

\noindent\textbf{Kombination zweier Implementierungen} Wie in Kapitel~\ref{sec:grundlagen} erläutert, basiert die vorliegende Implementierung
auf der Kombination zweier bestehender Ansätze. Für diese Entscheidung gab es mehrere Gründe.
Einerseits basieren viele Implementierungen, darunter auch \textit{egg}, auf der Standarddefintion von \textit{E-Graphs} wie sie auch in Kapitel~\ref{sec:grundlagen} vorliegt.
Das Google Colab Notebook, das als zweite Quelle dient, setzt aber bereits auf eine modifzierte Variante, was eventuell zu Verwirrung unter den Studentinnen und Studenten
führen würde. Daher wurde die Implementierung von \textit{egg} der zweiten Quelle vorgezogen.

Andererseits beschreibt das Paper, das die Hintergründe zu \textit{egg} beschreibt~\cite{2021-egg}, nicht genau, wie \textit{E-Matching} und \textit{Extraction} funktionieren.  
Diese beiden Konzepte konnte allerdings das Notebook zur Verfügung stellen. Somit komplementieren sich die beiden Ansätze.
Aus der Kombination resultierten jedoch auch einige Probleme. Neben der \textit{Darstellung der Datenstruktur} (siehe unten) mussten mehrere Sonderfälle in der \textit{add}-Methode
berücksichtigt werden, sodass die Methode insgesamt länger wurde als in den beiden Ansätzen vorgesehen.

\noindent\textbf{Darstellung der Datenstruktur} Nur eine der zwei Implementierungen, auf denen diese Arbeit fußt, verfügt über eine Methode zur Visualisierung der Datenstruktur.
Die Methode ist kurz und effizient programmiert. Die Darstellung hingegen weist einige Schwächen auf, besonders im Hinblick auf Verständlichkeit und Klarheit.
Diese beiden Aspekte sind jedoch für eine Anwendung, die von unerfahrenen oder fachfremden bedient werden soll, essenziell.
Außerdem wurde die Methode durch die Kombination der beiden Implementierungen unbrauchbar.
Dementsprechend wurde eine eigene Methode \textit{egraph\_to\_dot()} in der Datei \textit{EGraph.py} angefertigt.

Im Verlauf der Entwicklung hat es sich als schwierig herausgestellt, die Reihenfolge der \textit{E-Classes} im \textit{Dot}-Format zu kontrollieren.
Damit kann es zu einer falschen Darstellung kommen, da eventuell die Pfeile einer \textit{E-Node} verkehrt herum auf die Klassen zeigen.
Zusätzlich musste, aufgrund des Debugging-Features, Code eingefügt werden, der es ermöglicht, \textit{E-Classes} einzufärben.
Diese Anforderungen und Sonderfälle resultierten schließlich in einer Methode mit recht kompliziertem Code, der 106 Zeilen lang ist.

Daher wurde bei dieser Methode der \textit{PEP 8} Style Guide~\cite{pep} vernachlässigt sowie die Formatierung durch das Werkzeug \textit{Black}~\cite{black} unterlassen.
Das Testen dieser Methode gestaltete sich ebenfalls schwierig, da der Output ein String im \textit{Dot}-Format ist. Entsprechend bedarf das Testen der Methode
einer manuellen Ausführung der Tests mit anschließender Prüfung der Visualisierungen durch einen Menschen.

Zuletzt trat ein weiterer Fehler in der Darstellung auf, wenn der Sonderfall (siehe Kreis im E-Graph, weiter unten) eintritt. 
Dabei wurden zwei Pfeile, die beide auf die gleiche Klasse zeigen, nicht links und rechts der \textit{E-Node}, sondern beide auf der linken Seite gezeichnet.
Allerdings handelt es sich hierbei nur um einen minimalen Fehler, der nicht zu einem Missverständnis der Themen beiträgt.

\noindent\textbf{Spezialfall: Kreis im E-Graph} \textit{E-Graphs} könne Kreise enthalten. Als Beispiel sei der \textit{E-Graph} gegeben, der nur $a$ enthält.
Wendet man nun die Regel $x \Leftrightarrow x * 1$ darauf an, ergibt dies einen Kreis. Wendet man die Regel erneut an,  



