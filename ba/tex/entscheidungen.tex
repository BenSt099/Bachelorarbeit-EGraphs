\section{Technologiestack}\label{sec:entscheidungen}

In den folgenden Abschnitten wird der Technologiestack der Arbeit vorgestellt. Die Auswahl der eingesetzten Software wird erklärt und ihre Vorteile werden gegenüber Alternativen hervorgehoben.

\subsection{Python}

Die Auswahl einer Programmiersprache für die Durchführung eines Projektes entscheidet über dessen Potential und Einschränkungen.
Zudem bietet jede Programmiersprache unterschiedliche Herangehensweisen und Werkzeuge, um Probleme lösen zu können.
Auf Grundlage dieser Gegebenheiten wurde die Programmiersprache \textit{Python} in der Version \textit{3.12} gewählt.
Ausschlaggebend dafür waren folgende Gründe:

\begin{itemize}
    \item Obwohl die Programmiersprache Python einen engen Bezug zu Data Science, Machine Learning und ähnlichen Themen hat, gilt sie dennoch als Allzweck-Sprache,
    die in vielen Situationen Anwendung finden kann. Sie ist außerdem weit verbreitet und wird auf allen gängigen Plattformen unterstützt, sodass die Plattformunabhängigkeit garantiert ist. 
    Die Standardbibliothek bietet viele Funktionen und es gibt zahlreiche Bibliotheken, die eine Reihe von verschiedenen Themengebieten behandeln.
    \item Python erlaubt durch seine unkomplizierte Speicherverwaltung und eingebaute Garbage Collection das schnelle Konstruieren und Testen von Prototypen.
    Damit entfällt weniger Zeit auf Implementierungsdetails, wie Speicherverwaltung, und mehr Zeit auf die Umsetzung des übergeordneten Ziels.
\end{itemize}

\subsection{Graphviz}

Um das Ziel der Anwendung zu erreichen, muss dem Benutzer die Datenstruktur E-Graph visualisiert werden können. 
Durch die graphenähnliche Struktur der Datenstruktur, die in drei verschiedenen Datenstrukturen festgehalten ist, wurde die Software 
\textit{Graphviz}~\footnote{\hspace{1.5mm}\url{https://graphviz.org/}} ausgewählt. Folgende Gründe sprachen dabei für die Entscheidung:

\begin{itemize}
    \item Die Software Graphviz ist auf allen gängigen Plattformen verfügbar, weitreichend getestet und Open Source. Außerdem hat sie sich in einer Vielzahl von Anwendungsgebieten bewährt.
    \item Sie bietet eine allgemeine, abstrakte Grammatik (die sogenannte \textit{DOT}-Sprache) zur Beschreibung von Graphen an. Die \textit{DOT}-Sprache ist, ähnlich wie andere Sprachformate, wie
    zum Beispiel JSON oder YAML, zu einem Standard geworden. Dementsprechend gibt es auch andere Werkzeuge und Software, die dieses Sprachformat verarbeiten können.
    \item Es gibt viele Pakete, die auf Graphviz aufbauen, darunter das Python-Paket \textit{graphviz}~\footnote{\hspace{1.5mm}\url{https://pypi.org/project/graphviz/}}, was nativen Anschluss an die Software ermöglicht. 
    Daher kann auf die Interaktion mit Graphviz über ein CLI verzichtet werden.
    \item Graphviz kennt viele gängige Ausgabeformate, wie SVG, PDF oder PNG, in die ein im DOT-Format beschriebener Graph exportiert werden kann.
    \item Ein E-Graph muss im Browser des Benutzers angezeigt werden können. Hierzu existiert die JavaScript-Bibliothek \textit{D3}~\footnote{\hspace{1.5mm}\url{https://d3js.org/}}, die, im Zusammenspiel
    mit einer entsprechenden Bibliothek wie \textit{d3-graphviz}~\footnote{\hspace{1.5mm}\url{https://github.com/magjac/d3-graphviz}}, in der Lage ist, einen E-Graph in dem DOT-Format entgegen zu nehmen
    und als SVG zu rendern. Das \textit{SVG}-Format eignet sich dabei sehr gut für die Darstellung auf Websiten.
\end{itemize}

\subsection{FastAPI}

Für die Interaktion mit dem Benutzer stehen eine Vielzahl von Optionen zur Verfügung, zum Beispiel ein CLI, eine Website im Browser oder ein Anwendungsfenster.
Im Rahmen dieser Arbeit erschien eine Website, die lokal im Browser läuft, am sinnvollsten. Diese Entscheidung beruht auf folgenden Argumenten:

\begin{itemize}
    \item Es existieren umfangreiche GUI-Bibliotheken für Python, darunter PyQt, PySide oder TKinter. Jede von ihnen weist jedoch plattformspezifische Probleme und Einschränkungen auf.
    Darüber hinaus erscheint die Generierung von dynamischem Inhalt im Browser durch JavaScript (oder darauf basierenden Bibliotheken) deutlich einfacher. Auch die Eigenschaft moderner Browser,
    das SVG-Format problemos und ohne weitere Dependencies anzeigen zu können, hat sich als Vorteil erwiesen.
    \item Eine Website benötigt einen Server im Backend. Zudem stehen hier zahlreiche Frameworks zur Verfügung, darunter Flask, Django und FastAPI.
    \textit{Django} ist weit verbreitet und bei Entwicklern sehr beliebt, weist aber auch Nachteile auf. Unter anderem bietet das Framework ein umfangreiches Angebot an, das viele Anwendungsfälle abdeckt.
    Dadurch ist die Performance ein oft genannter Kritikpunkt. 
    Im Gegensatz dazu beinhaltet \textit{FastAPI}~\footnote{\hspace{1.5mm}\url{https://fastapi.tiangolo.com/}} nur essenzielle Funktionen. Dies vereinfacht das Aufsetzen eines Servers deutlich.
\end{itemize}

\subsection{Eingesetzte Frameworks}

Die Gestaltung der Benutzeroberfläche basiert auf dem CSS-Framework \textit{Bootstrap}~\cite{bootstrap} in der Version \textit{5.3.3}.
\textit{Bootstrap} ist leicht zu integrieren und vielfach erprobt worden. Dazu bietet es bereits fertige und optimierte Bausteine, die für Konsistenz sorgen und auf Accessibility getestet sind.

Die Dokumentation wurde mit \textit{Docusaurus}~\cite{docusaurus} in der Version \textit{3.7} erstellt. \textit{Docusaurus} ist ein Static-Site Generator, der sich leicht in
bestehende Infrastruktur einfügen lässt. Er unterstützt nativ Internationalisierung. Die Dokumentation kann in Markdown geschrieben werden. Folglich eignet sich \textit{Docusaurus} für diese Arbeit.

Für die Kommunikation zwischen Server und Benutzeroberfläche wurde kein Framework gewählt. Stattdessen wurden die notwendigen Funktionen selbst implementiert. Dies erlaubt eine bessere Kontrolle über die Benutzeroberfläche sowie die
Reduzierung von unnötigem Code. Zusätzlich sind die meisten JavaScript-Frameworks zu umfangreich für den Zweck dieser Arbeit. Außerdem nimmt es viel Zeit in Anspruch, ein passendes Framework auszuwählen. Daher wurde darauf verzichtet.

Beim Testen der Anwendung wurde auf \textit{pytest}~\footnote{\hspace{1.5mm}\url{https://docs.pytest.org/}} in der Version \textit{8.3.3} gesetzt.
Das Testing-Framework pytest ist sehr bekannt, verbreitet und zeichnet sich vor allem durch seine Unkompliziertheit aus.
Des Weiteren steht pytest auch für das Testen von FastAPI-Funktionalität zur Verfügung, was das Testen simplifiziert. 

\subsection{JSON-Format}

Für den Transport von Daten zwischen Server und Benutzeroberfläche stehen verschiedene Formate zur Verfügung. In dieser Arbeit wurde das \textit{JSON}-Format eingesetzt,
weil es sowohl von Python als auch von JavaScript nativ unterstützt wird. Daten im JSON-Format können in Python ohne großen Aufwand in ein Dictionary und in JavaScript in ein Objekt umgewandelt werden.
Zusätzlich eignet sich dieses Format zur Darstellung in Browsern, zum Beispiel in den Entwicklungswerkzeugen der meisten Browser, was das Debugging vereinfacht.
