\section{Entscheidungen und Probleme}\label{sec:entscheidungenundprobleme}

\subsection{Python}

Die Wahl einer Programmiersprache entscheidet über die Möglichkeiten sowie Einschränkungen in einem Projekt.
Jede Programmiersprache bietet unterschiedliche Herangehensweisen an Probleme an und wurde teilweise für einen spezifischen Zweck entworfen.
Für dieses Projekt wurde die Allzweck-Programmiersprache \textit{Python} in der Version \textit{3.12} gewählt.
Dafür gibt es mehrere Gründe. 

\noindent Erstens liegt der Hauptfokus der Anwendung nicht auf Speichereffizienz oder Schnelligkeit, sondern vor allem auf Plattformunabhängigkeit.
Zweitens wird während der Implementierung Zeit gespart, wodurch mehr Zeit für das übergeordnete Ziel zur Verfügung steht, nämlich eine Anwendung
für die Lehre zu bauen.
Drittens ist die Programmiersprache \textit{Python} sehr bekannt und weit verbreitet. Dadurch gibt es bereits zahlreiche Bibliotheken sowie eine große
Standardbibliothek.

\subsection{DOT-Sprache}

Um einen EGraph visuell darstellen zu können, benötigt man ein Format, in dem man derartige Graphen darstellen kann. 
\textit{Graphviz} erlaubt das Exportieren von Graphen in \textit{DOT}-Sprache in mehrere Formate, wie z.B. SVG, PDF oder PNG. 
Damit kann es auch im Browser angezeigt werden.
Es gibt bereits ein Python-Packet, was den nativen Anschluss an \textit{Graphviz} ermöglicht. Damit muss man nicht über Prozesse o.ä. mit den CLI-Tools
von \textit{Graphviz} interagieren.
\textit{Graphviz} ist platformunabhängig und weitreichend getestet.

\subsection{JSON-Format}

Für die Kommunikation zwischen Server und Benutzeroberfläche wurde das \textit{JSON}-Format ausgewählt. Nicht nur ist es für Menschen lesbar und kann im Browser dargestellt werden,
was das Debugging einfacher macht, sondern es wird sowohl von JavaScript als auch von Python nativ unterstützt. Auch \textit{FastAPI} unterstützt dieses Format.

\subsection{FastAPI}

Für die Interaktion mit dem Benutzer wurde der Browser als Hauptkomponente ausgewählt. Dadurch verzichtet man auf etwaige GUI-Bibliotheken, die zum Teil
platformspezifische Probleme und Eigenschaften mit sich bringen. Für den Server im Backend gibt es viele Frameworks, darunter Flask, Django und FastAPI.
\textit{Django} ist ein umfassendes Framework, das ein breites Spektrum an Anwendungsfällen abdeckt. Im Gegensatz dazu bietet \textit{FastAPI}~\cite{fastapi}
nur essenzielle Funktionen, die für die Entwicklung benötigt werden. Da der Server in diesem Fall nur wenige, standardmäßige Aufgaben erfüllen muss, wurde
\textit{FastAPI} als Framework ausgewählt, da es für den Nutzen ausreicht.

\subsection{Bootstrap-Framework}

Für die Benutzeroberfläche im Browser wurde auf das CSS-Framework \textit{Bootstrap}~\cite{bootstrap} in der Version 5.3.3 gesetzt. Das Framework ist nicht so JavaScript-lastig
wie andere Frameworks und enthält eine Standardpalette an nützlichen und optimierten UI-Bausteinen. 
