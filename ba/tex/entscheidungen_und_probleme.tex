\section{Entscheidungen und Probleme}\label{sec:entscheidungenundprobleme}

\subsection{Python}

Die Wahl einer Programmiersprache entscheidet über die Möglichkeiten sowie Einschränkungen in einem Projekt.
Zudem bietet jede Programmiersprache unterschiedliche Herangehensweisen an Probleme an und wurde teilweise für einen spezifischen Zweck entworfen.
Für dieses Projekt wurde die Allzweck-Programmiersprache \textit{Python} in der Version \textit{3.12} gewählt.
Dafür gibt es zwei Hauptgründe. 

\noindent Erstens gilt die Programmiersprache \textit{Python} allgemein als sehr einfach. 
Das erlaubt einerseits das schnelle Konstruieren und Testen von Prototypen. 
Andererseits erlaubt es eine Fokusierung auf das übergeordnete Ziel, eine Lernanwendung zu entwickeln. Dabei müssen Implementierungsdetails, wie zum Beispiel
die Garbage Collection, nicht beachtet werden.

Zweitens ist \textit{Python} eine sehr bekannte und weit verbreitete Programmiersprache. 
Die Standardbibliothek bietet großen Umfang und es gibt zahlreiche Bibliotheken, die eine Reihe von verschiedenen Themengebieten behandeln.
Außerdem wurde \textit{Python} auf allen gängigen Betriebssystemen bereits getestet, wodurch die Platformunabhängigkeit garantiert ist.

\subsection{Graphviz}

Jeder EGraph besteht aus drei Datenstrukturen. Um diese visuell darstellen zu können, wird eine Software benötigt, die Graphen erzeugen kann.
Dazu wurde die Software \textit{Graphviz}~\footnote{\hspace{1.5mm}\url{https://graphviz.org/}} ausgewählt. Folgende Gründe waren dabei ausschlaggebend.

\begin{itemize}
    \item \textit{Graphviz} ist auf allen gängigen Plattformen verfügbar und weitreichend getestet.
    \item \textit{Graphviz} bietet eine allgemeine, abstrakte Grammatik an, die sogenannte \textit{Dot}-Sprache, mit deren Hilfe verschiedene Typen von Graphen erstellt werden können.
    \item Die \textit{Dot}-Sprache ist, ähnlich zu anderen Formaten wie JSON oder YAML, zu einem Standard geworden, der auch von anderen Werkzeugen und Software verarbeitet werden kann.
    \item Ein in der \textit{Dot}-Sprache beschriebener EGraph kann mit \textit{Graphviz} in gängige Formate wie SVG, PNG oder PDF exportiert werden.
    \item Es gibt bereits ein Python-Paket, was nativen Anschluss an \textit{Graphviz} ermöglicht. Damit kann auf das Interagieren mit \textit{Graphviz} über ein CLI verzichtet werden.
    \item Durch die JavaScript-Bibliothek \textit{D3} und einer entsprechenden Bibliothek (\textit{d3-graphviz}~\footnote{\hspace{1.5mm}\url{https://github.com/magjac/d3-graphviz}}), kann ein EGraph auch im Browser als SVG-Datei angezeigt werden.
\end{itemize}

\subsection{FastAPI}

Die Interaktion mit einem Benutzer kann auf verschiedenen Wegen erfolgen, zum Beispiel durch eine CLI, über eine Website im Browser oder durch ein Anwendungsfenster.
In dieser Arbeit stellt eine Website, die lokal im Browser läuft zum Einsatz. Diese Entscheidung ist mit folgenden Argumenten zustande gekommen. 

\begin{itemize}
    \item Es gibt umfangreiche GUI-Bibliotheken, wie zum Beispiel PyQt, PySide oder TKinter. Diese bringen jedoch plattformspezifische Probleme und Einschränkungen mit sich.
    Außerdem ist die dynamische Generierung von Inhalten im Browser mit JavaScript (oder darauf basierenden Bibliotheken) deutlich einfacher.
    \item Für einen Server im Backend gibt es ebenfalls zahlreiche Frameworks, darunter Flask, Django und FastAPI.
    Gerade \textit{Django} ist zwar sehr beliebt und weit verbreitet, hat aber auch Performance-Schwierigkeiten.
    Das liegt unter anderem daran, dass \textit{Django} ein umfangreiches Angebot anbietet und für viele Anwendungsfälle entwickelt wurde.
    Für dieses Projekt werden die meisten Features jedoch nicht benötigt.
    \item Im Gegensatz zu \textit{Django} bietet \textit{FastAPI}~\footnote{\hspace{1.5mm}\url{https://fastapi.tiangolo.com/}} nur essenzielle Funktionen an, was das Aufsetzen eines Servers deutlich vereinfacht.
\end{itemize}

\subsection{JSON-Format}

Für die Kommunikation zwischen Server und Benutzeroberfläche stehen verschiedene Formate zur Verfügung. Die Entscheidung für das \textit{JSON}-Format wurde getroffen, 
da es sowohl von JavaScript als auch von Python nativ unterstützt wird. Außerdem ist es eines der am häufigsten verwendeten Formate in Browsern.
Zudem können die in diesem Format kodierten Daten im Browser dargestellt werden, unter anderem
zum Beispiel durch die Developer Tools von Firefox. Dieses Feature macht das Debugging deutlich einfacher. 

\subsection{Bootstrap-Framework}

Für die Gestaltung der Benutzeroberfläche wurde auf das CSS-Framework \textit{Bootstrap}~\footnote{\hspace{1.5mm}\url{https://getbootstrap.com/}} in der Version 5.3.3 zurückgegriffen.
\textit{Bootstrap} kam bereits auf zahlreichen Websiten zum Einsatz und ist daher vielfach erprobt worden.
Zudem enthält es viele bereits fertige und optimierte Bausteine, die nicht nur für Konsistenz sorgen, sondern auch für Accessibility ausgelegt sind. 
