\section{Entwicklung der Anwendung}\label{sec:entwicklung}

\subsection{E-Graphs}

Um die Erstellung von EGraphs zu vereinfachen, erfolgt der Aufbau in mehreren Schritten.
Im ersten Schritt wird ein gegebenener mathematischer Ausdruck in einen \textbf{Abstract Syntax Tree} (AST) umgewandelt.
Im zweiten Schritt wird der EGraph erstellt, indem der AST rekursiv traversiert und jeder Knoten dem EGraph hinzugefügt wird. 

\subsubsection{AST}

Für einen \textbf{Abstract Syntax Tree} werden die Klassen \textit{AbstractSyntaxTreeNode} (siehe Listing~\ref{lst:astnode}) und 
\textit{AbstractSyntaxTree} (siehe Listing~\ref{lst:ast}) benötigt.

% \begin{lstlisting}[language=Python, escapechar=|, caption=Klasse \textit{AbstractSyntaxTreeNode}, label={lst:astnode}]
% class AbstractSyntaxTreeNode:
%     def __init__(self):
%         self.left = None |\label{line:sp}|
%         self.key = str()
%         self.right = None
% \end{lstlisting}

% einen linken Zeiger (Zeile~\ref{line:sp}),

\begin{lstlisting}[language=Python, caption=Klasse \textit{AbstractSyntaxTreeNode}, label={lst:astnode}]
class AbstractSyntaxTreeNode:
    def __init__(self):
        self.left = None
        self.key = str()
        self.right = None
\end{lstlisting}

Die Klasse \textit{AbstractSyntaxTreeNode} repräsentiert einen Knoten im AST. Sie hat drei Attribute: einen linken Zeiger, einen rechten Zeiger
und einen Schlüssel. Der Schlüssel kann eine arithmetische Operation, eine Variable oder eine Zahl sein. Die Zeiger können auf einen weiteren Teilbaum zeigen, auf einen einzelnen Knoten
oder auf None. Mithilfe dieser Klasse kann nachher ein ganzer Baum dargestellt werden. Dazu braucht man nur den Wurzelknoten des Baumes abspeichern, denn damit
kann auf den gesamten Baum zugegriffen werden. 

Die Klasse \textit{AbstractSyntaxTree} ist für das Erstellen des AST aus einem Ausdruck zuständig. 
Daher wird die Klasse mit einem Ausdruck initialisiert, der sogleich durch die 
Methode \textit{\_process\_expression(expression)} in einen AST umgewandelt wird. Der AST wird dann im Attribut \textit{root\_node} abgespeichert.

Das zweite Attribut, \textit{string\_representation}, enthält den AST im String-Format. Dies dient primär Debug-Zwecken.
Eine String-Repräsentation kann durch die erste Methode~(Z. \ref{astpreorder}) \textit{\_preorder(self, ast\_node)} erreicht werden,
indem man den erstellten AST in Preorder-Traversierung durchgeht und dabei die Schlüssel der Knoten an einen String anhängt.
Zusätzlich wird eine öffnende Klammer angehängt, wenn der Knoten zwei Kinder hat (also eine arithmetische Operation repräsentiert) und eine schließende Klammer,
wenn beide Kinder abgearbeitet wurden. Somit erhält man den eingegebenen Ausdruck exakt zurück, falls dieser korrekt eingegeben wurde.

\begin{lstlisting}[language=Python, escapechar=|, caption=Klasse \textit{AbstractSyntaxTree}, label={lst:ast}]
class AbstractSyntaxTree:

    # ... __init__ und __str__  weggelassen 

    def _preorder(self, ast_node): |\label{astpreorder}|
        # ...
        self._preorder(ast_node.left)
        self._preorder(ast_node.right)
        # ...

    def _process_expression(self, expression): |\label{astprocess}|
        root_ast_node = None
        stack = deque()
        word = ""

        for character in expression:
            if character == "(": |\label{ast0}|
                if not stack:
                    ast_node = AbstractSyntaxTreeNode()
                    stack.append(ast_node)
                    root_ast_node = ast_node
                else:
                    last_ast_node = stack[-1]
                    ast_node = AbstractSyntaxTreeNode()
                    # ...
                    stack.append(ast_node)
            elif character in ("/", "*", "+", "-"): |\label{ast1}|
                last_ast_node = stack[-1]
                last_ast_node.key = character
            elif (word == "<" or word == ">") and (character == "<" or character == ">"): |\label{ast2}|
                word += character
                last_ast_node = stack[-1]
                last_ast_node.key = word
                word = ""
            elif (character == " " or character == ')') and word != "": |\label{ast3}|
                last_ast_node = stack[-1]
                # ...
                if character == ")":
                    stack.pop()
                word = ""
            elif character == ")" and word == "": |\label{ast4}|
                stack.pop()
            elif character == " ": |\label{ast5}|
                pass
            else: |\label{ast6}|
                word += character
        return root_ast_node
\end{lstlisting}

Die zweite Methode~(Z. \ref{astprocess}) \textit{\_process\_expression(expression)} bekommt einen mathematischen Ausdruck als String übergeben und initialisiert drei lokale Variablen.
Die erste Variable \textit{root\_ast\_node} speichert den Wurzelknoten des Baumes, den die Methode später zurückgibt. Die zweite Variable \textit{stack} speichert an der 
Position \textbf{Top of Stack} (TOS) den aktuell zu bearbeitenden Knoten. Die dritte Variable \textit{word} ist ein leerer String, der als kurzzeitiger Zwischenspeicher verwendet wird,
wenn Variablen, Zahlen oder Operationen mehrere Zeichen lang sind.
Der String wird Zeichen für Zeichen von links nach rechts durchgegangen, wobei sieben Fälle unterschieden werden.

Fall eins~(Z. \ref{ast0}) ist gegeben, wenn eine öffnende Klammer eingelesen wird. Ein neuer Knoten wird erstellt, der sogleich auf einen Stack geschoben wird. Falls der Stack leer ist, wird
dieser Knoten als Wurzelknoten betrachtet und in \textit{root\_ast\_node} gespeichert.
Nach einer öffnenden Klammer wird normalerweise eine arithmetische Operation erwartet. 
Daher behandelt Fall zwei~(Z. \ref{ast1}) die vier Operationen ($+, *, -, /$), indem die jeweilige Operation dem Knoten, der gerade TOS ist,
als Schlüssel zugewiesen wird. Da Shift-Operationen ($<<, >>$) zwei Zeichen lang sind, werden sie in Fall drei~(Z. \ref{ast2}) behandelt.
Zu Fall drei kommt es, wenn in \textit{word} bereits eines der Shift-Zeichen ($<, >$) eingelesen wurde und nun das nächste eingelesen wird.
Das Zeichen wird \textit{word} hinzugefügt und dem TOS-Knoten als Schlüssel übergeben.

Im vierten Fall~(Z. \ref{ast3}) werden Variablen und Zahlen verarbeitet und eventuell auch Knoten vom Stack genommen.
Zur Trennung von Variablen und Zahlen wird ein Leerzeichen erwartet und ein nichtleeres \textit{word}.
Dabei werden nochmal vier Fälle unterschieden, je nach Zustand des aktuellen TOS-Knotens. Dementsprechend wird entweder ein neuer Knoten mit dem Inhalt von \textit{word} als Schlüssel
erzeugt und links oder rechts angehängt oder der aktuelle TOS-Knoten hat noch kein Schlüssel und bekommt entsprechend einen. 
Wenn jedoch das Zeichen ein schließende Klammer ist, wird \textit{word} geleert und der TOS-Knoten vom Stack genommen.
Fall fünf~(Z. \ref{ast4}) behandelt einen ähnlichen Fall. Hier ist \textit{word} jedoch leer. Der TOS-Knoten wird ebenfalls vom Stack genommen.
Fall sechs~(Z. \ref{ast5}) wird nur behandelt, wenn das aktuelle Zeichen ein Leerzeichen ist. Diese werden ignoriert.
Der letzte Fall~(Z. \ref{ast6}) ist gegeben, wenn kein anderer Fall zutrifft. Dabei wird das aktuelle Zeichen an \textit{word} angehangen.

\subsubsection{EGraph}

Aus dem Ausdruck wurde im letzten Abschnitt ein AST erzeugt. Nun soll dieser in einen EGraph umgewandelt werden. Ein EGraph besteht aus EClasses und ENodes.
Die Klasse \textit{ENode} repräsentiert später einen Knoten aus dem AST. Daher enthält sie auch einen Schlüssel sowie eine Liste mit Argumenten. Die Argumente werden 
EClass-IDs sein.

\begin{lstlisting}[language=Python, caption=Klasse \textit{ENode}]
class ENode:
    def __init__(self, key, arguments):
        self.key = key
        self.arguments = arguments
\end{lstlisting}

Die Klasse \textit{EClass} bekommt bei der Initialisierung eine individuelle \textbf{EClass-ID} sowie ein Set der Knoten, die in dieser EClass vorhanden sind. Außerdem
wird ein \textit{parent}-Set für die Elternklassen dieser EClass benötigt.

\begin{lstlisting}[language=Python, caption=Klasse \textit{EClass}]
import uuid

class EClass:
    def __init__(self):
        self.id = str(uuid.uuid4())
        self.nodes = set()
        self.parents = set()
\end{lstlisting}

Mit diesen beiden Hauptbestandteilen kann jetzt ein \textbf{EGraph} implementiert werden. 
Die zugehörigen Methoden werden in den Abschnitten \ref{subsub:egg} und \ref{subsub:colab} vorgestellt.

\begin{lstlisting}[language=Python, caption=Klasse \textit{EGraph}]
# ... 

class EGraph:
    def __init__(self):
        self.u = DisjointSet()
        self.m = {}
        self.h = {}
        self.pending = []
        self.version = 0
        self.str_repr = ""
        self.is_saturated = False

    # ...
\end{lstlisting} 


\begin{lstlisting}[language=Python, caption=Klasse \textit{RewriteRule}]
import AbstractSyntaxTree

class RewriteRule:
    def __init__(self, name, expr_lhs, expr_rhs):
        self.name = name
        self.expr_lhs = AbstractSyntaxTree.AbstractSyntaxTree(expr_lhs)
        self.expr_rhs = AbstractSyntaxTree.AbstractSyntaxTree(expr_rhs)

    def __str__(self):
        return f"[{self.name}: {self.expr_lhs} => {self.expr_rhs}]"
\end{lstlisting}



\subsubsection{egg-Implementierung}\label{subsub:egg}

\subsubsection{Colab-Implementierung}\label{subsub:colab}

\subsection{Services \& Server}

\subsection{Frontend}

\subsubsection{Design}
