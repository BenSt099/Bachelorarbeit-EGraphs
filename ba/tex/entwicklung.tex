\section{Entwicklung der Anwendung}\label{sec:entwicklung}

\subsection{E-Graphs}

Die Erstellung eines EGraphs erfolgt in mehreren Schritten. 

\subsubsection{AST}

Um die Erstellung eines EGraphs aus einem gegebenen mathematischen Ausdruck zu vereinfachen, wird zuerst ein äquivalenter \textit{Abstract Syntax Tree} (AST)
erstellt. Dafür werden die Klassen \textit{AbstractSyntaxTreeNode} und \textit{AbstractSyntaxTree} benötigt.
Die Klasse \textit{AbstractSyntaxTreeNode} stellt einen Knoten im AST dar und hat drei Attribute: einen linken Zeiger, einen rechten Zeiger und einen Schlüssel.
Der Schlüssel speichert die jeweilige Information ab, während die beiden Zeiger auf einen weiteren Teilbaum zeigen oder leer sind.

Die Klasse \textit{AbstractSyntaxTree} wird mit einem Asudruck initialisiert. Der Asdruck wird sogleich durch die Methode \textit{\_process\_expression(expression)} 
in einen AST umgewandelt und im Attribut \textit{root\_node} gespeichert, welches den obersten Knoten des AST beinhaltet. 
Die Methode \textit{\_process\_expression(expression)} funktioniert wie folgt:




Das zweite Attribut, \textit{string\_representation}, enthält den AST im String-Format. Dies dient primär Debug-Zwecken.
Eine String-Repräsentation kann erreicht werden, indem man den erstellten AST in Preorder-Traversierung durchgeht und dabei die Schlüssel der Knoten an einen String anhängt.
Zusätzlich wird eine öffnende Klammer angehängt, wenn der Knoten zwei Kinder hat (also eine arithmetische Operation repräsentiert) und eine schließende Klammer,
wenn beide Kinder abgearbeitet wurden. Somit erhält man den eingegebenen Ausdruck exakt zurück, falls dieser korrekt eingegeben wurde.

\subsubsection{egg-Implementierung}

\subsubsection{Colab-Implementierung}

\subsection{Services \& Server}

\subsection{Frontend}

\subsubsection{Design}
