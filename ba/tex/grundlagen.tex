\section{Grundlagen}\label{sec:grundlagen}


\subsection{Äquivalenzrelation}

Eine Äquivalenzrelation beschreibt eine Gleichwertigkeit zwischen Objekten, die folgende Eigenschaften erfüllen muss~\cite{Ehrig2001}:

\begin{itemize}
  \item \textbf{Reflexivität} Das Objekt ist gleichwertig zu sich selbst.
  \item \textbf{Symmetrie} Wenn Objekt $a$ zu Objekt $b$ gleichwertig ist, gilt das auch andersherum.
  \item \textbf{Transitivität} Wenn Objekt $a$ zu Objekt $b$ gleichwertig ist, und Objekt $b$ zu Objekt $c$, dann gilt auch, dass $a$ zu $c$ gleichwertig ist.
\end{itemize}

Als Beispiel dienen hier Schuhe verschiedener Marken. Zwei Schuhe $a, b$ sind bzgl. ihrer Größe äquivalent, wenn sie dieselbe Schuhgröße haben. 

\subsection{Äquivalenzklasse}

Eine Äquivalenzrelation teilt eine Menge von Objekten in Äquivalenzklassen klein, in denen die Objekte jeweils die drei oben genannten Eigenschaften erfüllen~\cite{Ehrig2001}.
Die Schuhe werden also bzgl. ihrer Größe in die Äquivalenzklassen \textit{Schuhgröße} eingeteilt.

\subsection{E-Graphs}


Es gibt verschiedene Implementierungsmöglichkeiten für E-Graphs. Die hier benutzte stammt aus dem Paper~\cite{2021-egg}.
Ein E-Graph ist ein Tuple $(\mathbf{U}, \mathbf{M}, \mathbf{H})$ mit folgenden Definitionen:

\begin{itemize}
  \item $\mathbf{U}$ Eine Union-Find-Datenstruktur, die eine Äquivalenzrelation über e-IDs abspeichert.
  \item $\mathbf{M}$ Eine (Hash)-map, die IDs von e-Klassen auf e-Klassen abbildet. 
  \item $\mathbf{H}$ Eine (Hashcons)-map, die e-Nodes auf IDs von e-Klassen abbildet.
\end{itemize}

\subsection{Equality Saturation}

