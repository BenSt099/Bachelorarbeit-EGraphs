\section{Reflexion der Arbeit}\label{sec:reflexion}

Im vorliegenden Kapitel wird eine kritische Reflexion der Arbeit vorgenommen. 
Der Fokus der Untersuchung liegt dabei auf der Arbeit an sich und auf der Anwendung.
Dementsprechend beschäftigt sich der erste Abschnitt mit der Frage, inwiefern das Ziel der Bachelorarbeit erreicht wurde.
Der zweite Abschnitt thematisiert die Qualität der entwickelten Software.

% In diesem Abschnitt wird eine kritische Reflexion zur Entwicklung der Anwendung vorgenommen. Dazu werden die zwei folgenden, konkreten Fragen beantwortet:

% \begin{enumerate}[nolistsep]
%     \item Warum wurde das Ziel der Bachelorarbeit erreicht?
%     \item Weist die produzierte Software eine hohe Qualität auf?
% \end{enumerate}

\subsection{Ziel der Bachelorarbeit}\label{sub:ziel}

Das Ziel wurde in der Einleitung bereits definiert. Dort heißt es: 
\glqq Das Ziel dieser Bachelorarbeit ist es, eine Anwendung für die Lehre zu entwickeln, die Studentinnen und Studenten Wissen zu den Themen \textit{E-Graphs} und \textit{Equality Saturation}
vermittelt. Dabei sollen sie die Möglichkeit haben, sich sowohl auf theoretischer als auch praktischer Ebene mit den Themen auseinander setzen zu können.
Die Grundlage der theoretischen Ebene bildet diese Arbeit, in der notwendige Hintergrundkenntnisse erarbeitet werden. Außerdem wird ein Einblick in die Implementierung gegeben. 
Die praktische Ebene besteht aus der Anwendung, mit deren Hilfe Schritt für Schritt aufgezeigt wird, wie \textit{E-Graphs} und \textit{Equality Saturation} funktionieren.
Für grö{\ss}tmöglichen Nutzen soll die Anwendung zudem plattformunabhängig sein und möglichst nur von \textit{Open-Source-Software} (OSS) Gebrauch machen [\ldots]\grqq 
(Kapitel~\ref{sec:einleitung}, Ziel der Arbeit).

Die Entwicklung einer Lernanwendung wurde in dieser Arbeit durchgeführt. Damit können sich Studentinnen und Studenten auf praktische Weise mit den Themen 
\textbf{E-Graphs} und \textbf{Equality Saturation} auseinander setzen. Kapitel~\ref{sec:grundlagen} dieser Arbeit kann genutzt werden, um sich mit den
theoretischen Grundlagen vertraut zu machen. 
Wie in Kapitel~\ref{sec:entscheidungen} bereits festgestellt wurde, macht die Anwendung nur von \textit{Open-Source-Software} Gebrauch. Außerdem ist
sie plattformunabhängig (siehe~\ref{softwarequalität}).
Als zusätzliches, objektives Maß für die Vollständigkeit kann das Exposé herangezogen werden.
Im Exposé wurden dabei folgende Ziele definiert, die die Anwendung erfüllen muss:

\begin{tcolorbox}[enhanced, frame hidden, borderline west = {1.5pt}{0pt}{gray-700},lower separated=false,fontupper=\sffamily]
\begin{enumerate}[topsep=0pt,itemsep=-0.5ex,partopsep=1ex,parsep=1ex]
    \item {\sffamily\itshape \textbf{GUI} Die Anwendung stellt eine Benutzeroberfläche in Form einer HTML-Website zur Verfügung, die lokal im Browser läuft.
        Alle Funktionen sowie der theoretische Teil können darüber abgefragt werden.}
    \item {\sffamily\itshape \textbf{E-Graph} Mit der Anwendung ist es möglich, durch Eingabe eines mathematischen Terms, einen E-Graph schrittweise zu erzeugen und in der GUI durch eine SVG-Datei anzeigen zu lassen.}
    \item {\sffamily\itshape \textbf{Mathematische Terme} Terme bestehen aus Variablen (Buchstaben wie $x, y, z$), Operationen ($+, *, -, /, <<, >>$) und Zahlen.}
    \item {\sffamily\itshape \textbf{Rewrite Rules} Der Benutzer hat die Möglichkeit, Rewrite Rules zu definieren und diese unabhängig voneinander auf den E-Graph anzuwenden. Die dabei entstehenden
        Veränderungen spiegelt der E-Graph wieder, d.h. die Darstellung wird aktualisiert.}
    \item {\sffamily\itshape \textbf{Default Rewrite Rules} Wenn der Benutzer die Anwendung startet, sollen bereits einige vordefinierte Rewrite Rules eingespeichert werden, um dem Benutzer Arbeit zu sparen und
        ihm gleichzeitig als Beispiel dienen.}
    \item {\sffamily\itshape \textbf{Historie} Mithilfe der GUI ist es möglich, zwischen verschiedenen Versionen des E-Graphs zu wechseln und diese anzuzeigen.}
    \item {\sffamily\itshape \textbf{Extraktion} Der finale, optimierte Term kann aus dem E-Graph extrahiert werden.}
    \item {\sffamily\itshape \textbf{Export} Dem Benutzer ist es möglich, den momentan angezeigten E-Graph in ein Dateiformat wie PDF oder SVG zu exportieren.}
    \item {\sffamily\itshape \textbf{Sessions} Aus Effizienzgründen kann der Benutzer seine jeweilige Session in einem für Menschen lesbaren Dateiformat abspeichern und bei einem erneuten Start der Anwendung wieder
        laden. Eine Session besteht dabei aus: dem mathematischen Term, den eingegebenen Rewrite Rules inklusive der momentan angewendeten und dem evtl. bereits extrahierten, optimalen Term.}
    \item{\sffamily\itshape \textbf{Hilfe} Durch die Benutzeroberfläche kann der Nutzer auf eine Dokumentation zurückgreifen, in der die Bedienung der Anwendung erklärt wird.}
\end{enumerate}\vspace{-2mm}
    
\begin{flushright}
    --- Kapitel Methodik, Exposé
\end{flushright}
\end{tcolorbox}

Somit kann abschließend resümiert werden, dass das vorher festgelegte Ziel der Bachelorarbeit erreicht wurde.























\subsection{Qualität der Software}\label{softwarequalität}

Bei der Bewertung von Software kann ihre Qualität als zuverlässiger Faktor miteinbezogen werden. Qualität kann unter verschiedenen Gesichtspunkten gemessen werden und wird
unterschiedlich definiert. Eine gute Orientierung bietet der \textit{ISO/IEC 25010:2011 Standard}~\footnote{\hspace{1.5mm}\url{https://www.innoq.com/de/blog/2021/10/quality-value-chain-evolution/}}. 
In diesem sind folgende Qualitätsmerkmale aufgelistet:

\noindent\textbf{Übertragbarkeit} Die Anwendung läuft problemlos auf allen gängigen Plattformen. Das liegt erstens daran, dass nur Open-Source-Software eingesetzt wurde,
wodurch Schwierigkeiten beim Beschaffen von Lizenzen vermieden werden können. 
Zweitens basieren vier der fünf Dependencies auf Python und die fünfte, \textit{Grahviz}, wird für alle gängigen Plattformen angeboten. Da Python plattformunabhängig ist, gilt dies auch
für die Dependencies.
Drittens läuft die Benutzeroberfläche im Browser, der als Dependency vorausgesetzt werden kann.
Somit kann auf eine Plattformunabhängigkeit der Anwendung geschlossen werden.

\noindent\textbf{Wartbarkeit} Die Architektur der Anwendung folgt einem logischen Aufbau mit klar definierten Abhängigkeiten, was das Hinzufügen neuer Funktionen erlaubt.
Der Code wurde mit \textit{Black}~\cite{black} formatiert und richtet sich nach \textit{PEP 8}~\cite{pep}. Jede Funktion oder Methode, die in Python oder JavaScript geschrieben wurde, 
ist mit einem Kommentar versehen, der Parameter, Rückgabewerte und Zweck erläutert. Weiterhin ist die Anwendung mit X Unit-Tests und zahlreichen händischen Simulationen abgedeckt, die eine breites Spektrum an 
möglichen Eingaben und Zuständen simulieren.

\noindent\textbf{Sicherheit} Für das Ausführen der Anwendung werden keine erweiterten Berechtigungen des Nutzers eingefordert. Bei der Verwendung der Software werden keine persönlichen Daten gespeichert 
oder weitergegeben und die Kommunikation findet ausschließlich im lokalen Netzwerk statt.
Zusätzlich wurde der gesamte Code der Anwendung auf \textit{GitHub} veröffentlicht, wodurch eine unabhängige Prüfung auf Sicherheitslücken und mögliche, illegale Aktivitäten der Software vorgenommen werden kann.

\noindent\textbf{Zuverlässigkeit} Die Anwendung basiert auf einem publizierten Paper~\cite{2021-egg}, indem auch die Vorgehensweise für die Bibliothek \textit{egg} erklärt wird.
Aufgrund der Tatsache, dass E-Graphs basierend auf dem Paper implementiert wurden, lässt sich mit Gewissheit sagen, dass zumindest in den grundlegenden Funktionen kein logischer Fehler sein kann.



\noindent\textbf{Funktionale Eignung} In Abschnitt~\ref{sub:ziel} wurde bereits dargelegt, dass die im Exposé festgehaltenen Anforderungen an die Software erreicht worden sind.
Die Anwendung bietet Studentinnen und Studenten die Möglichkeit, sich auf interaktive Weise mit E-Graphs und Equality Saturation auseinander zu setzen.





\noindent\textbf{Leistungseffizienz} Bei der Entwicklung der Anwendung wurde das Ziel der Leistungseffizienz anerkannt, aber nicht prioritisiert, da die Software in der Lehre eingesetzt werden sollte
und somit andere Anforderungen wie zum Beispiel die Benutzbarkeit erfüllen muss. Dementsprechend ist die Wahl der Programmiersprache Python nicht optimal. Auch die Übertragung eines E-Graphs ins DOT-Format 
geschieht mit Aufwand. Trotzdem ist die Software in der Lage, Darstellungen flüssig zu rendern und dem Benutzer eine verzögerungsfreie Interaktion mit E-Graphs zu ermöglichen.

\noindent\textbf{Kompatibilität} Die Software setzt auf bekannte Standards wie dem JSON- und dem DOT-Format und kann daher, mit eventuellen Modifikationen, auch mit anderen Systemen Daten austauschen.
Außerdem ist die Kommunikation zwischen Server und Weboberfläche so gestaltet, dass auch andere Systeme wie zum Beispiel eine REST-API oder eine andere Benutzeroberfläche an die Anwendung anbinden könnten.

\noindent\textbf{Benutzbarkeit} Die Weboberfläche setzt auf eine Website, die mithilfe des Frameworks \textit{Bootstrap}~\cite{bootstrap} übersichtlich gestaltet werden konnte.
Über die Website lässt sich die Dokumentation abrufen, die als Bedienungsanleitung fungiert und dem Benutzer mit Beispielen behilflich ist.
Bei der Visualisierung von E-Graphs wurde besonders großen Wert auf die Benutzerfreundlichkeit gelegt. Nicht nur lassen sich andere Strukturen der Website für eine bessere Übersicht 
einklappen, sondern die Darstellung eines E-Graphs lässt sich vergrößern, verkleinern und herumschieben. Es ist auch möglich die Grafik in einem neuen Tab zu öffnen, damit auch Benutzer
mit einem kleineren Bildschirm die Anwendung nutzen können.
Des Weiteren kann ein E-Graph durch das Debugging-Feature in jedem Stadium beobachtet werden. Hilfreiche Statusmeldungen erklären dem Benutzer die Abläufe.
Es ist auch möglich, eine Session bzw. die \textit{rewrite rules} in einer Datei zu speichern und später hochzuladen, sodass dem Benutzer Zeit erspart wird.
