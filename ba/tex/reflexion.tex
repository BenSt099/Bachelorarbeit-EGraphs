\section{Reflexion der Arbeit}\label{sec:reflexion}

In diesem Abschnitt wird eine kritische Reflexion zur Entwicklung der Anwendung vorgenommen. Dazu werden die zwei folgenden, konkreten Fragen beantwortet:

\begin{enumerate}[nolistsep]
    \item Warum wurde das Ziel der Bachelorarbeit erreicht?
    \item Weist die produzierte Software eine hohe Qualität auf?
\end{enumerate}

\subsection{Ziel der Bachelorarbeit}

Das Ziel der Bachelorarbeit wurde in der Einleitung bereits definiert. Dort heißt es: 
\glqq Das Ziel dieser Bachelorarbeit ist es, eine Anwendung für die Lehre zu entwickeln, die Studentinnen und Studenten Wissen zu den Themen \textit{E-Graphs} und \textit{Equality Saturation}
vermittelt. Dabei sollen sie die Möglichkeit haben, sich sowohl auf theoretischer als auch praktischer Ebene mit den Themen auseinander setzen zu können.
Die Grundlage der theoretischen Ebene bildet diese Arbeit, in der notwendige Hintergrundkenntnisse erarbeitet werden. Außerdem wird ein Einblick in die Implementierung gegeben. 
Die praktische Ebene besteht aus der Anwendung, mit deren Hilfe Schritt für Schritt aufgezeigt wird, wie \textit{E-Graphs} und \textit{Equality Saturation} funktionieren.
Für grö{\ss}tmöglichen Nutzen soll die Anwendung zudem plattformunabhängig sein und möglichst nur von \textit{Open-Source-Software} (OSS) Gebrauch machen [\ldots]\grqq 
(Kapitel~\ref{sec:einleitung}, Ziel der Arbeit).

Die Entwicklung einer Lernanwendung wurde in dieser Arbeit durchgeführt. Damit können sich Studentinnen und Studenten auf praktische Weise mit den Themen 
\textbf{E-Graphs} und \textbf{Equality Saturation} auseinander setzen. Kapitel~\ref{sec:grundlagen} dieser Arbeit kann genutzt werden, um sich mit den
theoretischen Grundlagen vertraut zu machen. 
Wie in Kapitel~\ref{sec:entscheidungenundprobleme} bereits festgestellt wurde, macht die Anwendung nur von \textit{Open-Source-Software} Gebrauch. Außerdem ist
sie plattformunabhängig (siehe~\ref{softwarequalität} Qualität der Software).




Das Exposé dient als Plan für eine Arbeit, dem sowohl Prüfer als auch Prüfling zugestimmt haben. Somit wird dem Prüfling eine Orientierung
und dem Prüfer ein objektives Maß für die Vollständigkeit der Arbeit gegeben. Im Exposé wurden dabei folgende Ziele definiert, die die Anwendung
erfüllen muss.

\begin{tcolorbox}[enhanced, frame hidden, borderline west = {1.5pt}{0pt}{gray-700},lower separated=false,fontupper=\sffamily]
\begin{enumerate}[topsep=0pt,itemsep=-0.5ex,partopsep=1ex,parsep=1ex]
    \item {\sffamily\itshape \textbf{GUI} Die Anwendung stellt eine Benutzeroberfläche in Form einer HTML-Website zur Verfügung, die lokal im Browser läuft.
        Alle Funktionen sowie der theoretische Teil können darüber abgefragt werden.}
    \item {\sffamily\itshape \textbf{E-Graph} Mit der Anwendung ist es möglich, durch Eingabe eines mathematischen Terms, einen E-Graph schrittweise zu erzeugen und in der GUI durch eine SVG-Datei anzeigen zu lassen.}
    \item {\sffamily\itshape \textbf{Mathematische Terme} Terme bestehen aus Variablen (Buchstaben wie $x, y, z$), Operationen ($+, *, -, /, <<, >>$) und Zahlen.}
    \item {\sffamily\itshape \textbf{Rewrite Rules} Der Benutzer hat die Möglichkeit, Rewrite Rules zu definieren und diese unabhängig voneinander auf den E-Graph anzuwenden. Die dabei entstehenden
        Veränderungen spiegelt der E-Graph wieder, d.h. die Darstellung wird aktualisiert.}
    \item {\sffamily\itshape \textbf{Default Rewrite Rules} Wenn der Benutzer die Anwendung startet, sollen bereits einige vordefinierte Rewrite Rules eingespeichert werden, um dem Benutzer Arbeit zu sparen und
        ihm gleichzeitig als Beispiel dienen.}
    \item {\sffamily\itshape \textbf{Historie} Mithilfe der GUI ist es möglich, zwischen verschiedenen Versionen des E-Graphs zu wechseln und diese anzuzeigen.}
    \item {\sffamily\itshape \textbf{Extraktion} Der finale, optimierte Term kann aus dem E-Graph extrahiert werden.}
    \item {\sffamily\itshape \textbf{Export} Dem Benutzer ist es möglich, den momentan angezeigten E-Graph in ein Dateiformat wie PDF oder SVG zu exportieren.}
    \item {\sffamily\itshape \textbf{Sessions} Aus Effizienzgründen kann der Benutzer seine jeweilige Session in einem für Menschen lesbaren Dateiformat abspeichern und bei einem erneuten Start der Anwendung wieder
        laden. Eine Session besteht dabei aus: dem mathematischen Term, den eingegebenen Rewrite Rules inklusive der momentan angewendeten und dem evtl. bereits extrahierten, optimalen Term.}
    \item{\sffamily\itshape \textbf{Hilfe} Durch die Benutzeroberfläche kann der Nutzer auf eine Dokumentation zurückgreifen, in der die Bedienung der Anwendung erklärt wird.}
\end{enumerate}\vspace{-2mm}
    
\begin{flushright}
    --- Kapitel Methodik, Exposé
\end{flushright}
\end{tcolorbox}

Somit kann abschließend resümiert werden, dass das vorher festgelegte Ziel der Bachelorarbeit erreicht wurde.


\subsection{Qualität der Software}\label{softwarequalität}

Ein Kernfaktor bei der Bewertung von Software ist ihre Qualität. Dabei bedeutet Qualität im Sinne der Softwareentwicklung: Wartbarkeit, Dokumentation, Testing, 
Portierbarkeit und Bedienbarkeit.  

\noindent\textbf{Wartbarkeit} Die Software hat eine klare Struktur, die es ermöglicht neue Strukturen ohne zu viel Aufwand zu implementieren.
Der Code wurde mit \textit{Black}~\footnote{\hspace{1.5mm}\url{https://black.readthedocs.io/en/stable/}} formatiert und richtet sich nach \textit{PEP 8}~\footnote{\hspace{1.5mm}\url{https://peps.python.org/pep-0008/}}.

\noindent\textbf{Dokumentation} Jede Funktion bzw. Methode im Python- oder JavaScript-Code ist mit einem Kommentar versehen, der Parameter (falls vorhanden), Rückgabewerte (falls vorhanden) und
Sinn erklären. Darüber hinaus enthalten alle Dateien einschließlich der Testdateien einen Kommentar, der den Inhalt der Datei erläutert.
Über die Benutzeroberfläche kann der Benutzer außerdem auf die Bedienungsanleitung der Website zugreifen. In dieser ist genau erklärt, wie die Anwendung gesteuert werden kann. 

\noindent\textbf{Testing} Die Anwendung ist durch Unit-Tests abgedeckt, die eine breite Palette an möglichen Eingaben und Zuständen simulieren.
Insgesamt wurden {\color{red} X} Tests geschrieben. Einfache \textit{Getter}-Methoden wurden dabei nicht getestet.

\noindent\textbf{Portierbarkeit} Die eingesetzten Komponenten laufen problemlos auf verschiedenen Plattformen inklusive der drei größten Betriebssysteme Windows, MacOS und Linux.
Durch den Einsatz von Python als Hauptprogrammiersprache, mit der auch der Server erstellt wurde, kann auf eine sehr gute Plattformunabhängigkeit geschlossen werden.
Die Benutzeroberfläche der Anwendung läuft als lokale Website im Browser des Benutzers, was ebenfalls keine Hürde darstellt. Zuletzt können vier der fünf Dependencies mit Python
heruntergeladen werden. Die fünfte Dependency, \textit{Grahviz}, ist auf allen gängigen Plattformen vertreten und bereits intensiv erprobt worden. 

\noindent\textbf{Bedienbarkeit} Die Benutzeroberfläche integriert Komponenten aus dem Framework \textit{Bootstrap}~\footnote{\hspace{1.5mm}\url{https://getbootstrap.com/}}, das auf 
Benutzerfreundlichkeit und Kompatibilität ausgelegt ist. Zudem kann über die Benutzeroberfläche die Bedienungsanleitung eingesehen werden.
