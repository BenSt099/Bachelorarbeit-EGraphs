%%%%%%%%%%%%%%%%%%%%%%%%%%%%%%%%%%%%%%%%%%%%%%%%%%%%%%%%%%%%%%%%%%%%%%%%%%%%%%%%
% Universität Düsseldorf                                                       %
% Lehrstuhl für Softwaretechnik und Programmiersprachen                        %
% Vorlage für Bachelor- und Masterarbeiten                                     %
% Erstellt: 2019-09-03                                                         %
%%%%%%%%%%%%%%%%%%%%%%%%%%%%%%%%%%%%%%%%%%%%%%%%%%%%%%%%%%%%%%%%%%%%%%%%%%%%%%%%
\documentclass{hhuthesis}


%%%%%%%%%%%%%%%%%%%%%%%%%%%%%%%%%%%%%%%%%%%%%%%%%%%%%%%%%%%%%%%%%%%%%%%%%%%%%%%%
%% Einstellungen zur Personalisierung                                         %%
%%                                                                            %%
%% Im Folgenden können Sie Ihre Arbeit personalisieren.                       %%
%%%%%%%%%%%%%%%%%%%%%%%%%%%%%%%%%%%%%%%%%%%%%%%%%%%%%%%%%%%%%%%%%%%%%%%%%%%%%%%%

%% Spracheinstellung
%% Kommentieren Sie die entsprechende Zeile ein bzw. aus.
%% Wir empfehlen jedem sich an einer englischen Arbeit zu versuchen.
% \usepackage[ngerman,english]{babel} % English
\usepackage[english,ngerman]{babel} % Deutsch

%% Ihr Name
\author{Ben Steinhauer}

%% Der Titel der Arbeit
\title{Entwicklung einer Anwendung zum Vermitteln von Lerninhalten über E-Graphs und Equality Saturation für die Lehre}

%% Der zu erreichende Abschluss, entweder Bachelor oder Master
\gratuationtype{Bachelor}
% \gratuationtype{Master}

%% Beginn- und Abgabedaten der Arbeit
\begindate{15. November 2024} % Beginn
\duedate{17. Februar 2025} % Abgabe

%% Erst- und Zweitgutachter
\firstexaminer{Dr.~John~Witulski}
\secondexaminer{Dr.~Carl~Bolz-Tereick}

%% Farb- oder Schwarzweißdruck
% Benutzen Sie das Kommando \blackwhiteprint,
% wenn sie in schwarzweiß drucken möchten.
% Im Farbdruck ist jede farbige Seite idR teurer.
% \blackwhiteprint  % Kommentarzeichen entfernen für Schwarzweißdruck

%%%%%%%%%%%%%%%%%%%%%%%%%%%%%%%%%%%%%%%%%%%%%%%%%%%%%%%%%%%%%%%%%%%%%%%%%%%%%%%%
%% (Ende) Einstellungen zur Personalisierung                                  %%
%%%%%%%%%%%%%%%%%%%%%%%%%%%%%%%%%%%%%%%%%%%%%%%%%%%%%%%%%%%%%%%%%%%%%%%%%%%%%%%%
%% LaTeX Packages in Nutzung                                                  %%
%%                                                                            %%
%% Im folgenden können Sie für die Niederschrift Ihrer Arbeit benötigte       %%
%% LaTeX-Pakete einbinden.                                                    %%
%% Diese Vorlage kommt bereits mit einigen nützlichen inkludierten Paketen.   %%
%%%%%%%%%%%%%%%%%%%%%%%%%%%%%%%%%%%%%%%%%%%%%%%%%%%%%%%%%%%%%%%%%%%%%%%%%%%%%%%%

%% Macht den \todo-Befehl verfügbar.
%% Hiermit können Sie Abschnitte annotieren,
%% welche weiterer Bearbeitung bedürfen.
\usepackage[textsize=scriptsize]{todonotes}

%% Zeige Zeilennummern in der Arbeit an.
%% Der \linenumbers Befehl muss hierzu aufgerufen werden.
%% Praktisch für Feedback Ihrer potentiellen Korrekturleser!
\usepackage{lineno}
% \linenumbers % <- Kommentar entfernen!


%% Häufig benutzte mathematische Packages.
\usepackage{amsfonts}
\usepackage{amsmath}
\usepackage{amssymb}

\usepackage{setspace}

\usepackage{listings} % Einbindung von Code
\usepackage{algorithmicx} % Angabe von Algorithmen in Pseudocode
\usepackage{siunitx} % \num Befehl zum einfacheren Formatieren von Zahlen.
\usepackage{enumitem} % Leichter konfigurierbare enumerate-Umgebungen.
\usepackage{subcaption} % Unterteilung von Figures in Subfigures.
\usepackage{hyperref} % Klickbare Referenzen (z.B. im Inhaltsverzeichnis)
\usepackage{url} % \url Kommando für Darstellung von Links
\usepackage{csquotes} % Improved quoting.
\usepackage{xspace} % Nicht terminierte Kommandos essen keinen Whitespace mehr.

%% Tabellen
\usepackage{tabularx} % tabularx Umgebung für mehr Kontrolle über Tabellen.
\usepackage{booktabs} % \toprule, \midrule, \bottomrule
\usepackage{multirow}
\usepackage{multicol}
\usepackage{longtable} % Große Tabellen gehen über mehrere Seiten.

%% Intelligenteres Referenzieren mittels \cref.
%% \languagename um dynamisch zwischen ngerman oder english zu wechseln.
\usepackage[\languagename,capitalize]{cleveref}

\usepackage{lcep}
\usepackage{float}
% \usepackage{fancyvrb}
\graphicspath{ {./fig/} }

\lstdefinestyle{codestyleBA}{
    backgroundcolor=\color{white},   
    commentstyle=\color{red-500},
    keywordstyle=\color{blue-500},
    numberstyle=\tiny\color{black},
    stringstyle=\color{green-500},
    basicstyle=\ttfamily\footnotesize,
    breakatwhitespace=false,         
    breaklines=true,                 
    captionpos=b,                    
    keepspaces=true,                 
    numbers=left,                    
    numbersep=5pt,                  
    showspaces=false,                
    showstringspaces=false,
    showtabs=false,                  
    tabsize=2,
    postbreak=\mbox{\textcolor{black}{$\hookrightarrow$}\space},
    literate={ä}{{\"a}}1 {ö}{{\"o}}1 {ü}{{\"u}}1 {Ä}{{\"{A}}}1 {Ö}{{\"{O}}}1 {Ü}{{\"{U}}}1 {ß}{{\ss}}1,
}

\lstdefinelanguage{JavaScript}{
  keywords={break, case, catch, continue, debugger, default, delete, do, else, false, finally, for, function, if, in, instanceof, new, null, return, switch, this, throw, true, try, typeof, var, void, while, with},
  morecomment=[l]{//},
  morecomment=[s]{/*}{*/},
  morestring=[b]',
  morestring=[b]",
  ndkeywords={class, export, boolean, throw, implements, import, this},
  keywordstyle=\color{blue-500}\bfseries,
  ndkeywordstyle=\color{darkgray}\bfseries,
  identifierstyle=\color{black},
  commentstyle=\color{red-500}\ttfamily,
  stringstyle=\color{green-500}\ttfamily,
  sensitive=true
}


\lstset{style=codestyleBA}

\usepackage{tcolorbox}
\tcbuselibrary{skins}

\usepackage{algpseudocodex}
\algrenewcommand\algorithmicfunction{\textbf{Funktion}}
\usepackage{algorithm}
\floatname{algorithm}{Algorithmus}
\renewcommand{\listalgorithmname}{Algorithmenverzeichnis}


\usepackage{wrapfig}

\raggedbottom

%%%%%%%%%%%%%%%%%%%%%%%%%%%%%%%%%%%%%%%%%%%%%%%%%%%%%%%%%%%%%%%%%%%%%%%%%%%%%%%%
%% (Ende) LaTeX Packages in Nutzung                                           %%
%%%%%%%%%%%%%%%%%%%%%%%%%%%%%%%%%%%%%%%%%%%%%%%%%%%%%%%%%%%%%%%%%%%%%%%%%%%%%%%%


\begin{document}
%% Set up title page, declaration of authorship, abstract, acknowledgements
\frontmatter
\makefrontmatter

%%%%%%%%%%%%%%%%%%%%%%%%%%%%%%%%%%%%%%%%%%%%%%%%%%%%%%%%%%%%%%%%%%%%%%%%%%%%%%%%
%% Danksagungen                                                               %%
%%%%%%%%%%%%%%%%%%%%%%%%%%%%%%%%%%%%%%%%%%%%%%%%%%%%%%%%%%%%%%%%%%%%%%%%%%%%%%%%
\begin{acknowledgements}
Das Schreiben einer Bachelorarbeit ist mit Anstrengung und Hingabe verbunden.
Vor und auch während des Schreibprozesses wurde ich von einigen Menschen unterstützt
und begleitet, denen ich hiermit meine Dankbarkeit aussprechen möchte.

Ein herzliches Dankeschön geht an Dr.~John~Witulski, der das Thema dieser Bachelorarbeit
vorschlug und sich bereit erklärte, mich zu betreuen. Neben dem wöchentlichen 
Feedback gab er mir viele hilfreiche Ratschläge und Anregungen.
Zusätzlich gilt mein Dank Dr.~Carl~Bolz-Tereick, der sich bereit erklärt hat, 
als Zweitgutachter zu fungieren. Zudem bedanke ich mich für das wertvolle Feedback 
zur Anwendung.

Ich danke außerdem meiner Familie für die Unterstützung während dieser Zeit.
\end{acknowledgements}
%%%%%%%%%%%%%%%%%%%%%%%%%%%%%%%%%%%%%%%%%%%%%%%%%%%%%%%%%%%%%%%%%%%%%%%%%%%%%%%%
%% (Ende) Danksagungen                                                        %%
%%%%%%%%%%%%%%%%%%%%%%%%%%%%%%%%%%%%%%%%%%%%%%%%%%%%%%%%%%%%%%%%%%%%%%%%%%%%%%%%


\tableofcontents

\mainmatter

%%%%%%%%%%%%%%%%%%%%%%%%%%%%%%%%%%%%%%%%%%%%%%%%%%%%%%%%%%%%%%%%%%%%%%%%%%%%%%%%
%% Der Inhalt der Arbeit                                                      %%
%%                                                                            %%
%% Hier können Sie die schriftliche Ausarbeitung ihrer Arbeit                 %%
%% niederschreiben. Der Übersicht halber bietet sich jedoch an, dies in einer %%
%% oder mehreren separaten Dateien zu tun, welche mittels \input eingebunden  %%
%% werden --- wie auch in der Vorlage geschieht.                              %%
%%%%%%%%%%%%%%%%%%%%%%%%%%%%%%%%%%%%%%%%%%%%%%%%%%%%%%%%%%%%%%%%%%%%%%%%%%%%%%%%

\section{Einleitung}

\subsection{Motivation}

%~\cite{phaseorder-2009}

\subsection{Ziel der Arbeit}

Das Ziel dieser Bachelorarbeit ist es, ein sinnvolles Werkzeug für die Lehre zu erstellen,
um Studentinnen und Studenten die Themen \textbf{E-Graphs} und \textbf{Equality Saturation} näher zu bringen.
Dabei sollen sie die Möglichkeit haben, sich sowohl auf theoretischer als auch praktischer Ebene mit E-Graphs auseinander setzen zu können.
Die theoretische Ebene soll den Studenten die notwendigen Hintergrundkenntnisse vermitteln sowie einen Enblick in die Implementierung geben.
Die praktische Ebene soll Schritt für Schritt aufzeigen, wie der \textbf{E-Graph} aufgebaut wird, und wie an diesem \textbf{Equality Saturation} durchgeführt werden kann.
Für grö{\ss}tmöglichen Nutzen soll die Anwendung plattformunabhängig sein und möglichst nur von \textit{Open-Source-Software} (OSS) Gebrauch machen.
Damit wird das Problem der unterschiedlichen Betriebssysteme der Studenten umgangen und zeitgleich die Hürden für Erweiterungen gesenkt.

\subsection{Aufbau der Arbeit}

\noindent {\itshape Kapitel \ref{sec:grundlagen} - Grundlagen}

Dieses Kapitel zeichnet den Beginn der Arbeit, indem notwendige (mathematische) Kenntnisse über \textbf{E-Graphs} und \textbf{Equality Saturation} vermittelt werden.

\noindent {\itshape Kapitel \ref{sec:entwicklung} - Entwicklung}

\noindent {\itshape Kapitel \ref{sec:entwicklung} - Architektur}

\noindent {\itshape Kapitel \ref{sec:entscheidungenundprobleme} - Entscheidungen und Probleme}

\noindent {\itshape Kapitel \ref{sec:zusammenfassung} - Zusammenfassung}

\subsection{Verwandte Arbeiten}

\noindent\textbf{egg} Das Akronym \textit{egg} steht für \textit{e-graphs good} und ist eine in der Programmiersprache \textit{Rust} geschriebene Bibliothek.
Die Bibliothek implementiert die Datenstruktur E-Graph und stellt Funktionen für die Interaktion mit dieser zur Verfügung.
Neben der Erzeugung von E-Graphs kann auch \textit{Equality Saturation} auf diesen durchgeführt werden.
Das gesamte Projekt wurde in einem Paper vorgestellt, auf welches sich diese Arbeit stützt~\cite{2021-egg}.


\section{Grundlagen}\label{sec:grundlagen}

% Es beginnt mit einem naiven Ansatz zur Optimierung von Ausdrücken und leitet zu \textit{E-Graphs} und \textit{Equality Saturation} über.

Dieses Kapitel leitet mit einem naiven Ansatz zur Optimierung ein und diskutiert die daraus entstehenden Nachteile.
Danach wir schrittweise zu \textit{E-Graphs}
übergeleitet und schließlich der Prozess \textit{Equality Saturation} erklärt.
Die folgenden beiden Abschnitte~\ref{subsec:naiv} und~\ref{subsec:egraphs} stützen sich auf~\cite{cole}. Viele Beispiele wurden aus dieser Quelle übernommen.

\subsection{Naives Optimieren}\label{subsec:naiv}

Für das naive Optimieren von Ausdrücken wird zuerst das Konzept der \textit{rewrite rules} benötigt. Gegeben sei ein beliebiger mathematischer Ausdruck, z.B.
$a * 1$. Um eine Umformung dieses Ausdrucks durchführen zu können, kann eine \textit{rewrite rule} auf diesen Ausdruck angewendet werden.
Als Beispiel sei hier die folgende Regel gegeben: $[\mathbf{1}]: x * 1 \Leftrightarrow x$. Die \textit{rewrite rule} besteht aus einem linken und rechten Teil. Der linke Teil
beschreibt ein \textbf{Muster}, das dem Ausdruck entsprechen muss. Das wird als \textit{Matching} bezeichnet. Dabei spielt der Name der Variablen, in diesem Falle $x$, keine Rolle.
Der rechte Teil vermittelt, in welche neue Form dieses Muster gebracht werden soll. Wendet man also Regel $[1]$ an,
ergibt sich: $a * 1  \overset{[2]}{\rightarrow} a$. 
Eine Regel wie $\frac{x}{x} \Leftrightarrow 1$ \textit{matcht} nicht und würde damit auch nicht angewendet werden können.
Mit diesen Bausteinen kann bereits ein naiver Algorithmus entworfen werden, der in Ansatz~\ref{alg:ausdruck1} skizziert wurde. 

% Gegeben sei ein mathematischer Ausdruck, zum Beispiel: $a * 1$. Um diesen Ausdruck umzuformen, kann eine \textit{rewrite rule} eingesetzt werden.
% Eine \textit{rewrite rule} beschreibt die Umformung eines Ausdrucks in einen anderen. Als Beispiel sei hier folgende Regel gegeben: $x * 1 \Leftrightarrow x \;\; (1)$.
% Angewendet auf den obigen Ausdruck würde sich folgendes ergeben: $a * 1  \overset{(1)}{\Leftrightarrow} a$.

% Die Anwendung dieser \textit{rewrite rule} ist nur möglich, weil die Regel \textit{matcht}. Das bedeutet, dass der Ausdruck und die linke Seite der Regel identisch sind 
% (abgesehen vom Namen der Variablen). Die folgende \textit{rewrite rule} würde nicht \textit{matchen} und damit auch nicht angewendet werden können: $\frac{x}{x} \Leftrightarrow 1$.
% Mit diesen Bausteinen lässt sich bereits ein naiver Algorithmus anfertigen, der in Abbildung~\ref{alg:ausdruck1} dargestellt ist.

\begin{algorithm}[H]
  \caption{Naiver Algorithmus zur Optimierung von Ausdrücken}\label{alg:ausdruck1}
  \begin{algorithmic}
    \Function{optimize\_expression}{expression}
    \State rules $\gets$ [\ldots]
    
    \While{old\_expression $\neq$ expression}
      \State old\_expression $\gets$ expression

      \For{rule in rules}
        \If{match(expression, rule)}
        \State apply(expression, rule)
        \EndIf
      \EndFor
    \EndWhile

    \State \Return expression
    \EndFunction
  \end{algorithmic}
\end{algorithm}

Der Algorithmus durchläuft solange die while-Schleife, bis keine Änderungen am Ausdruck mehr auftreten.
Dabei werden alle \textit{rewrite rules} auf den Ausdruck angewendet, die \textit{matchen}.
Diese Lösung zeichnet sich durch ihre Einfachheit aus, hat aber auch ihre Schwächen. So läuft der Algorithmus
bereits bei einfachen Beispielen in lokale Optima.
Als Beispiel sei hier der Ausdruck $(a * 2) / 2$ gegeben, der, durch Anwendung folgender Regeln, zum Ausdruck $a$ umgeformt werden kann: \\ \\
$[\mathbf{2}]: (x * y) / z \Leftrightarrow x * (y / z)$, \\
$[\mathbf{3}]:x / x \Leftrightarrow 1$, \\
$[\mathbf{4}]:x * 1 \Leftrightarrow x$ 

$(a * 2) / 2 \overset{[2]}{\Leftrightarrow} a * (2 / 2) \overset{[3]}{\Leftrightarrow} a * 1 \overset{[4]}{\Leftrightarrow} a$.

Wird statt Regel $[2]$ eine weitere Regel $[5]$ angewendet, zum Beispiel $[\mathbf{5}]: x * 2 \Leftrightarrow x << 1$, könnte der Algorithmus auch mit folgendem Ausdruck enden: $(a << 1) / 2$.
Das Ergebnis hängt also davon ab, in welcher Reihenfolge die Regeln auf den Ausdruck angewendet werden. 
Dieses Problem ist allgemein als das \textit{Phase Ordering Problem} bekannt~\cite{phaseorder-2009}.
Selbst mit der Anwendung von Heuristiken würde dieses Problem bestehen bleiben, da auch sie dazu tendieren, kurzsichtige Entscheidungen zu treffen~\cite{phaseorder-2009}.

\noindent Trotz der genannten Schwierigkeiten lässt sich das \textit{Phase Ordering Problem} umgehen. Der vorherige Algorithmus~\ref{alg:ausdruck1} wird dazu um eine Liste erweitert.
Indem alle erzeugbaren Ausdrücke in einer Liste gespeichert werden (Duplikate sind hier ausgeschlossen), kann der Algorithmus in kein lokales Optimum laufen.
Diese Idee setzt Algorithmus~\ref{alg:ausdruck2} um.

% \noindent Um diese Probleme zu umgehen, kann der vorherige Algorithmus um eine Liste erweitert werden 
% Diese Liste speichert alle erzeugten Ausdrücke ohne Duplikate.

% Nun wird versucht, alle möglichen Regeln auf die Ausdrücke in der Liste anzuwenden. 
% Dies setzt der nächste Algorithmus in Abbildung~\ref{alg:ausdruck2} um.

\begin{algorithm}[H]
  \caption{Verbesserter, naiver Algorithmus zur Optimierung von Ausdrücken}\label{alg:ausdruck2}
  \begin{algorithmic}
    \Function{optimize\_expression}{expression}
    \State rules $\gets$ [\ldots]
    \State expression\_list $\gets$ list(expression)
    
    \While{length(expression\_list) $\neq$ old\_length}
      \State old\_length $\gets$ length(expression\_list)

      \For{expr in expression\_list}
        \For{rule in rules}
          \If{match(expr, rule)}
          \State new\_expr = apply(expr, rule)
          \State expression\_list.add(new\_expr)
          \EndIf
        \EndFor
      \EndFor
    \EndWhile

    \State \Return expression\_list.get\_best()
    \EndFunction
  \end{algorithmic}
\end{algorithm}

Algorithmus~\ref{alg:ausdruck2} durchläuft ebenfalls eine while-Schleife bis sich keine Änderungen mehr ergeben. In jedem Schleifendurchlauf wird wiederrum die Liste durchlaufen.
Für jeden Ausdruck der Liste wird festgestellt, ob eine Regel anwendbar ist, und in diesem Fall angewendet. Der Trick ist hierbei den neu erzeugten Ausdruck wieder zur Liste 
hinzuzufügen. Somit kann am Ende aus einer Liste mit allen möglichen Konfigurationen des Ursprungsausdrucks ein optimaler Ausdruck extrahiert werden.
Durch das Umgehen des \textit{Phase Ordering Problems} treten indessen zwei neue Probleme auf.
Der Speicherbedarf des Algorithmus (also die Größe der Liste) lässt sich durch die Konstruktion einer trivialen Situation exponentiell erhöhen.
Dadurch würde auch die Extraktion des optimalen Ausdrucks exponentiell lange dauern.

Sei der Ausdruck $(a * 2) / 2$ mit den entstehenden Regeln aus dem vorherigen Beispiel als Situation gegeben. Durch die Anwendung einer einfachen \textit{rewrite rule} wie 
$[\mathbf{6}]: a \Leftrightarrow b$, würde sich die Größe der Liste bereits verdoppeln.
Eine effizientere Lösung wäre der Einsatz von \textit{E-Graphs}.

% Am Ende der Ausführung steht eine komplette Liste mit allen möglichen Konfigurationen des Ursprungsausdrucks zur Verfügung, aus der man den optimalen Ausdruck extrahieren kann.
% Zwar konnte hier das \textit{Phase Ordering Problem} umgangen werden, jedoch hat der Algorithmus eine exponentielle Laufzeit.
% \textit{E-Graphs} sind Datenstrukturen, die diese Probleme lösen können.

\subsection{E-Graphs}\label{subsec:egraphs}

Um ein grundlegendes Verständnis von \textit{E-Graphs} zu erlangen und den Unterschied zu den vorherigen Ansätzen besser nachvollziehen zu können,
werden zunächst einige wichtige Konzepte im Zusammenhang mit \textit{E-Graphs} vorgestellt. 

\subsubsection{Äquivalenzrelation}

Eine Äquivalenzrelation beschreibt eine Gleichwertigkeit zwischen Objekten, die folgende Eigenschaften erfüllen muss~\cite{Ehrig2001}:

\begin{itemize}
  \item \textbf{Reflexivität} Das Objekt ist gleichwertig zu sich selbst.
  \item \textbf{Symmetrie} Wenn Objekt $a$ zu Objekt $b$ gleichwertig ist, gilt das auch andersherum.
  \item \textbf{Transitivität} Wenn Objekt $a$ zu Objekt $b$ gleichwertig ist, und Objekt $b$ zu Objekt $c$, dann gilt auch, dass $a$ zu $c$ gleichwertig ist.
\end{itemize}

Als Beispiel dienen hier Schuhe verschiedener Marken. Zwei Schuhe $a, b$ sind bzgl. ihrer Größe äquivalent, wenn sie dieselbe Schuhgröße haben. 

\subsubsection{Äquivalenzklasse}

Eine Äquivalenzrelation teilt eine Menge von Objekten in Äquivalenzklassen klein, in denen die Objekte jeweils die drei oben genannten Eigenschaften erfüllen~\cite{Ehrig2001}.
Die Schuhe werden also bzgl. ihrer Größe in die Äquivalenzklassen \textit{Schuhgröße} eingeteilt.

\subsubsection{Zusammenhang zu E-Graphs}

Die vorherigen Ansätze basierten auf einer Liste von konkreten Ausdrücken. Dadurch musste für jede kleine Änderung, wie zum Beispiel $[\mathbf{6}]: a \Leftrightarrow b$ (siehe zweiter Ansatz),
ein neuer Ausdruck eingefügt werden. Der neue Ansatz fußt auf zwei Ideen: \textbf{Sharing} und \textbf{Klassen}.
Sharing bedeutet, dass Elemente, die in verschiedenen Ausdrücken vorkommen, also zum Beispiel eine Zahl oder eine Variable, nicht mehr im konkreten Ausdruck abgespeichert werden.
Stattdessen wird dort ein Platzhalter (Pointer) eingefügt, der auf das Element zeigt. Abbildung~\ref{fig:sharing} zeigt eine Veranschaulichung des Prinzips.
Zusätzlich werden Klassen eingefügt. Die Platzhalter zeigen jetzt nicht mehr auf einzelne Elemente, sondern auf eine Klasse, die Elemente enthält. Wird also Regel $[6]$ angewendet,
verdoppelt sich die Liste nicht, denn die Klasse mit Element $a$ wird nur um ein weiteres Element $b$ erweitert (siehe Abbildung~\ref{fig:classes}). 

{\captionsetup[figure]{oneside,margin={0.4cm,0cm}}
\begin{minipage}[c]{0.5\linewidth}
    \begin{figure}[H]
        \centering
        \includegraphics[scale=1.6]{../fig/sharing.pdf}
        \caption{Sharing bei einer Liste von Ausdrücken}
        \label{fig:sharing}
    \end{figure}
    \end{minipage}
    \begin{minipage}[c]{0.5\linewidth}
    \begin{figure}[H]
        \centering
        \includegraphics[scale=1.6]{../fig/classes.pdf}
        \caption{Kombination aus Sharing und Klassen bei einer Liste von Ausdrücken}
        \label{fig:classes}
    \end{figure}
\end{minipage}}

Auf den beiden vorgestellten Prinzipien fußen auch \textit{E-Graphs}. \textit{E-Graphs} bestehen aus zwei Komponenten: \textit{E-Classes} und \textit{E-Nodes}. 
Abbildung~\ref{fig:egraphexp} zeigt einen beispielhaften E-Graph. \textit{E-Nodes} sind dabei weiß und \textit{E-Classes} beige eingefärbt.

\begin{figure}[H]
  \centering
  \includegraphics[scale=0.5]{../fig/egraph_exp.png}
  \caption{Beispiel eines E-Graphs, der die Ausdrücke $(a * 2) / 2$ und $(b * 2) / 2$ enthält}
  \label{fig:egraphexp}
\end{figure}

\textit{E-Nodes} sind Knoten (ähnlich zum Element $a$ in Abbildung~\ref{fig:sharing}), die arithmetische Operationen, Zahlen oder Variablen enthalten können.
% Dabei ist zu beachten, dass bei arithmetischen Operationen kein Sharing betrieben wird, da diese auf unterschiedliche  zeigen können.  
Einer \textit{E-Node} werden als Argumente \textit{E-Classes} übergeben. In Beispiel~\ref{fig:egraphexp} wird der \textit{E-Node} $\mathbf{*}$ die 
\textit{E-Class} mit den beiden \textit{E-Nodes} $\mathbf{a}$ und $\mathbf{b}$ und die \textit{E-Class} mit der \textit{E-Node} $\mathbf{2}$ übergeben. 
\textit{E-Classes} stellen also Äquivalenzklassen dar, die äquivalente \textit{E-Nodes} enthalten.

% \textit{E-Classes} sind Äquivalenzklassen mit \textit{E-Nodes}, die äquivalent zueinander sind. 
% Im Beispiel~\ref{fig:egraphexp} sind $a$ und $b$ äquivalent zueinander.
% Als Argumente von \textit{E-Nodes} werden \textit{E-Classes} übergeben.
% Im Beispiel~\ref{fig:egraphexp} wird dem \textit{E-Node} $*$ die \textit{E-Class} mit den beiden \textit{E-Nodes} $a$ und $b$ und die \textit{E-Class} mit dem \textit{E-Node} $2$ übergeben. 
% Es wird also ausgenutzt, dass $a$ und $b$ in der gleichen \textit{E-Class} sind. 

\subsubsection{Implementierung von E-Graphs}

Es gibt verschiedene Ansätze für die Implementierung von \textit{E-Graphs}. Der Ansatz, auf dem diese Arbeit beruht, ist eine Zusammenstellung zweier Ansätze.
Der erste stammt aus dem Paper~\cite{2021-egg}, in dem auch \textit{egg} vorgestellt wurde (siehe verwandte Arbeiten~\ref{sub:verwandtearbeiten}). Der zweite 
ist einen Google Colab Notebook entnommen~\cite{devito}.
Ein \textbf{E-Graph} wird nach~\cite{2021-egg} als Tuple $(\mathbf{U}, \mathbf{M}, \mathbf{H})$ mit folgenden Eigenschaften definiert:

\begin{itemize}
  \item $\mathbf{U}$ Eine Union-Find-Datenstruktur, die eine Äquivalenzrelation über IDs abspeichert.
  \item $\mathbf{M}$ Eine (Hash)-map, die IDs von E-Classes auf E-Classes abbildet. 
  \item $\mathbf{H}$ Eine (Hashcons)-map, die E-Nodes auf IDs von E-Classes abbildet.
\end{itemize}

Der \textit{E-Graph} als Datenstruktur für Beispiel~\ref{fig:egraphexp} würde also wie folgt aussehen:

\begin{itemize}
  \item $\mathbf{U}$: \{ID1\}, \{ID2\}, \{ID3\}, \{ID4, ID5\} 
  \item $\mathbf{M}$: $ID1 \rightarrow EClass(\ldots)$, $ID2 \rightarrow EClass(\ldots)$, $ID3 \rightarrow EClass(\ldots)$, \ldots 
  \item $\mathbf{H}$: $/ \rightarrow ID1$, $* \rightarrow ID2$, $2 \rightarrow ID3$, $a \rightarrow ID4$, $b \rightarrow ID5$
\end{itemize}

\subsection{Equality Saturation}

Nachdem \textit{E-Graphs} eingeführt wurden stellt sich jetzt die Frage, wie damit ein optimaler Ausdruck gefunden werden kann.
Hierzu kann die Methode der \textit{Equality Saturation} eingesetzt werden.
\textit{Equality Saturation} beschreibt eine aus zwei Phasen bestehende, iterative Methode.
In der ersten Phase werden \textit{rewrite rules} auf den \textit{E-Graph} angewendet, bis sich keine Änderungen an diesem mehr ergeben.
Der E-Graph gilt nach dieser Phase als \textit{saturiert}.
In der zweiten Phase kann mithilfe einer Kostenfunktion der optimale Ausdruck aus dem E-Graph extrahiert werden.
Algorithmus~\ref{alg:eqsat} zeigt den Prozess der \textit{Equality Saturation}.

\begin{algorithm}[H]
  \caption{Traditioneller Equality Saturation Workflow nach~\cite{2021-egg}}\label{alg:eqsat}
  \begin{algorithmic}
    \Function{eqsat}{expr, rewrites}
    \State egraph $\gets$ initial\_egraph(expr)
    
    \While{$\mathbf{not}$ egraph.is\_saturated\_or\_timeout()}
      
      \For{rw in rewrites}
        \For{(subst, eclass) $\mathbf{in}$ egraph.ematch(rw.lhs)}
          \State  eclass2 $\gets$ egraph.add(rw.rhs.subst(subst))
          \State egraph.merge(eclass, eclass2)
        \EndFor 
      \EndFor
    \EndWhile

    \State \Return egraph.extract\_best()
    \EndFunction
  \end{algorithmic}
\end{algorithm}

Mit Algorithmus~\ref{alg:eqsat} gibt es nun eine effiziente Lösung, mit der das \textit{Phase Ordering Problem} umgangen und Speicherbedarf gesenkt werden kann.
Doch auch diese Methode kommt mit einem Haken. Der Extraktionsschritt ist ein NP-schweres Problem~\cite{phaseorder-2009}. In der Praxis ist dennoch möglich ein optimales
Ergebnis in kurzer Zeit zu erhalten, wie die Bibliothek \textit{egg} demonstriert.  


\section{Technologiestack}\label{sec:entscheidungen}

\subsection{Python}

Die Auswahl einer Programmiersprache zum Durchführen eines Projektes entscheidet über dessen Potential und Einschränkungen.
Zudem bietet jede Programmiersprache unterschiedliche Herangehensweisen und Werkzeuge an, um Probleme lösen zu können.
Auf Grundlage dieser Gegebenheiten wurde die Programmiersprache \textit{Python} in der Version \textit{3.12} gewählt.
Dafür gab es folgende Gründe:

\begin{itemize}
    \item Obwohl die Programmiersprache \textit{Python} einen engen Bezug zu Data Science, Machine Learning und ähnlichen Themen hat, gilt sie dennoch als Allzweck-Sprache,
    die in vielen Situation Anwendung finden kann. Sie ist außerdem bekannt und wird auf allen gängigen Plattformen genutzt, sodass die Platformunabhängigkeit garantiert ist. 
    Die Standardbibliothek bietet viele Funktionen und es gibt zahlreiche Bibliotheken, die eine Reihe von verschiedenen Themengebieten behandeln.
    \item \textit{Python} erlaubt durch seine unkomplizierte Speicherverwaltung und eingebaute Garbage Collection das schnelle Konstruieren und Testen von Prototypen.
    Damit entfällt weniger Zeit auf Implementierungsdetails wie Speicherverwaltung und mehr Zeit auf das übergeordnete Ziel.
\end{itemize}


% Die Wahl einer Programmiersprache entscheidet über die Möglichkeiten sowie Einschränkungen in einem Projekt.
% Zudem bietet jede Programmiersprache unterschiedliche Herangehensweisen an Probleme an und wurde teilweise für einen spezifischen Zweck entworfen.

% Für dieses Projekt wurde die Allzweck-Programmiersprache \textit{Python} in der Version \textit{3.12} gewählt.
% Dafür gibt es zwei Hauptgründe. 

% \noindent Erstens gilt die Programmiersprache \textit{Python} allgemein als sehr einfach. 
% Das erlaubt einerseits das schnelle Konstruieren und Testen von Prototypen. 
% Andererseits erlaubt es eine Fokusierung auf das übergeordnete Ziel, eine Lernanwendung zu entwickeln. Dabei müssen Implementierungsdetails, wie zum Beispiel
% die Garbage Collection, nicht beachtet werden.

% Zweitens ist \textit{Python} eine sehr bekannte und weit verbreitete Programmiersprache. 
% Die Standardbibliothek bietet großen Umfang und es gibt zahlreiche Bibliotheken, die eine Reihe von verschiedenen Themengebieten behandeln.
% Außerdem wurde \textit{Python} auf allen gängigen Betriebssystemen bereits getestet, wodurch die Platformunabhängigkeit garantiert wird.

\subsection{Graphviz}

Um das Ziel der Anwendung zu erreichen, muss dem Benutzer die Datenstruktur E-Graph visuell angezeigt werden können. 
Durch die graphenähnliche Struktur der Datenstruktur, die in drei verschiedenen Datenstrukturen festgehalten ist, wurde die Software 
\textit{Graphviz}~\footnote{\hspace{1.5mm}\url{https://graphviz.org/}} ausgewählt. Folgende Gründe sprachen waren dabei für die Entscheidung ausschlaggebend:

% Jeder EGraph besteht aus drei Datenstrukturen. Um diese visuell darstellen zu können, wird eine Software benötigt, die Graphen erzeugen kann.
% Dazu wurde die Software \textit{Graphviz}~\footnote{\hspace{1.5mm}\url{https://graphviz.org/}} ausgewählt. Folgende Gründe waren dabei ausschlaggebend.

\begin{itemize}
    \item Die Software \textit{Graphviz} ist auf allen gängigen Plattformen verfügbar, weitreichend getestet und Open Source. Außerdem findet sie in einer Vielzahl von Gebieten Anwendung.
    \item Sie bietet eine allgemeine, abstrakte Grammatik (die sogenannte \textit{DOT}-Sprache) zur Beschreibung von Graphen an. Die \textit{DOT}-Sprache ist, ähnlich wie andere Sprachformate wie
    zum Beispiel JSON oder YAML, zu einem Standard geworden. Dementsprechend gibt es auch andere Werkzeuge und Software, die dieses Sprachformat verarbeiten können.
    \item Es gibt viele Pakete, die auf \textit{Graphviz} aufbauen, darunter das Python-Paket \textit{graphviz}~\footnote{\hspace{1.5mm}\url{https://pypi.org/project/graphviz/}}, was nativen Anschluss an die Software ermöglicht. 
    Daher kann auf das Interagieren mit \textit{Graphviz} über ein CLI verzichtet werden.
    \item \textit{Graphviz} kennt viele gängige Ausgabeformate, wie SVG, PDF oder PNG, in die ein im \textit{DOT}-Format beschriebener Graph exportiert werden kann.
    \item Ein E-Graph muss im Browser des Benutzers angezeigt werden können. Hierzu existiert die JavaScript-Bibliothek \textit{D3}~\footnote{\hspace{1.5mm}\url{https://d3js.org/}}, die, im Zusammenspiel
    mit einer entsprechenden Bibliothek wie \textit{d3-graphviz}~\footnote{\hspace{1.5mm}\url{https://github.com/magjac/d3-graphviz}}, in der Lage ist, ein E-Graph in \textit{DOT}-Format entgegen zu nehmen,
    und ihn als SVG zu rendern. Das \textit{SVG}-Format eignet sich dabei sehr gut für die Darstellung auf Websiten.
    % \item Sie bietet eine allgemeine, abstrakte Grammatik an, die sogenannte \textit{Dot}-Sprache, mit deren Hilfe verschiedene Typen von Graphen erstellt werden können. 
    % Die \textit{Dot}-Sprache ist, ähnlich zu anderen Formaten wie JSON oder YAML, zu einem Standard geworden, der auch von anderen Werkzeugen und Software verarbeitet werden kann.
    % \item Ein in der \textit{Dot}-Sprache beschriebener EGraph kann mit \textit{Graphviz} in gängige Formate wie SVG, PNG oder PDF exportiert werden.
    % \item Es gibt bereits ein Python-Paket, was nativen Anschluss an \textit{Graphviz} ermöglicht. Damit kann auf das Interagieren mit \textit{Graphviz} über ein CLI verzichtet werden.
    % \item Durch die JavaScript-Bibliothek \textit{D3}~\footnote{\hspace{1.5mm}\url{https://d3js.org/}} und einer entsprechenden Bibliothek (\textit{d3-graphviz}~\footnote{\hspace{1.5mm}\url{https://github.com/magjac/d3-graphviz}}), kann ein EGraph auch im Browser als SVG-Datei gerendert werden.
\end{itemize}

\subsection{FastAPI}

Für die Interaktion mit dem Benutzer stehen eine Vielzahl von Optionen zur Verfügung, zum Beispiel eine CLI, über eine Website im Browser oder durch ein Anwendungsfenster.
Im Rahmen dieser Arbeit erschien eine Website, die lokal im Browser läuft, am sinnvollsten. Diese Entscheidung beruht auf folgenden Argumenten:

% Die Interaktion mit einem Benutzer kann auf verschiedenen Wegen erfolgen, zum Beispiel durch eine CLI, über eine Website im Browser oder durch ein Anwendungsfenster.
% In dieser Arbeit stellt eine Website, die lokal im Browser läuft zum Einsatz. Diese Entscheidung ist mit folgenden Argumenten zustande gekommen. 

\begin{itemize}
    \item Es sind umfangreiche GUI-Bibliotheken für Python vorhanden, darunter PyQt, PySide oder TKinter. Jede von ihnen kommt jedoch mit plattformspezifischen Problemen und Einschränkungen.
    Darüber hinaus erscheint die Generierung von dynamischem Inhalt im Browser durch JavaScript (oder darauf basierenden Bibliotheken) deutlich einfacher. Auch die Eigenschaft moderner Browser,
    das SVG-Format problemos und ohne weitere Dependencies anzeigen zu können, hat sich als Vorteil herausgestellt.
    \item Eine Website benötigt einen Server im Backend. Auch hier stehen zahlreiche Frameworks zur Verfügung, darunter Flask, Django und FastAPI.
    \textit{Django} ist weit verbreitet und sehr beliebt, hat aber auch Nachteile. Unter anderem bietet das Framework ein umfangreiches Angebot an, das viele Anwendungsfälle abdeckt.
    Dadurch ist die Performance ein oft genannter Kritikpunkt. 
    Im Gegensatz dazu beinhaltet \textit{FastAPI}~\footnote{\hspace{1.5mm}\url{https://fastapi.tiangolo.com/}} nur essenzielle Funktionen, was das Aufsetzen eines Servers deutlich vereinfacht.
    % \item Es gibt umfangreiche GUI-Bibliotheken, wie zum Beispiel PyQt, PySide oder TKinter. Diese bringen jedoch plattformspezifische Probleme und Einschränkungen mit sich.
    % Außerdem ist die dynamische Generierung von Inhalten im Browser mit JavaScript (oder darauf basierenden Bibliotheken) deutlich einfacher.
    % \item Für einen Server im Backend gibt es ebenfalls zahlreiche Frameworks, darunter Flask, Django und FastAPI.
    % Gerade \textit{Django} ist zwar sehr beliebt und weit verbreitet, hat aber auch Performance-Schwierigkeiten.
    % Das liegt unter anderem daran, dass \textit{Django} ein umfangreiches Angebot anbietet und für viele Anwendungsfälle entwickelt wurde.
    % Für dieses Projekt werden die meisten Features jedoch nicht benötigt.
    % \item Im Gegensatz zu \textit{Django} bietet \textit{FastAPI}~\footnote{\hspace{1.5mm}\url{https://fastapi.tiangolo.com/}} nur essenzielle Funktionen an, was das Aufsetzen eines Servers deutlich vereinfacht.
\end{itemize}

\subsection{Eingesetzte Frameworks}

Die Gestaltung der Benutzeroberfläche basiert auf dem CSS-Framework \textit{Bootstrap}~\cite{bootstrap} in der Version \textit{5.3.3}.
\textit{Bootstrap} ist leicht zu integrieren und vielfach erprobt worden. Dazu bietet es bereits fertige und optimierte Bausteine, die für Konsistenz sorgen und auf Accessibility getestet sind.

Die Dokumentation wurde mit \textit{Docusaurus}~\cite{docusaurus} in der Version \textit{3.7} erstellt. \textit{Docusaurus} ist ein Static-Site Generator, der sich leicht in
bestehende Infrastruktur einfügen lässt. Er unterstützt nativ Internationalisierung und die Dokumentation kann in Markdown geschrieben werden. Damit eignet sich \textit{Docusaurus} für diese Arbeit.

Für die Kommunikation zwischen Server und Benutzeroberfläche wurde kein Framework gewählt. Stattdessen wurden die notwendigen Funktionen von Hand geschrieben. Dies erlaubt eine bessere Kontrolle sowie die
Reduzierung von unnötigem Code. Zusätzlich sind die meisten JavaScript-Frameworks zu umfangreich für den Zweck der Arbeit und eines der vielen Frameworks auszuwählen ist zeitintensiv. Daher wurde darauf verzichtet.

Beim Testen der Anwendung wurde auf \textit{pytest}~\footnote{\hspace{1.5mm}\url{https://docs.pytest.org/}} in der Version \textit{8.3.3} gesetzt.
\textit{pytest} ist ein sehr bekanntes und verbreitetes Testing-Framework, was sich vor allem durch seine Unkompliziertheit auszeichnet.
Des Weiteren steht \textit{pytest} auch für das Testen von FastAPI-Funktionalität zur Verfügung, was das Testen simplifiziert. 

\subsection{JSON-Format}

Für den Transport von Daten zwischen Server und Benutzeroberfläche stehen verschiedene Formate zur Verfügung. In dieser Arbeit wurde das \textit{JSON}-Format eingesetzt,
weil es sowohl von Python als auch von JavaScript nativ unterstützt wird. Daten im \textit{JSON}-Format können in Python ohne großen Aufwand in ein Dictionary und in JavaScript in ein Objekt umgewandelt werden.
Zusätzlich eignet sich dieses Format zur Darstellung in Browsern, zum Beispiel in den Entwicklungswerkzeugen der meisten Browser, was das Debugging vereinfacht.

% Für die Kommunikation zwischen Server und Benutzeroberfläche stehen verschiedene Formate zur Verfügung. Die Entscheidung für das \textit{JSON}-Format wurde getroffen, 
% da es sowohl von JavaScript als auch von Python nativ unterstützt wird. Außerdem ist es eines der am häufigsten verwendeten Formate in Browsern.
% Zudem können die in diesem Format kodierten Daten im Browser dargestellt werden, unter anderem
% zum Beispiel durch die Developer Tools von Firefox. Dieses Feature macht das Debugging deutlich einfacher. 


\section{Architektur der Anwendung}\label{sec:architektur}

Die im letzten Kapitel vorgestellten theoretischen Konzepte sollen nun in die Praxis umgesetzt werden.
Dazu wird in diesem Kapitel die Architektur hinter der Anwendung erläutert sowie die Funktionsweise der Komponenten erklärt.

\subsection{Grundlegende Komponenten}

Die Anwendung kann in vier Hauptkomponenten eingeteilt werden: \textit{EGraph}, \textit{Service}, \textit{Server} und \textit{Weboberfläche}.
Abbildung~\ref{fig:architektur} zeigt den Zusammenhang zwischen diesen Komponenten. 


% Die Anwendung besteht aus vier Hauptkomponenten: der Klasse \textit{EGraph} mit ihren Hilfsklassen, dem \textit{EGraphService}, dem Server (\textit{server.py}) und der
% Weboberfläche (\textit{index.html}). Abbildung~\ref{fig:architektur} gibt einen Überblick über den Zusammenhang zwischen den Komponenten. 

Die erste Komponente (\textit{E-Graph}) ist für die Erstellung und Verwaltung der E-Graph Datenstruktur zuständig. Dafür existieren sechs Klassen:
\textit{AbstractSyntaxTreeNode}, \textit{AbstractSyntaxTree}, \textit{RewriteRule}, \textit{ENode}, \textit{EClass} und \textit{EGraph}.
Die Klasse \textit{EGraph} enthält die Datenstruktur sowie einige weitere Funktionen, wie zum Beispiel die Visualisierung dieser.
Sie ist abhängig von vier weiteren Klassen. Die beiden Klassen \textit{ENode} und \textit{EClass} repräsentieren \textit{E-Nodes} und \textit{E-Classes}.
Die Klasse \textit{RewriteRule} verkörpert eine \textit{rewrite rule}, die aus einem linken und einem rechten Ausdruck besteht. Beide Ausdrücke werden in einen AST umgewandelt, 
um das Anwenden der Regel auf den E-Graph zu vereinfachen.
Die letzte Abhängigkeit besteht von der Klasse \textit{AbstractSyntaxTree}. Sie stellt einen AST (\textbf{Abstract Syntax Tree}) dar und benötigt dafür eine Klasse, 
die die Knoten repräsentiert (\textit{AbstractSyntaxTreeNode}).
Ein mathematischer Ausdruck, aus dem ein E-Graph entstehen soll, wird zuerst in einen AST umgewandelt. Dies vereinfacht später die Erstellung des E-Graphs.


% Die erste Komponente bildet die Klasse \textit{EGraph}. Instanzen dieser Klasse implementieren die Datenstruktur E-Graph. Dazu sind fünf weitere Klassen nötig: 
% \textit{AbstractSyntaxTreeNode}, \textit{AbstractSyntaxTree}, \textit{RewriteRule}, \textit{ENode} und \textit{EClass}.

% Die Klasse \textit{AbstractSyntaxTree} stellt einen AST (\textbf{Abstract Syntax Tree}) dar und benötigt dafür eine Klasse, die die Knoten repräsentiert (\textit{AbstractSyntaxTreeNode}).
% Ein mathematischer Ausdruck, aus dem ein E-Graph entstehen soll, wird zuerst in einen AST umgewandelt, was später die Erstellung des E-Graphs vereinfacht.

% Die Klasse \textit{RewriteRule} verkörpert eine \textit{rewrite rule}, die aus einem linken und einem rechten Ausdruck besteht. Beide Ausdrücke werden in einen AST umgewandelt, um das
% Anwenden der Regel auf den E-Graph zu vereinfachen.

% Die beiden Klassen \textit{ENode} und \textit{EClass} fungieren als Datencontainer, auf die der E-Graph aufbaut.


Die zweite Komponente (\textit{Service}) ist für die Interaktion zwischen dem Server und der Datenstruktur verantwortlich. Die zuständige Klasse heißt \textit{EGraphService}. 
Von dieser wird eine Instanz beim Start des Servers erzeugt. 
Sie stellt dem Server Funktionen, wie zum Beispiel das Debugging, zur Verfügung und regelt die Sicherheitsüberprüfungen für Eingaben und Zustände. 


% Die zweite Komponente stellt die Klasse \textit{EGraphService} dar. Über sie läuft die Interaktion zwischen dem Server und einer EGraph-Instanz.
% Außerdem regelt sie die Sicherheitsüberprüfungen der Eingabe, z.B. ob ein empfangener Ausdruck vom Server die entsprechende Form aufweist.
% Neben diesen Aufgaben ist der Service auch für das Debugging-Feature zuständig. Dieses Feature erlaubt es dem Benutzer, einen E-Graph während verschiedener Operationen
% zu beobachten und Informationen über die Vorgehensweise zu erhalten. 

Die dritte Komponente (\textit{Server}) stellt das Backend der Anwendung dar. Sie enthält eine Datei \textit{server.py}, in der ein mit FastAPI erstellter Webserver
Anfragen der Weboberfläche entgegennimmt und zur Umsetzung der Anfragen mit dem Service interagiert. Da zwischen dem Server und der Weboberfläche Daten ausgetauscht werden
müssen, kommunizieren sie über das HTTP-Protokoll, während die Daten im JSON-Format übertragen werden.

% Die dritte Komponente ist der Server (\textit{server.py}). Er bildet die Verbindung zwischen dem Service und der Weboberfläche. Für die Umsetzung wurde auf das Framework 
% \textit{FastAPI} gesetzt (siehe Kapitel~\ref{sec:entscheidungen}). 

Die vierte Komponente (\textit{Weboberfläche}) versieht die Anwendung mit einer Benutzeroberfläche, einer Dokumentation sowie der nötigen Kommunikation mit dem Server.
Die Benutzeroberfläche besteht aus einer Website (\textit{index.html}), die lokal im Browser läuft. Die Website erhält ihre Funktionalität durch die Funktionen in der 
Datei \textit{index.js}. Diese Datei schickt Anfragen und verarbeitet Antworten, rendert den E-Graph und generiert Inhalte auf der Website. 
Die Dokumentation wurde separat mit \textit{Docusaurus} erstellt und erklärt die Installation der Anwendung, die Ausführung der Tests und die Bedienung der Benutzeroberfläche.  

% Die vierte Komponente bildet die Weboberfläche (\textit{index.html}). Sie läuft lokal im Browser und dient zur Steuerung der Anwendung. Mittels JavaScript können Anfragen an den
% Server abgeschickt und Antworten verarbeitet werden.


\begin{figure}[H]
  \centering
  \includegraphics[scale=0.6]{../fig/components.png}
  \caption{Architekturdiagramm der Anwendung}
  \label{fig:architektur}
\end{figure}

\subsection{Kommunikation zwischen Server und Weboberfläche}

% In diesem Abschnitt

In diesem Abschnitt wird die Kommunikation zwischen Server, Weboberfläche und Benutzer stärker beleuchtet.
Hierzu zeigt Abbildung~\ref{fig:ablauf} die leicht vereinfachte Erstellung eines E-Graphs durch den Benutzer.

% Abbildung~\ref{fig:ablauf} zeigt eine beispielhafte Interaktion zwischen Server, Weboberfläche und Benutzer.

Die Interaktion startet mit der Eingabe eines Ausdrucks und dem Klicken auf den entsprechenden Button (Abb.~\ref{fig:ablauf}, Schritt 1) durch den Benutzer.
Das triggert die in JavaScript geschriebene Funktion \textit{create()}. 
Die Funktion \textit{create()} überprüft zuerst, ob der Service bereits einen E-Graph geladen hat (Abb.~\ref{fig:ablauf}, Schritt 2-3). Wenn dies der Fall ist, wird der Benutzer
gefragt, ob er diesen und alle mit ihm verbundenen Daten löschen möchte. Im Beispiel ist dieser Schritt nicht gezeigt.
Wenn der Benutzer dies bestätigt oder noch kein E-Graph im Service geladen ist, wird die Funktion \textit{createEGraph()} aufgerufen. 
Sie schickt per \textit{POST}-Request den Inhalt des Eingabefeldes an den Server (Abb.~\ref{fig:ablauf}, Schritt 4), der sich um die Erstellung kümmert.
Nach der Erstellung gibt die Antwort des Servers (Abb.~\ref{fig:ablauf}, Schritt 5) dem Benutzer Auskunft darüber (Abb.~\ref{fig:ablauf}, Schritt 6), ob das Erstellen des E-Graphs erfolgreich war.
Im nächsten Schritt muss die Benutzeroberfläche aktualisiert werden, um den E-Graph und die zugehörigen \textit{rewrite rules} darzustellen.
Dafür ist die Funktion \textit{loadData()} verantwortlich, die auch bei einem erneuten Laden der Website dafür sorgt, dass die Inhalte weiterhin angezeigt werden.
In zwei Schritten lädt sie einmal den E-Graph und danach die \textit{rewrite rules}.
Zunächst wird also eine \textit{GET}-Anfrage an den Server geschickt (Abb.~\ref{fig:ablauf}, Schritt 7), deren Antwort einen Status und die angeforderten Daten beinhaltet (Abb.~\ref{fig:ablauf}, Schritt 8).
In den Daten befindet sich der entsprechende E-Graph im \textit{DOT}-Format, der sogleich als SVG gerendert und auf 
der Weboberfläche angezeigt wird (Abb.~\ref{fig:ablauf}, Schritt 9).
Abschließend werden die Regeln durch eine weitere \textit{GET}-Anfrage geladen (Abb.~\ref{fig:ablauf}, Schritt 10-11). Zuletzt werden sie auf der Benutzeroberfläche gerendert (Abb.~\ref{fig:ablauf}, Schritt 12).

% Die Funktion überprüft zuerst, ob es bereits einen E-Graph gibt (Abb., Schritt 2-3) und fragt, ob der Benutzer
% diesen und die mit ihm verbundenen Daten löschen will (im Beispiel nicht gezeigt). 
% Falls noch kein E-Graph vorhanden ist, wird nun die Funktion \textit{createEGraph()} aufgerufen. 
% Sie schickt den Ausdruck an den Server (Abb., Schritt 4) und dieser kümmert sich um die Erstellung. Die Antwort des Servers (Abb., Schritt 5) gibt dem Benutzer 
% Auskunft (Abb., Schritt 6), ob dieser Schritt erfolgreich war. 
% Anschließend muss der E-Graph und zugehörige \textit{rewrite rules} geladen werden, was durch die Funktion \textit{loadData()} passiert.
% Sie schickt eine Anfrage (Abb., Schritt 7), deren Antwort einen Status und die Daten enthält (Abb., Schritt 8). 
% In den Daten befindet sich der entsprechende E-Graph im 
% \textit{DOT}-Format (siehe Kapitel~\ref{sec:entscheidungen}), der sogleich als SVG gerendert und auf der Weboberfläche angezeigt wird (Abb., Schritt 9).
% In der nächsten Anfrage werden die Regeln geladen (Abb., Schritt 10-11) und anschließend auf der Weboberfläche dargestellt (Abb., Schritt 12).

Insgesamt spiegelt die in Abbildung~\ref{fig:ablauf} dargestellte Vorgehensweise den Kommunikationsverlauf bei anderen Funktionen wider.
Dabei wurde beabsichtigt, die Kommunikation mit dem Server möglichst gering und effizient zu halten. Folglich konnte auf ein JavaScript-Framework verzichtet werden (siehe Kapitel~\ref{sec:entscheidungen}).

\begin{figure}[H]
  \centering
  \includegraphics[scale=0.6]{../fig/query.png}
  \caption{Ablaufdiagramm der Kommunikation für die Erstellung eines E-Graphs zwischen Server, Weboberfläche und Benutzer}
  \label{fig:ablauf}
\end{figure}


\section{Entwicklung der Anwendung}\label{sec:entwicklung}

In Kapitel~\ref{sec:grundlagen} und~\ref{sec:architektur} wurden die notwendigen Grundlagen und die Architektur vermittelt.
Darauf aufbauend wird in den folgenden Abschnitten ein tieferer Einblick in die Entwicklungsphase vorgenommen und dazu passend Teile des Codes vorgestellt.
Dazu ist dieses Kapitel in vier Abschnitte gegliedert, in denen jeweils eine Komponente der Anwendung behandelt wird.

\subsection{Erste Komponente: E-Graph}

Die Erstellung eines E-Graphs beginnt mit einem mathematischen Ausdruck, aus dem ein \textbf{Abstract Syntax Tree} (AST) aufgebaut wird.
Der AST wird später traversiert, um jeden Knoten dem E-Graph hinzuzufügen. Dieser Zwischenschritt vereinfacht die Erzeugung von E-Graphs und
kann gleichzeitig als Validierungsphase genutzt werden, da hier bereits feststeht, ob der Ausdruck der korrekten Form entspricht.

\subsubsection{AbstractSyntaxTree \& AbstractSyntaxTreeNode}

Ein AST erfordert die Klassen \textit{AbstractSyntaxTreeNode} (Listing~\ref{lst:astnode}) und 
\textit{AbstractSyntaxTree} (Listing~\ref{lst:ast}). Die Klasse \textit{AbstractSyntaxTreeNode} repräsentiert
einen Knoten im AST. Sie hat drei Attribute: einen linken Zeiger, einen rechten Zeiger und einen Schlüssel.

\begin{lstlisting}[language=Python, caption=Klasse \textit{AbstractSyntaxTreeNode}, label={lst:astnode}]
class AbstractSyntaxTreeNode:
    def __init__(self):
        self.left = None
        self.key = str()
        self.right = None
\end{lstlisting}

Der Schlüssel kann eine arithmetische Operation, eine Variable oder eine Zahl sein. Die Zeiger können auf einen weiteren Teilbaum zeigen, auf einen einzelnen Knoten
oder auf nichts. Mithilfe dieser Klasse kann nachher ein ganzer Baum dargestellt werden. Dazu braucht man nur den Wurzelknoten des Baumes abspeichern, um über dessen Kinder
Zugriff auf den gesamten Baum zu bekommen. 

Über die Klasse \textit{AbstractSyntaxTree} kann aus einem Ausdruck ein AST erstellt werden.
Entsprechend wird die Klasse mit einem Ausdruck initialisiert, der direkt durch die Methode \textit{\_process\_expression(self, expression)} verarbeitet wird.
Die Methode gibt den Wurzelknoten des erstellten AST zurück, der im Attribut \textit{root\_node} der Klasse abgespeichert wird.
Anschließend wird der AST in ein String-Format gebracht und im zweiten Attribut \textit{string\_representation} abgespeichert.
Das Umwandeln eines AST in einen String übernimmt die Methode~(Z. \ref{astpreorder}) \textit{\_preorder(self, ast\_node)}. Sie startet mit dem Wurzelknoten (\textit{root\_node})
und durchläuft den AST in Preorder-Traversierung. Dabei werden die Schlüssel der Knoten an einen String angehangen. Zusätzlich wird eine öffnende Klammer
angehangen, wenn ein Knoten zwei Kinder hat und eine schließende Klammer, wenn beide Kinder bearbeitet wurden. Dadurch kann der eingegebene Ausdruck zurückerhalten werden, falls
dieser dem korrekten Format entsprach.

\begin{lstlisting}[language=Python, escapechar=|, caption=Auszug aus der Klasse \textit{AbstractSyntaxTree}, label={lst:ast}]
class AbstractSyntaxTree:

    # ... __init__ und __str__  weggelassen 

    def _preorder(self, ast_node): |\label{astpreorder}|
        # ...
        self._preorder(ast_node.left)
        self._preorder(ast_node.right)
        # ...

    def _process_expression(self, expression): |\label{astprocess}|
        root_ast_node = None
        stack = deque()
        word = ""
        for character in expression:
            if character == "(": |\label{ast0}|
                if not stack:
                    ast_node = AbstractSyntaxTreeNode()
                    stack.append(ast_node)
                    root_ast_node = ast_node
                else:
                    last_ast_node = stack[-1]
                    ast_node = AbstractSyntaxTreeNode()
                    # ...
                    stack.append(ast_node)
            elif character in ("/", "*", "+", "-"): |\label{ast1}|
                last_ast_node = stack[-1]
                last_ast_node.key = character
            elif (word == "<" or word == ">") and (character == "<" or character == ">"): |\label{ast2}|
                word += character
                last_ast_node = stack[-1]
                last_ast_node.key = word
                word = ""
            elif (character == " " or character == ')') and word != "": |\label{ast3}|
                last_ast_node = stack[-1]
                # ...
                if character == ")":
                    stack.pop()
                word = ""
            elif character == ")" and word == "": |\label{ast4}|
                stack.pop()
            elif character == " ": |\label{ast5}|
                pass
            else: |\label{ast6}|
                word += character
        return root_ast_node
\end{lstlisting}

Die Methode~(Z. \ref{astprocess}) \textit{\_process\_expression(self, expression)} initialisiert drei lokale Variablen.
Die erste Variable \textit{root\_ast\_node} speichert den Wurzelknoten des Baumes, den die Methode später zurückgibt. Die zweite Variable \textit{stack} speichert an der 
Position \textbf{Top of Stack} (TOS) den aktuell zu bearbeitenden Knoten. Die dritte Variable \textit{word} ist ein leerer String, der als kurzzeitiger Zwischenspeicher verwendet wird,
wenn Variablen, Zahlen oder Operationen aus mehreren Zeichen bestehen.
Der String wird von links nach rechts Zeichen für Zeichen verarbeitet, wobei sieben Fälle unterschieden werden.

Fall eins~(Z. \ref{ast0}) ist gegeben, wenn eine öffnende Klammer eingelesen wird. Ein neuer Knoten wird erstellt, der sogleich auf einen Stack geschoben wird. Falls der Stack leer ist, wird
dieser Knoten als Wurzelknoten betrachtet und in \textit{root\_ast\_node} gespeichert.
Nach einer öffnenden Klammer wird eine arithmetische Operation erwartet. 
Daher behandelt Fall zwei~(Z. \ref{ast1}) die vier Operationen ($+, *, -, /$), indem die jeweilige Operation dem Knoten, der gerade TOS ist,
als Schlüssel zugewiesen wird. Da Shift-Operationen ($<<, >>$) zwei Zeichen lang sind, werden sie in Fall drei~(Z. \ref{ast2}) behandelt.
Zu Fall drei kommt es, wenn in \textit{word} bereits eines der Shift-Zeichen ($<, >$) eingelesen wurde und nun das nächste eingelesen wird.
Das Zeichen wird \textit{word} hinzugefügt und dem TOS-Knoten als Schlüssel übergeben.

Im vierten Fall~(Z. \ref{ast3}) werden Variablen und Zahlen verarbeitet und eventuell auch Knoten vom Stack genommen.
Zur Trennung von Variablen und Zahlen wird ein Leerzeichen erwartet und ein nichtleeres \textit{word}.
Dabei werden nochmal vier Fälle unterschieden, je nach Zustand des aktuellen TOS-Knotens. Dementsprechend wird entweder ein neuer Knoten mit dem Inhalt von \textit{word} als Schlüssel
erzeugt und links oder rechts angehängt oder der aktuelle TOS-Knoten hat noch kein Schlüssel und bekommt entsprechend einen. 
Wenn jedoch das Zeichen ein schließende Klammer ist, wird \textit{word} geleert und der TOS-Knoten vom Stack genommen.
Fall fünf~(Z. \ref{ast4}) behandelt einen ähnlichen Fall. Hier ist \textit{word} jedoch leer. Der TOS-Knoten wird ebenfalls vom Stack genommen.
Fall sechs~(Z. \ref{ast5}) wird nur behandelt, wenn das aktuelle Zeichen ein Leerzeichen ist. Diese werden ignoriert.
Der letzte Fall~(Z. \ref{ast6}) ist gegeben, wenn kein anderer Fall zutrifft. Dabei wird das aktuelle Zeichen an \textit{word} angehangen.

\subsubsection{ENode}

Die Implementierung einer \textit{E-Node} kann ohne weitere Modifikationen implementiert werden. Hierzu dient die Klasse \textit{ENode} (Listing~\ref{lst:enode}).
Ihre werden zwei Parameter übergeben, ein Schlüssel und eine Liste von EClass-IDs als Strings, die sie als ihre Attribute speichert.

\begin{lstlisting}[language=Python, caption=Klasse \textit{ENode}, label={lst:enode}]
class ENode:
    def __init__(self, key, arguments):
        self.key = key
        self.arguments = arguments
\end{lstlisting}

\subsubsection{EClass}

Für Erzeugung von \textit{E-Classes} ist die Klasse \textit{EClass} (Listing~\ref{lst:eclass}) verantwortlich. Durch die Initialisierung wird eine UUID als die EClass-ID generiert 
und im Attribut \textit{id} gespeichert. Zusätzlich werden zwei Sets erzeugt, eines für die E-Nodes, die die Klasse enthält, und eines für die Elternklassen.

\begin{lstlisting}[language=Python, caption=Klasse \textit{EClass}, label={lst:eclass}]
# ... 

class EClass:
    def __init__(self):
        self.id = str(uuid.uuid4())
        self.nodes = set()
        self.parents = set()
\end{lstlisting}

\subsubsection{RewriteRule}

Die Klasse \textit{RewriteRule} (Listing~\ref{lst:rr}) dient als Repräsentant einer \textit{rewrite rule}. Als Parameter bekommt sie einen Namen der Regel, einen linken sowie einen 
rechten Ausdruck übergeben. Die beiden Ausdrücke werden in AST umgewandelt. Die Attribute \textit{expr\_lhs} und \textit{expr\_rhs} speichern deshalb nur die Wurzelknoten der ASTs.

\begin{lstlisting}[language=Python, caption=Klasse \textit{RewriteRule}, label={lst:rr}]
# ... 

class RewriteRule:
    def __init__(self, name, expr_lhs, expr_rhs):
        self.name = name
        self.expr_lhs = AbstractSyntaxTree.AbstractSyntaxTree(expr_lhs)
        self.expr_rhs = AbstractSyntaxTree.AbstractSyntaxTree(expr_rhs)

    # ... 
\end{lstlisting}

\subsubsection{EGraph}

Mit den bereits definierten Klassen kann jetzt die wichtigste Klasse der Komponente definiert werden. Die Init-Methode der Klasse \textit{EGraph} (Listing~\ref{lst:egraph})
legt eine Union-Find Datenstruktur \textit{u} und zwei Dictionaries \textit{m} und \textit{h} an. Die weiteren vier Attribute werden erst im Verlauf der Entwicklung wichtig und 
an den entsprechenden Stellen erläutert.

\begin{lstlisting}[language=Python, caption=Auszug aus der Klasse \textit{EGraph}, label={lst:egraph}]
# ... 

class EGraph:
    def __init__(self):
        self.u = DisjointSet()
        self.m = {}
        self.h = {}
        self.pending = []
        self.version = 0
        self.str_repr = ""
        self.is_saturated = False

    # ...
\end{lstlisting} 

Die nächsten sieben Methoden statten den E-Graph mit Basisfunktionalität aus.

\begin{lstlisting}[language=Python, caption=Auszug aus der Klasse \textit{EGraph}, label={lst:methods}]
def _add(self, enode):
    enode = self._canonicalize(enode)
    # ... 
    elif enode in self.h.keys():
        return self.h[enode]
    # ... 
    else:
        self.version += 1
        eclass_id = self._new_singleton_eclass(enode)
        for child in enode.arguments:
            self.m[child].parents.add((enode, eclass_id))
        self.h[enode] = eclass_id
        return eclass_id
\end{lstlisting} 

\begin{lstlisting}[language=Python, caption=Auszug aus der Klasse \textit{EGraph}, label={lst:methods}]
def add_node(self, ast_node):
    if ast_node is not None:
        if ast_node.left is not None and ast_node.right is not None:
            return self._add(ENode(ast_node.key, [self.add_node(ast_node.left), self.add_node(ast_node.right)],))
        elif ast_node.left is not None:
            return self._add(ENode(ast_node.key, [self.add_node(ast_node.left)]))
        elif ast_node.right is not None:
            return self._add(ENode(ast_node.key, [self.add_node(ast_node.right)]))
        else:
            return self._add(ENode(ast_node.key, []))
\end{lstlisting} 

% Bis jetzt hat die Klasse \textit{EGraph} noch keine Funktionalität, um tatsächlich ihre Aufgabe zu erfüllen. Die folgenden zwei Abschnitte 
% behandeln zwei Quellen, mit deren Hilfe diese Funktionalität erreicht werden konnte.

% \subsubsection{egg-Implementierung}\label{subsub:egg}

% In der Einleitung (Abschnitt \ref{sub:verwandtearbeiten}) wurde bereits die in Rust geschriebene Bibliothek \textbf{egg} vorgestellt, die auf einem Paper (\cite{2021-egg}) basiert.
% Für diese Implementierung wurden verschiedene, grundlegende Methoden übernommen, darunter \textit{add}, \textit{merge}, \textit{repair}, \textit{rebuild}, \textit{canonicalize} und
% \textit{find}. Abgesehen von einigen wenigen Änderungen wurden die Methoden aus dem Paper weitestgehend übernommen.



% \subsubsection{Colab-Implementierung}\label{subsub:colab}

% Das erwähnte Paper behandelt die grundlegenden Funktionen eines E-Graphs. Allerdings werden andere Aspekte nicht beleuchtet. Hierzu wurde auf ein Google Colab Notebook zurückgegriffen \cite{devito},
% welches weiterführende Konzepte aufzeigt.



\subsection{Zweite Komponente: Service}

% Für die Interaktion zwischen E-Graphs und Benutzer sind zwei Komponenten wichtig: ein Server, der Anfragen der Benutzeroberfläche beantwortet sowie ein Service, der im Hintergrund
% die Erstellung und Benutzung von E-Graphs regelt.


\begin{lstlisting}[language=Python, caption=Auszug aus der Datei \textit{EGraphService.py}]
# ... 

class EGraphService:
    def __init__(self):
        self.rrc = 0
        self.dict_of_rules = {}
        self.applied_rules = set()
        self.egraph = None
        self.expr = None
        self.egraphs = [[]]
        self.current_major = 0
        self.current_minor = 0

# ... 
\end{lstlisting} 





\begin{lstlisting}[language=Python, caption=Methoden für das Debugging aus der Datei \textit{EGraphService.py}]
# ... 

    def move_backward(self):
        if self.current_minor == 0:
            if self.current_major == 0:
                pass
            else:
                self.current_major -= 1
                self.current_minor = len(self.egraphs[self.current_major]) - 1
        else:
            self.current_minor -= 1

    def move_forward(self):
        if len(self.egraphs[self.current_major]) - 1 == self.current_minor:
            if self.current_major == len(self.egraphs) - 1:
                pass
            else:
                self.current_minor = 0
                self.current_major += 1
        else:
            self.current_minor += 1

    def move_fastbackward(self):
        if self.current_major != 0:
            self.current_major -= 1
            self.current_minor = len(self.egraphs[self.current_major]) - 1

    def move_fastforward(self):
        if self.current_major != len(self.egraphs) - 1:
            self.current_major += 1
            self.current_minor = len(self.egraphs[self.current_major]) - 1

# ... 
\end{lstlisting} 



\subsection{Dritte Komponente: Server}

% Die dritte Komponente (Server ) stellt das Backend der Anwendung dar. Sie enthält ei-
% ne Datei server.py, in der ein mit FastAPI erstellter Webserver Anfragen der Webober-
% fläche entgegennimmt und zur Umsetzung der Anfragen mit den Service interagiert. Da
% zwischen dem Server und der Weboberfläche Daten ausgetauscht werden müssen, kom-
% munizieren sie über das HTTP-Protokoll, während die Daten im JSON-Format übertragen
% werden.

Für die Kommunikation zwischen den beiden Komponenten Service und Weboberfläche ist die Komponente Server verantwortlich.
In der Datei \textit{server.py} (Listing~\ref{lst:server}) ist ein FastAPI-Server implementiert.
Der Server beginnt mit der Initialisierung einer Instanz in Z.~\ref{serverinit}. Dabei wird die Funktion \textit{lifespan} aufgerufen, die beim Starten und Herunterfahren des
Servers nutzerdefinierte Aktionen ausführen kann. In diesem Fall wird der Browser mit der Adresse des Servers geöffnet (Z.~\ref{serveropen}).
Als nächstes wird eine Instanz des EGraphService erzeugt (Z.~\ref{serverservice}).

\begin{lstlisting}[language=Python, escapechar=|, caption=Auszug aus der Datei \textit{server.py}, label={lst:server}]
# ... 

@asynccontextmanager
async def lifespan(app: FastAPI):
    open_new(r"http://127.0.0.1:8000") |\label{serveropen}|
    yield

app = FastAPI(lifespan=lifespan) |\label{serverinit}|
egraphService = EGraphService() |\label{serverservice}|

@app.get("/getrules")
def get_rules():
    result, msg, data = egraphService.get_all_rules()
    return {"response": str(result), "msg": msg, "payload": data}

@app.post("/addrule")
async def add_rule(request: Request):
    payload = await request.body()
    json_data = json.loads(payload)
    result, msg = egraphService.add_rule(json_data["lhs"], json_data["rhs"])
    return {"response": str(result), "msg": msg}

# ... 

app.mount("/", StaticFiles(directory=realpath(f"{realpath(__file__)}/../static"), html=True), name="static")
\end{lstlisting} 


\subsection{Vierte Komponente: Weboberfläche}


\subsubsection{Benutzeroberfläche}

% \afterpage{\clearpage}
% \begin{sidewaysfigure}[h!]
% % \begin{figure}[H]
% \centering
% \includegraphics[scale=0.45]{../fig/website.png}
% \caption{Website beim Start des Servers}
% \label{fig:website}
% % \end{figure}
% \end{sidewaysfigure}

\newpage
\begin{figure}[H]
\centering
\includegraphics[scale=0.42, angle=90]{../fig/website.png}
\caption{Website beim Start des Servers}
\label{fig:website}
\end{figure}
\newpage


\subsubsection{JavaScript}

\subsubsection{Dokumentation}





% Die Benutzeroberfläche besteht aus einer Website, die im Browser läuft. Näheres zu dieser Entscheidung kann in Kapitel \ref{sec:entscheidungenundprobleme} nachgelesen werden.





\section{Reflexion der Arbeit}\label{sec:reflexion}

Im vorliegenden Kapitel wird eine kritische Reflexion der Arbeit vorgenommen. 
Der Fokus der Untersuchung liegt dabei auf der Arbeit an sich und auf der Anwendung.
Dementsprechend beschäftigt sich der erste Abschnitt mit der Frage, inwiefern das Ziel der Bachelorarbeit erreicht wurde.
Der zweite Abschnitt thematisiert die Qualität der entwickelten Software.

\subsection{Ziel der Bachelorarbeit}\label{sub:ziel}

Das Ziel wurde in der Einleitung bereits definiert. Dort heißt es: 
\glqq Das Ziel dieser Bachelorarbeit ist es, eine Anwendung für die Lehre zu entwickeln, die Studentinnen und Studenten Wissen zu den Themen \textit{E-Graphs} und \textit{Equality Saturation}
vermittelt. Dabei sollen sie die Möglichkeit haben, sich sowohl auf theoretischer als auch praktischer Ebene mit den Themen auseinander setzen zu können.
Die Grundlage der theoretischen Ebene bildet diese Arbeit, in der notwendige Hintergrundkenntnisse erarbeitet werden. Außerdem wird ein Einblick in die Implementierung gegeben. 
Die praktische Ebene besteht aus der Anwendung, mit deren Hilfe Schritt für Schritt aufgezeigt wird, wie \textit{E-Graphs} und \textit{Equality Saturation} funktionieren.
Für grö{\ss}tmöglichen Nutzen soll die Anwendung zudem plattformunabhängig sein und möglichst nur von \textit{Open-Source-Software} (OSS) Gebrauch machen [\ldots]\grqq 
(Kapitel~\ref{sec:einleitung}, Ziel der Arbeit).

Die Entwicklung einer Lernanwendung wurde in dieser Arbeit durchgeführt. Damit können sich Studentinnen und Studenten auf praktische Weise mit den Themen 
\textbf{E-Graphs} und \textbf{Equality Saturation} auseinander setzen. Kapitel~\ref{sec:grundlagen} dieser Arbeit kann genutzt werden, um sich mit den
theoretischen Grundlagen vertraut zu machen. 
Wie in Kapitel~\ref{sec:entscheidungen} bereits festgestellt wurde, macht die Anwendung nur von \textit{Open-Source-Software} Gebrauch. Außerdem ist
sie plattformunabhängig (siehe~\ref{softwarequalität}).
Als zusätzliches, objektives Maß für die Vollständigkeit kann das Exposé herangezogen werden.
Im Exposé~\cite{expose} wurden dabei folgende Funktionen definiert, die in die Anwendung integriert werden müssen.

Aus Kapitel~\ref{sec:entwicklung} geht hervor, dass diese Funktionen in der Arbeit umgesetzt werden konnten. Darüber hinaus wurden zwei weitere Feature-Requests eingearbeitet und 
die optionale Funktion, die Anwendung auch auf Englisch verfügbar zu machen, konnte umgesetzt werden.
Somit kann abschließend resümiert werden, dass das Ziel der Bachelorarbeit erreicht wurde.

\subsection{Qualität der Software}\label{softwarequalität}

Bei der Bewertung von Software kann ihre Qualität als zuverlässiger Faktor miteinbezogen werden. Qualität kann unter verschiedenen Gesichtspunkten gemessen werden und wird
unterschiedlich definiert. Eine gute Orientierung bietet der \textit{ISO/IEC 25010:2011 Standard}~\footnote{\hspace{1.5mm}\url{https://www.innoq.com/de/blog/2021/10/quality-value-chain-evolution/}}. 
In diesem sind folgende Qualitätsmerkmale aufgelistet:

\noindent\textbf{Übertragbarkeit} Die Anwendung läuft problemlos auf allen gängigen Plattformen. Das liegt erstens daran, dass nur Open-Source-Software eingesetzt wurde,
wodurch Schwierigkeiten beim Beschaffen von Lizenzen vermieden werden können. 
Zweitens basieren vier der fünf Dependencies auf Python und die fünfte, \textit{Grahviz}, wird für alle gängigen Plattformen angeboten. Da Python plattformunabhängig ist, gilt dies auch
für die Dependencies.
Drittens läuft die Benutzeroberfläche im Browser, der als Dependency vorausgesetzt werden kann.
Somit kann auf eine Plattformunabhängigkeit der Anwendung geschlossen werden.

\noindent\textbf{Wartbarkeit} Die Architektur der Anwendung folgt einem logischen Aufbau mit klar definierten Abhängigkeiten, was das Hinzufügen neuer Funktionen erlaubt.
Der Code wurde mit \textit{Black}~\cite{black} formatiert und richtet sich nach \textit{PEP 8}~\cite{pep}. Jede Funktion oder Methode, die in Python oder JavaScript geschrieben wurde, 
ist mit einem Kommentar versehen, der Parameter, Rückgabewerte und Zweck erläutert. Weiterhin ist die Anwendung mit X Unit-Tests und zahlreichen händischen Simulationen abgedeckt, die eine breites Spektrum an 
möglichen Eingaben und Zuständen simulieren.

\noindent\textbf{Sicherheit} Für das Ausführen der Anwendung werden keine erweiterten Berechtigungen des Nutzers eingefordert. Bei der Verwendung der Software werden keine persönlichen Daten gespeichert 
oder weitergegeben und die Kommunikation findet ausschließlich im lokalen Netzwerk statt.
Zusätzlich wurde der gesamte Code der Anwendung auf \textit{GitHub} veröffentlicht, wodurch eine unabhängige Prüfung auf Sicherheitslücken und mögliche, illegale Aktivitäten der Software vorgenommen werden kann.

\noindent\textbf{Zuverlässigkeit} Die Anwendung basiert auf einem publizierten Paper~\cite{2021-egg}, indem auch die Vorgehensweise für die Bibliothek \textit{egg} erklärt wird.
Aufgrund der Tatsache, dass E-Graphs basierend auf dem Paper implementiert wurden, lässt sich mit Gewissheit sagen, dass zumindest in den grundlegenden Funktionen kein logischer Fehler sein kann.



\noindent\textbf{Funktionale Eignung} In Abschnitt~\ref{sub:ziel} wurde bereits dargelegt, dass die im Exposé festgehaltenen Anforderungen an die Software erreicht worden sind.
Die Anwendung bietet Studentinnen und Studenten die Möglichkeit, sich auf interaktive Weise mit E-Graphs und Equality Saturation auseinander zu setzen.





\noindent\textbf{Performance} Bei der Entwicklung der Anwendung wurde das Ziel der Leistungseffizienz anerkannt, aber nicht prioritisiert, da die Software in der Lehre eingesetzt werden sollte
und somit andere Anforderungen wie zum Beispiel die Benutzbarkeit erfüllen muss. Dementsprechend ist die Wahl der Programmiersprache Python nicht optimal. Auch die Übertragung eines E-Graphs ins DOT-Format 
geschieht mit Aufwand. Trotzdem ist die Software in der Lage, Darstellungen flüssig zu rendern und dem Benutzer eine verzögerungsfreie Interaktion mit E-Graphs zu ermöglichen.

\noindent\textbf{Kompatibilität} Die Software setzt auf bekannte Standards wie dem JSON- und dem DOT-Format und kann daher, mit eventuellen Modifikationen, auch mit anderen Systemen Daten austauschen.
Außerdem ist die Kommunikation zwischen Server und Weboberfläche so gestaltet, dass auch andere Systeme wie zum Beispiel eine REST-API oder eine andere Benutzeroberfläche an die Anwendung anbinden könnten.

\noindent\textbf{Benutzbarkeit} Die Weboberfläche setzt auf eine Website, die mithilfe des Frameworks \textit{Bootstrap}~\cite{bootstrap} übersichtlich gestaltet werden konnte.
Über die Website lässt sich die Dokumentation abrufen, die als Bedienungsanleitung fungiert und dem Benutzer mit Beispielen behilflich ist.
Bei der Visualisierung von E-Graphs wurde besonders großen Wert auf die Benutzerfreundlichkeit gelegt. Nicht nur lassen sich andere Strukturen der Website für eine bessere Übersicht 
einklappen, sondern die Darstellung eines E-Graphs lässt sich vergrößern, verkleinern und herumschieben. Es ist auch möglich die Grafik in einem neuen Tab zu öffnen, damit auch Benutzer
mit einem kleineren Bildschirm die Anwendung nutzen können.
Des Weiteren kann ein E-Graph durch das Debugging-Feature in jedem Stadium beobachtet werden. Hilfreiche Statusmeldungen erklären dem Benutzer die Abläufe.
Es ist auch möglich, eine Session bzw. die \textit{rewrite rules} in einer Datei zu speichern und später hochzuladen, sodass dem Benutzer Zeit erspart wird.


\section{Fazit und Ausblick}\label{sec:fazit}

\subsection{Fazit}

Ziel der vorliegenden Arbeit war es, eine Lernsoftware für die Lehre zu entwickeln, um Studentinnen und Studenten die Möglichkeit zu geben, 
die theoretischen Grundlagen der Themen \textbf{E-Graphs} und \textbf{Equality Saturation} zu verstehen und sie auf interaktive Weise zu vertiefen.

Die entwickelte Anwendung 








\subsection{Ausblick}

Aufgrund der zeitlichen Beschränkungen konnten verschiedene Ansätze und Weiterentwicklungen nicht mehr ausprobiert und in die Anwendung integriert werden.
Nachfolgend sind mehrere Aspekte aufgelistet, die als Erweiterung denkbar wären. 

\textbf{Erweitertes Testing} Momentan ist das Testen von E-Graphs aufwendig, da auf die internen Datenstrukuren zugegriffen werden muss.
Denkbar wäre hier die Implementierung einer Vergleichsmethode, die für einen gegebenen Ausdruck die beiden E-Graphs vergleicht,
die mit der eigenen Implementierung und mit der von \textit{egg} (\cite{2021-egg}) erstellt wurden. 
Dadurch lässt sich die Qualität der Software einfacher überprüfen und Fehler können schneller erkannt werden.
Mit drei Wochen Zeit wäre zu rechnen.

\textbf{Erstellung von E-Graphs visualisieren} Die Visualisierungen setzen bei der Anwendung von \textit{rewrite rules} auf den E-Graph an.
Möglich wäre hier eine detailierte Schritt für Schritt Darstellung, die auch den Aufbau eines E-Graphs zeigt. Je nach Detailgrad könnte dies in ein bis zwei Wochen umgesetzt werden.

\textbf{Anbindung an egg} Die Anwendung arbeitet zurzeit mit einer eigenen Implementierung von E-Graphs. Dem Benutzer könnte eine weitere Option zur Verfügung stehen, eine weitere
Implementierung, zum Beispiel die von \textit{egg}~(~\cite{2021-egg}), als Basis zu nutzen.
Für dieses Unterfangen müsste jedoch der Debug-Output von \textit{egg} geparst und verarbeitet werden oder die Methoden der Bibliothek selbst müssten modifiziert werden.
Dabei kann der Umstand ausgenutzt werden, dass Methoden, die in \textit{Rust} geschrieben wurden, mit wenigen Modifikationen auch von \textit{Python} aus aufgerufen werden können.
Für dieses Vorhaben müsste wahrscheinlich sechs Wochen eingeplant werden.

\textbf{EClass Analysis}
Die Implementierung ist zurzeit nicht in der Lage eine \textit{EClass Analysis} durchzuführen. Bei dieser Art von Analyse kommen mehrere Techniken zum Einsatz, darunter
\textit{Conditional and Dynamic rewrites} und \textit{Constant Folding}. Zwei bis drei Wochen wären vermutlich für die Umsetzung nötig.



%% Dieser Part kann auskommentiert werden, sollte kein Anhang nötig sein
\appendix
\section{Installation der Anwendung}

\begin{enumerate}
  \item Das Repository des Projektes kann auf \textbf{GitHub} unter folgender Adresse erreicht werden: \url{https://github.com/BenSt099/Bachelorarbeit-EGraphs}.
  \item Unter der Sektion \textbf{Releases} die aktuelle Version herunterladen.
  \item Nach dem Entpacken bitte die notwendigen Dependencies mit folgendem Befehl installieren:
  
  \begin{verbatim}
    pip install -r requirements.txt
  \end{verbatim}

  \item Zum Starten der Anwendung ein Terminal im \textit{src}-Ordner öffnen und folgenden Befehl ausführen:

  \begin{verbatim}
      fastapi run server.py
  \end{verbatim}
  
  \item Falls sich der Browser nicht automatisch öffnet, kann die Anwendung unter folgender Adresse erreicht werden:
  \url{http://127.0.0.1:8000}
  \item Die Dokumentation kann im Browser in der Navigationsleiste oder unter der Adresse \url{http://127.0.0.1:8000/dokumentation.html} gefunden werden.
\end{enumerate}

%%%%%%%%%%%%%%%%%%%%%%%%%%%%%%%%%%%%%%%%%%%%%%%%%%%%%%%%%%%%%%%%%%%%%%%%%%%%%%%%
%% (Ende) Der Inhalt der Arbeit                                               %%
%%%%%%%%%%%%%%%%%%%%%%%%%%%%%%%%%%%%%%%%%%%%%%%%%%%%%%%%%%%%%%%%%%%%%%%%%%%%%%%%


\backmatter
\listoffigures
%\listoftables
\lstlistoflistings
\doublespacing
\listofalgorithms
\singlespacing

\clearpage
\bibliography{references}
%% Depending on Language, use german alphadin or original alpha
\iflanguage{ngerman}{
  \bibliographystyle{alphadin}
}{
  \bibliographystyle{alpha}
}

\end{document}
